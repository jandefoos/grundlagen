% erstellt am 2015-07-18 von Jan Dreyling-Eschweiler
\documentclass[pdftex, a4paper, 12pt, oneside]{scrbook}  % oneside skip one blank page before new chapter

% für Umlaute im Quelltext, ngerman und german haben unterschiedliche Trennungsregeln 
\usepackage[ngerman]{babel}
\usepackage[utf8]{inputenc}
\usepackage[T1]{fontenc}

\usepackage{amssymb,amsmath}
\usepackage[font=small,format=plain,labelfont=bf,up]{caption} % fuer kleinere caption

%\usepackage{subfigure}
\usepackage{subcaption}
\usepackage{wrapfig}
\usepackage{sidecap}

\usepackage{natbib}

\usepackage{array}
\usepackage{geometry}
\geometry{includehead,includefoot,inner=4cm,outer=3cm,top=2.5cm,bottom=2.5cm} % test layout

\usepackage{txfonts} % serif font
\setkomafont{sectioning}{\bfseries} % section als in serif

\usepackage{hyperref} % fuer links[hidelinks]
\usepackage{supertabular} % fuer tabellen ueber mehrere seiten
\usepackage{pbox}
\usepackage{sidecap}
\usepackage{multirow} % zum verbinden von tabellenzellen, vertikal
\usepackage{mathrsfs} % fuer \mathscr{L} 

%% pdftex 
%\pdfcompresslevel=1
%\pdfimageresolution=40

%\usepackage{mdwlist} % using “compacted” lists, \begin{itemize*}

\usepackage{enumitem} % also for compact list

\usepackage{setspace}

% drawings
\usepackage{tikz}
\usetikzlibrary{arrows}

%appendix
\usepackage[title,titletoc]{appendix}

%different tocdepth
\usepackage{tocvsec2}

% extra content
\usepackage{titletoc}

% stretch tables
\renewcommand{\arraystretch}{1.5}

% einzug fussnoten
\usepackage[hang]{footmisc}
\setlength{\footnotemargin}{-0.8em}

\usepackage[acronym,toc]{glossaries}
\makeglossaries
%%%%%%%%%%%%%%%%%%%%%%
% Akronyme und Abkürzungen

\newacronym{dtfb}{DTFB}{Deutscher Tischfußball Bund}

%%%%%%%%%%%%%%%%%%%%%
% (Fach-)Begriffe

\newglossaryentry{abroller}{
name={Abroller},
description={Eine Schußtechnik, bei dem die Stange durch Abrollen der Handfläche gedreht wird. Siehe auch Pin.}}
\newglossaryentry{offensive}{
name={Offensive},
description={Spielaktionen mit Ballbesitz: Passen und Schiessen}}
\newglossaryentry{defensive}{
name={Defensive},
description={Spielaktionen und Stellungsspiel ohne Ballbesitz}}
\newglossaryentry{pass}{
name={Pass},
description={Spielaktion, bei der man den Ball von einem Bereich zu einem anderen Bereich spielt, also z.B. vom Torwartbereich zum Mittelfeld oder vom Mittelfeld in den Sturm.}}
\newglossaryentry{abspielen}{
name={Abspielen},
description={Spielaktion, bei der man den Ball von einer Figur zu einer anderen Figur der gleichen Stange (oder im Torwartbereich) spielt, also z.B. Dribbling im Torwartbereich oder Tic-Tac im Mittelfeld.}}
\newglossaryentry{abwehr}{
name={Abwehr},
description={Spielbereich, der die Torwart- und 2er-Stange umfasst.}}
\newglossaryentry{sturm}{
name={Sturm},
description={Spielbereich, der die 3er-Stange umfasst.}}
\newglossaryentry{mittelfeld}{
name={Mittelfeld},
description={Spielbereich, der die 5er-Stange umfasst.}}


\usepackage{nameref}

% abstand von item 
%\newlength{\wideitemsep}
%\setlength{\wideitemsep}{0.5\itemsep}
%\addtolength{\wideitemsep}{-3pt}
%\let\olditem\item
%\renewcommand{\item}{\setlength{\itemsep}{\wideitemsep}\olditem}

%\setlength{\itemsep}{0pt}

\setitemize{itemsep=-2pt,topsep=2pt}
\setenumerate{itemsep=-2pt,topsep=2pt}

\setlength{\bibsep}{5pt}
% further option
\setlength{\parindent}{15pt} % stell den absatz einzg ein z.b. auch null
%\setlength{\parskip}{4pt} % legt den abstand zwischen paragraphen fest, aber auch aufzählungen oder so...

% globale trennung
%\hyphenation{}

% no vertical stretch
\raggedbottom %instead of default flushbottom


%%%%%%%%%%%%%%%%%%%%%%%%%%%%%%%%%%%%%%%%%%%%%%%%%%%%%%%%%%%%%%%%%%%%%%
%%%%%%%%%%%%%%%%%%%%%%%%%%%%%%%%%%%%%%%%%%%%%%%%%%%%%%%%%%%%%%%%%%%%%%
%%%%%%%%%%%%%%%%%%%%%%%%%%%%%%%%%%%%%%%%%%%%%%%%%%%%%%%%%%%%%%%%%%%%%%
\begin{document}


%%%%%%%%%%%%%%%%%%%%%%%%%%%%%%%%%%%%%%%%%%%%%%%%%%%%%%%%%%%%%%%%%%%%%%
% title page
\title{Tischfußball-Grundlagen}
\author{DTFJ Team}
\newcommand*{\titlebskfactor}{0.68}

\makeatletter
\begin{titlepage}
  \rule{\textwidth}{0pt}
  \vfill
  \begin{center}
    \normalfont \bfseries
    \begin{spacing}{1.2}
      \Huge \@title
    \end{spacing}

    \vspace{1cm}

    {
      \Large Ein Dokument
      \\[\titlebskfactor\baselineskip]
      zum Spielen und Erlernen 
      \\[\titlebskfactor\baselineskip]
      von Tischfußball
      \\[\titlebskfactor\baselineskip]
    }

    \vspace{0.5cm}
    
    \begin{minipage}{0.5\textwidth}
	\centering
	\includegraphics[width=0.8\textwidth]{img/logos/tischfussball.jpg}
    \end{minipage}%
%    \begin{minipage}{0.5\textwidth}
%	\centering
%	\includegraphics[width=0.7\textwidth]{img/logos/DTFJ_Logo.jpg}
%    \end{minipage}

    \vspace{0.5cm}

    {
      \large Zusammengestellt vom 
      \\[\titlebskfactor\baselineskip]
      \@author
      \\[\titlebskfactor\baselineskip]
      \vspace{.5cm}
      November 2015
    }

  \end{center}
  \vfill
  \vfill
\end{titlepage}
\makeatother



%%%%%%%%%%%%%%%%%%%%%%%%%%%%%%%%%%%%%%%%%%%%%%%%%%%%%%%%%%%%%%%%%%%%%%
% Inhalt %%%
\cleardoublepage 
\newpage

% page numbering
\pagenumbering{arabic}
\setcounter{page}{1}

% contents
\setcounter{secnumdepth}{3} % Nummerierungstiefe
\setcounter{tocdepth}{3} % Anzeigetiefe
\renewcommand{\contentsname}{Inhalte}
\tableofcontents


%%%%%%%%%%%%%%%%%%%%%%%%%%%%%%%%%%%%%%%%%%%%%%%%%%%%%%%%%%%%%
% Text 
\cleardoublepage
\thispagestyle{empty}

%%%%%%%%%%%%%%%%%%%%%%%%%%%%%%%%%%%%%%%%%%%%%%%%%%%%%%%%%%%%%
% abstract 
\addcontentsline{toc}{chapter}{Vorwort}
\chapter*{Vorwort}

% Warum?
Jeder, der Tischfußball schon einmal gespielt hat, erinnert sich bestimmt daran, dass auf jeden Fall viel gelacht wurde.
Das Tischfußball-Spiel fasziniert und heutzutage ist es Breiten- und Leistungsport. 

Dieses Dokument soll die Grundlagen des heutigen Tischfußball-Sport und dessen Faszination aufzeigen. Dabei gliedert sich der Inhalt im Wesentlichen in folgende Kategorien:
\begin{itemize}
\item Spielgerät und Zubehör (Kapitel \ref{tisch})
\item Regeln (Kapitel \ref{regeln})
\item Technik (Kapitel \ref{technik})
\item Systemspiel (Taktik) (Kapitel \ref{taktik})
\item Spielpsychologie (Kapitel \ref{tisch})
\item Spielformen und Trainingsspiele (Kapitel \ref{mental})
\item Turniere und Meisterschaften (Kapitel \ref{turniere})
\end{itemize}
Durch eine Einleitung (Kapitel \ref{einleitung}) und Kapitel "\nameref{jugend}" werden die Hauptkapitel ergänzt. 
Die Autoren hoffen, dass diese Grundlagen uns anregen, neue Erkenntnisse und Trainingsmethoden im Tischfußball-Sport zu entdecken.


%%%%%%%%%%%%%%%%%%%%%%%%%%%%%%%%%%%%%%%%%%%%%%%%%%%%%%%%%%%%%
% einleitung 
\chapter{Einleitung}

\section{Spielprinzip}

Das Runde muss ins Eckige! Wie beim Fußball ist beim Tischfußball, also dem Fußball auf dem Tisch, das Ziel eines Teams mit seinen elf Spielern den Ball ins gegnerische Tor zu bringen.  
Im Gegensatz zum Fußball sind es meist aber nur zwei Menschen, die in einem Tischfußball-Team, dem sogenannten Doppel, gegen ein anderes Doppel spielen.
Jede Spielerin und jeder Spieler kann durch Bewegen und Drehen der Stangen mit den daran befestigten Spielfiguren den Ball beeinflussen und Tischfußball spielen.
Dabei gibt es natürlich wie bei jeden anderen Sport bestimmte Spielregeln:
\begin{itemize}
\item Der Tisch und die Bälle unterliegen bestimmten Anforderungen.
\item Es gibt Regeln, die festlegen, was erlaubt beim Spielen ist und was ein Fouls sein kann.
\item Es gibt Aufstellungsregeln bei verschiedenen Spiel- und Turnierformen. 
\end{itemize}


\section{Mehr als nur ein Spiel}

\begin{itemize}
\item Spiel für jederfrau und jedermann, für Jung und Alt. 
\item Soziale Aspekte: Zusammenkommen und Kennenlernen, modernes Vereinsleben.
\item Individuelle Entwicklung: Motorik, Konzentration, Verlieren lernen
\end{itemize}

\section{Tischfußball als Sport}

\begin{itemize}
\item Vereine und Verbände
\item Einzel, Doppel, Team
\item Turniere und Ranglisten
\item Weltmeisterschaften, Deutsche Meisterschaften, Landesmeisterschaften, ...
\end{itemize}



%%%%%%%%%%%%%%%%%%%%%%%%%%%%%%%%%%%%%%%%%%%%%%%%%%%%%%%%%%%%%
% sportgerät und zubehör
\chapter{Spielgerät und Zubehör}
\label{tisch}


%%%%%%%%%%%%%%%%%%%%%%%%%%%%%%%%%%%%%%%%%%%%%%
\section{Der Tisch}
\label{tisch:tisch}

Der Tisch zusammen mit einem Ball ist das Spielgerät beim Tischfußball. 
Das etwa 1,1 m lange und 0,7 m breite Spielfeld ist im Tischkorpus, der auf vier Beinen steht, eingelassen und von Banden umrundet. An den Stirnseiten sind die Tore platziert, die mit einer Torauffangschale oder einem Ballrücklauf ausgestattet sind.  
Die jeweils 11 Spielfiguren sind auf 4 Stangen (für drei Spielbereiche) verteilt:
\begin{itemize}  
\item die Torwartstange, auch Torwart (\gls{abwehr}) 
\item die 2er-Stange, auch die Zwei (\gls{abwehr}) 
\item die 5er-Stange, auch die Fünf (\gls{mittelfeld})
\item die 3er-Stange (\gls{sturm})
\end{itemize}  
Die Stangen können mittels Griffen vor- und zurückgedreht und rein- und rausgeschoben werden.
Wenn man am Tisch steht, ist die Spielrichtung von links nach rechts, also das linke Tor ist das eigene und das rechte Tor das vom Gegner.
Neben diesem Grundaufbau gibt es viele Unterschiede bei den vielen \nameref{tisch:tisch:modelle}n. 

\subsection{Tischmodelle}
\label{tisch:tisch:modelle}

Inzwischen gibt es Tischmodelle vieler Hersteller in allen Qualitäts- und Preis-Kategorien:
\begin{itemize}
\item günstige, aber auch billige Tische gibt es schon ab 100 Euro 
\item qualitiative Tische für Hobbyspieler gibt es ab 300-600 Euro
\item Trainingstische für Turnierspieler gibt es ab 600-1.200 Euro
\item offizieller Turniertische gibt es ab 1.200 Euro 
\end{itemize}

Letztendlich sollte der Tisch dem Spielniveau angemessen sein, damit der Spielspaß hochgehalten wird. Dennoch gibt es ein paar Grundsätze, die man beispielsweise bei einem Tischkauf beachten sollte:
\begin{itemize}
\item Der Korpus sollte ein gewisses Gewicht haben, damit er nicht leicht verrutscht. Turniertische beispielsweise wiegen über 100 kg.
\item Die Beine sollten nicht wackeln und höhenverstellbar sein.
\item Die Stangen sollten sich nicht leicht verbiegen lassen.
\item {\bf Empfehlung:} Bei einer Anschaffung eines Tisches und von Bällen sollte man darauf achten, dass ein griffiges Ball-Handling von Vorteil für das Erlernen von einem kontrollierten Spiel ist.
\end{itemize}

In Tabelle \ref{tab:tische} werden die Merkmale und Unterschiede der 5 offiziellen Tischmodelle des \gls{itsf} und der zwei weiteren Partnertische des \gls{dtfb} verglichen. 
Die Angaben in der Tabelle beziehen sich auf das jeweilige offizielle Turniermodell, jedoch hat jeder dieser Tischhersteller vergleichbare Modelle für den Anfänger, Jugend- oder Hobbyspieler.    

% Besonderheiten:
% verkürztes Spielfeld, Torwartbeweglichkeit
% Tornado: Torwartstange mit drei Figuren

{\small
\begin{center} 
\begin{table} 
\begin{tabular}{ p{1.5cm}||p{2cm}|p{2cm}|p{2cm}|p{2cm}|p{2cm}} 
 	& Figuren & Spielfläche & Torbreite & Griffe & Region \\ 
\hline 
\hline 
Leonhart (DTFB, ITSF) & Soccer (Plastik) & hart (Plastik) und normalen Banden & 20,5 cm & rund (Gummi) & Deutschland und Nachbarländer \\ 
\hline 
Ullrich  (DTFB) &  Soccer (Plastik) &  hart (Plastik) und normalen Banden & 20,5 cm & 10-kantig (Gummi) & Deutschland \\ 
\hline 
Lettner (DTFB)  & Soccer (Plastik)  &  hart (Plastik) und normalen Banden & 20,5 cm & rund (Gummi) & Deutschland \\ 
\hline 
Bonzini (DTFB, ITSF)  & schwerer Fuss (Metall) & weich (Linoleum) und normalen Banden & ??? & keilförmig (Plastik), wechselbar & Frankreich und Nord-Europa \\ 
\hline 
Garlando (ITSF)  & schmal, Soccer-ähnlich (Plastik) &  hart (Glass) und schrägen Banden & ??? & rund (Plastik und Holz) & Österreich und Südost-Europa \\ 
% Spielfeld 120x70,5 cm 
\hline 
Roberto (ITSF) & quaderförmig (Plastik) &  hart (Plastik) und normalen Banden & ??? & rund (Gummi) & Italien und Südost-Europa \\ 
% 111 x 70
\hline 
Tornado (ITSF)  & keilförmig (Plastik) &  hart (Plastik) und normalen Banden & 20 cm breit & 6-kantig (Holz oder Gummi) & Nordamerika und englischsprachige Länder \\ 
\end{tabular} 
%\caption{Offizielle \gls{dtfb} und \gls{itsf} Tische mit ihren Eigenheiten. [\cite{www:kickerbau}, \cite{www:tischfussball-online}]}
\label{tab:tische}
\end{table} 
\end{center}
}



%%%%%%%%%%%%%%%%%%%%%%%%%%%%%%%%%%%%%%%%%%%%%%
\subsection{Standort}
\label{tisch:tisch:standort}




Ein Tisch sollte genügend Platz für den Tisch selbst und die Spieler bei voll ausgezogene Stangen haben. Ein Tisch ist  0,75 m x 1,5 m groß und braucht daher eine etwa 2 m x 2,5 m große Fläche (Abb. \ref{fig:tisch:platzbedarf}).

Zudem sollte der Tisch auf einem stabilen und relativ geraden Boden stehen. Bekommt ein Tisch einen neuen Standort, sollte er ausgerichtet werden. Mit Hilfe einer Wasserwaage oder an Hand des Rollen des Balls kann man die Spielfläche durch Höhenrvrstellen der Beine gerade ausrichten (Abb. \ref{fig:tisch:ausrichten}). Dann sollte der Ball ruhig liegenbleiben, wenn man diesen irgendwo auf das Spielfeld legt.

\begin{figure}
%\begin{wrapfigure}{r}{0.6\textwidth} 
\centering 
\begin{subfigure}[b]{0.7\textwidth} 
\includegraphics[width=\textwidth]{img/tisch_platzbedarf.png} 
\caption{Platzbedarf} 
\label{fig:tisch:platzbedarf} 
\vspace{0.5cm}
\end{subfigure} 
\begin{subfigure}[b]{0.7\textwidth} 
\includegraphics[width=\textwidth]{img/tisch_ausrichten.png} 
\caption{Ausrichten} 
\label{fig:tisch:ausrichten} 
\end{subfigure} 
\label{fig:tisch} 
\caption{Aufstellen eines Tischs [\cite{itsf_basics}]} 
\end{figure}
%\end{wrapfigure}

%%%%%%%%%%%%%%%%%%%%%%%%%%%%%%%%%%%%%%%%%%%%%%
\subsection{Pflege und Wartung}
\label{tisch:tisch:wartung}

Um einen gleichbleibenden Spielspaß zu haben, wird eine regelmäßige Tischpflege und Wartung empfohlen:
\begin{itemize}  
\item {\bf Stangen:} Damit die Stangen leicht laufen, schmieren viele Spieler die Stangen mit Pronto-Spray (Möbelpolitur) oder Silikonöl, bevor Sie Tischfußball spielen. Dabei ist in jedem Fall darauf zu achten, die ganz herausgezogene oder ganz reingeschobenene Stange außerhalb des Tisches und nie über der Spielfläche zu beschmieren. Danach sollten die Stangen hin- und hergedreht werden, während man die Stange rein- und rauszieht, damit sich das Schmiermittel gut über die Stange verteilt  und die Stangen gut in den Lagern laufen.
\item {\bf Spielfeld:} Auf dem Spielfeld sammeln sich über die Zeit Staub, auflösende Gummistückechen vom Puffer oder Ballspuren. Diese lassen sich am einfachsten und schonend mit etwas Glasreiniger und Haushaltspapier entfernen -- zumindest bei Soccer-Spielflächen wie bei Leonhart- oder Ullrich-Tischen. In jedem Fall sollte man auf Spülmittel und Scheuerschwämme verzichten. 
\item {\bf Lager und Puffer:} Das Schmiermittel für die Stangen kann leider auch leicht Staub binden. Diese Masse setzt sich gerne in den Lagern über die Zeit ab. Mit einem zurechtgerollten normalen Schreib-Papier kann man die Ablagerungen durch Durchschieben der Rolle entfernen. Manchmal lohnt sich aber auch eine Grundreinigung und man baut die Figuren, Stangen und Lager aus, um die Lager ohne Stange gründlich zu reinigen. Dabei sollte man sich überlegen, ob man die Puffer ersetzt. Puffer lösen sich durch die Schmiermittel und die Belastung nach einiger Zeit auf.    
\item {\bf Allgemeines:} Natürlich sollten alle Schrauben am Tisch fest sitzen. Es lohnt sich ab und zu zu überprüfen, ob alle Schrauben noch festsitzen.  
\end{itemize}  


%%%%%%%%%%%%%%%%%%%%%%%%%%%%%%%%%%%%%%%%%%%%%%
%%%%%%%%%%%%%%%%%%%%%%%%%%%%%%%%%%%%%%%%%%%%%%
\section{Bälle}
\label{tisch:baelle}

Es gibt verschiedene Typen von Bällen, obwohl der Durchmesser typischerweise 35 mm beträgt -- manchmal auch 34 mm \citep{www:kickerbau:baelle}.
Tischfußball-Bälle unterscheiden sich vor allem in ihrer Griffigkeit und ihrem Gewicht, zudem in ihrer Farbe und auch beim Material.
Insbesondere die Griffigkeit, und damit die Oberflächenbeschaffenheit des Balls, und das Gewicht sind entscheidend für die Spieleigenschaften.  

Vorgestellt werden hier drei Bälle, die auf den Soccertischen (Leonhart, Ullrich, Lettner) üblicherweise gespielt werden und alle aus weißem Plastik sind \citep{www:tfc-reutlingen}:
\begin{itemize}
\item Der Ullrich Ball:
Der Ullrich Ball ist durch seine eher weiche Oberfläche der griffigste Ball unter den Soccer-Bällen. 
Das ermöglicht eine Ballführung mit viel Kontrolle. Insbesondere beim Klemmen kann man damit den Ball an verschiedenen Punkten unter der Stange führen und spielen \ref{klemmen}.
% Grafik ? Foto 
Mit einem üblichen 35 mm Durchmesser wiegt der Ball 24 g.
Dadurch das der Ball relativ weich ist, bekommt er mit der Zeit Macken und Dellen, so dass ein langsamer Ball zum Teil nicht mehr ganz gerade rollt.
Er ist offizieller Ball in Deutschland und mit dem DTFB- und dem P4P-Schriftzug bedruckt.
\\
Preis: 2,00 Euro. 
\\
Spieler: Ab Anfänger bis Profi
\item Der Leonhart Ball: 
Mit einem Durchmesser von 35 mm und einem Gewicht von etwa 27 g ist im Vergleich zu vielen Bällen relativ schwer. 
Zusammen mit seiner harten und doch leicht griffigen Oberfläche hat er damit ruhiges und genaues Rollverhalten.
Er ist mit dem ITSF Schriftzug bedruckt, da er offizeller ITSF Ball für den Leonhart-Tisch ist.
\\
Preis: 2,50 Euro. 
\\
Spieler: Ab Fortgeschrittene bis Profi
\item Der Lettner Ball (Contus Avant):
Der dritte DTFB zertifizierte Ball ist für den Lettner Tisch entwickelt.
Er hat ein Gewicht von 27 g und einem Durchmesser von 34,9 mm.
Der Ball hat eine harte Oberfläche, ist jedoch leicht griffig. 
Er ist nicht so verbreitet wie der Leonhart- oder Ullrich-Ball. 
\\
Preis: 2,75 Euro.
\\
Spieler: Ab Amatuer bis Profi
\end{itemize}

Die Griffigkeit hängt nicht nur von der Art des Balls ab, sondern ist ein Zusammenspiel zwischen Figur, Ball und Spieloberfläche. 
Zum Beispiel spielt sich ein Leonhart-Ball auf einem Ullrich-Tisch eher weniger griffig.  
Und auch die offiziellen Tische -- die 5 ITSF- und 3 DTFB-Tische -- haben jeweils ihren eigenen zertifizierten Ball und dadurch unterschiedlichste Spieleigenschaften:
Der Bonzini-Ball ist orange und hart, aber auf dem Bonzini-Tisch sehr griffig. 
Im Gegensatz zum Tornado Ball, der aus roten Urethan ist, was ihn sehr hart und schwer macht -- und sehr ungriffig bzw. rutschig. 



%%%%%%%%%%%%%%%%%%%%%%%%%%%%%%%%%%%%%%%%%%%%%%
\section{Zubehör}
\label{tisch:zubehoer}

Das wichtigste Spielzubehör ist für eine gute Griffigkeit verantwortlich: \nameref{tisch:zubehoer:griffe}.
Auch auf angemessene \nameref{tisch:zubehoer:kleidung} legen Amateuer- und Profi-Spieler wert.
Zudem gibt es einige \nameref{tisch:zubehoer:training}, die für ein abwechslungsreiches und fokussiertes Trainieren sorgen können.

%%%%%%%%%%%%%%%%%%%%%%%%%%%%%%%%%%%%%%%%%%%%%%
\subsection{Griffbänder und Handschuhe}
\label{tisch:zubehoer:griffe}

Alle Spielaktionen beim Tischfußball geschehen durch die Hände der Spieler, die die Stangen über die Griffe bewegen.
Egal bei welcher \nameref{griffhaltung} und egal für welche Techniken ist eine gewisse Griffigkeit notwendig. 

Obwohl die festinstalliersierten Griffe bei den DTFB-Tischen (Leonhart, Ullrich, Lettner) aus schwarzem Gummi gurndsätzlich eine gute Griffigkeit bieten, kann sich die Griffigkeit bei längerem Spiel wegen Schwitzens verschlechtern.
Daher verwenden viele Spieler (Tennis-)Griffbänder, die sie vor dem Spielen um die Griffe wickeln. 
Griffbänder bieten zwei entscheidende Vorteile:
\begin{itemize}
\item Schweißabsorbtion und dadurch gleichbleibende Griffigkeit
\item besserer Griff beziehungsweise bessere Stangenkontrolle
\end{itemize}
Nach dem Spielen sollte man die Bänder wieder aufwickeln, damit die Oberfläche nicht einstaubt, und die Griffigkeit erhalten bleibt.
Manche Griffbänder kann man sogar mit in der Waschmaschine waschen, so dass diese nach längerer Benutzung wieder fast  wie neu sind. 
Griffbänder gibt es von vielen verschiedenen Herstellern und kosten etwa zwischen 0,50 und 2,50 Euro. 

\paragraph{Wie wickelt man ein Griffband auf?} Die Standardwicklung: 
\begin{itemize}
\item[a)] Man beginnt bei der tischnäheren Seite des Griffs und wickelt nach außen.
Man sollte darauf achten, dass der Bandanfang durch die ersten Wicklungen fixiert wird. 
\item[b)] Durch Drehen der Stange wickelt man das Band gleichmässig auf den Griff, wobei das Band stets etwa zur Hälfte die darunter leigende Schicht überdecken sollte.
\item[c)] Ist das Band ganz aufgewickelt, benutzt man ein Abschlussgummi, um die letzte Wicklung zu fixieren.
\end{itemize}
Video-Link: \url{https://www.youtube.com/watch?v=KgOqDdcp4n4}


Ein ebenfalls beliebtes Zubehör sind (Golf-)Handschuhe. 
Damit kann man insbesondere in Kombination mit Griffschläuchen oder -gummis, die man über den Griff zieht, eine erhöhte Griffigkeit erlangen.
Dieses Material ist bei Spielern, die mit der Abroller-Technik schießen, besonders beliebt, da man mit wenig Druck auf die Stange, dennoch eine hohe Seitwärts-Bewegung erreichen kann (\ref{abroller}).  

Spieler, die Jet schießen, benutzen Material um ihre Handgelenke bzw. den Unterarm. 
Manche benutzen dafür ein verkürztes Griffband bzw. eigens dafür hergestellte Jet-Armbänder.
Dadurch soll ebenfalls eine schnelle Seitwärts-Bewegung erreicht werden, bei wenig Kraftaufwand (\ref{jet}).



\paragraph{Hintergrund:} Da sich die Griffe der 5 offiziellen ITSF-Tische in Dicke und Form, Material und Griffigkeit unterscheiden, wurde vor einigen Jahren ein Griffwechselsystem bei internationalen Turnieren eingeführt.
Das ermöglicht den Spielern ihren Lieblingsgriff an allen Tischen zu spielen. 

%%%%%%%%%%%%%%%%%%%%%%%%%%%%%%%%%%%%%%%%%%%%%%
\subsection{Kleidung}
\label{tisch:zubehoer:kleidung}

Nach dem ITSF-Regelwerk gibt es beim Tischfußball einen Dresscode, der Sportkleidung vorschreibt. 
Bei internationalen und nationalen Wettkämpfen tragen die Spieler Trikots mit ihrem Vereinslogo und ihrem Namen darauf. 
Neben der Teamzugehörigkeit bieten Sporttrikots eine gewisse Atmungsaktivität, die bei einem schweisstreibendem Spiel von Vorteil sind. 
Manche Spieler haben sogar ein Handtuch in Tischnähe, um Ihren Schweiss in Auszeiten abzuwischen.

Während eines Turniertages ziehen sich viele Spieler, zwischen den Spielen eine Trainingsjacke an, um nicht auszukühölen.
Bei der Hosenwahl -- ob kurz oder lang -- ist es den Spielern überlasssen, wie man sich am wohlsten fühlt. 
Bei der Schuhwahl sollte man darauf achten, dass man beim Tischfußball überwiegends steht. 
Daher wird ein wohlfühlendes Schuhbett empfohlen, wie etwa ein gut gedämpfter Laufschuh.  

%%%%%%%%%%%%%%%%%%%%%%%%%%%%%%%%%%%%%%%%%%%%%%
\subsection{Trainingsmaterialien}
\label{tisch:zubehoer:training}

%%%%%%%%%%%%%%%%%%%%%%%%%%%%%%%%%%%%%%%%%%%%%%
\subsubsection{Stangenklemmen oder Rod-Locks}
\label{tisch:zubehoer:training:rodlock}

\begin{figure}
%\begin{wrapfigure}{r}{0.4\textwidth} 
\centering 
\includegraphics[width=0.38\textwidth]{img/rodlock_goalie.jpg} 
\caption{Um ein Goalie-Einzel zu spielen, verbindet hier eine rote Stangenklemme die gelbe 3er- mit der schwarzen 5er-Reihe und eine zweite Klemme die gelbe 5er- mit der schwarzen 3er-Reihe. [\cite{www:rod-lock}]} 
\label{fig:rod-lock} 
\end{figure}
%\end{wrapfigure}

Stangenklemmen, oder im Englischen Rod-Locks, sind wohl das gängigsten Trainingszubehör. 
Eine Klemme kann an zwei Stellen an eine Stange geklemmt werden, so dass sich zwei Stangen miteinander fixiert werden können. 
Damit ergeben sich folgende Trainingsvarianten \citep{www:rod-lock}:
\begin{itemize}
\item \nameref{spielformen:sonderregeln:goalie}: Mit zwei Stangenklemmen kann man die Mittelfeld- und Sturmreihen hochklappen, so dass die Spielfiguren waagrecht bleiben (siehe Abb.~\ref{fig:rod-lock}). 
Somit kann man Torwart gegen Torwart spielen. 
\\
Zielgruppe: ab Anfänger
\\
Trainingseffekt: Torwartdasein
\item Schusstraining: Mit einer Klemme kann man die Torwart- und 2er-Stange festklemmen. 
Durch Variation der Distanz zwischen Torwart- und Abwehrfigur sowie deren Stellungswinkel kann man verschiedene statische Deckungen stellen.
Dadurch kann trainieren auf verschiedene Lücken Tore zu schießen.  
\\
Zielgruppe: ab Fortgeschrittene
\\
Trainingseffekt: Hand-Auge-Koordination, Präzession beim Schiessen
\item Mittelfeldtraining: Eine Stangenklemme and die gegnerische Mittelreihe einseitig geklemmt, so dass die Figuren nach vorne gestellt sind, wenn die Klemme nach unten hängt, ist für ein Mittelfeldtraining effektiv:
Beim Passtraining von der 5er- auf die 3er-Stange versucht man durch die gegnerische 5er-Reihe zu spielen. 
Trifft man die gegnerische Figur, wird der Ball durch das Auslenken der Stangenklemme automatisch zurückgespielt, so dass man zum Einen lernt die Abpraller zu kontrollieren und zum Anderen gleich den nächsten Passversuch spielen kann, ohne den Ball mit der Hand zurechtzulegen.
\\
Zielgruppe: ab Amateurspieler
\\
Trainingseffekt: Mittefeldspiel, Präzession beim Passen
\end{itemize}
Eine Stangenklemme kostet 10 bis 20 Euro.

Für das \nameref{spielformen:sonderregeln:goalie} kann man zwei Stangen auch mit einem einfachen Haushaltsgummi hochstellen.
Dafür legt man den Gummi um den Kopf einer Figur (z.B. die mittlere Figur der 3er-Reihe), dreht dann das andere Ende des Gummis um eine halbe Umdrehung und legt dieses um des Kopf der Figur an der zweiten Stange (z.B. die mittlere Figur der 5er-Reihe). 

%%%%%%%%%%%%%%%%%%%%%%%%%%%%%%%%%%%%%%%%%%%%%%
\subsubsection{Weiteres Zubehör}
\label{tisch:zubehoer:training:weiteres}

\begin{itemize}
%%%%%%%%%%%
\item {\bf Backbouncer:}
Der Backbouncer ist aus Holz gebaute U-Form, die die Länge einer Tischbreite hat \citep{www:kickertrainer}.
An einer Seite ist ein elastischer Zellkautschuk angebracht.
Die U-Form lässt sich um jede Stange und deren Figuren stellen. 
Damit prallt der Ball zurück, wenn man diesen gegen den Kautschuk schiesst: 
Man kann alleine üben, abgeprallte Bälle zu stoppen und zu kontrollieren, in dem man verschieden stark und mit unterschiedlichen Winkeln gegen den Kautschuk spiel. 
Den Backbouncer gibt es auch mit einem zackig geschnittenen Kautschuk, so dass der Ball unvorhergesehen abprallt.  
\\
Zielgruppe: ab Anfänger 
\\
Trainingseffekt: Hand-Auge-Koordination, Reaktion, Ballgefühl
%%%%%%%%%%%
\item {\bf FoosTrain:}
Mit Federn verbindet man die 2er- und die 3er-Stange \citep{www:foostrain}.
Sobald man eine Seitwärts-Bewegung macht, folgt die Abwehrstange der Bewegung, so dass man schnell genug zum Schuss kommen muss, um nicht von der heranfahrenden Abwehrfigur gehalten zu werden.
Durch größere Federstärken erhöht man den Schwierigkeitsgrad, da die Abwehrstange schneller der Bewegung folgt.
\\
Zielgruppe: ab Fortgeschrittene
\\
Trainingseffekt: Schusstechnik und Bewegungsablauf 
%%%%%%%%%%%
\item {\bf Table Soccer Coach oder Visual-Kickertrainer:}
Der Table Soccer Coach ist eine Smart-Phone App \citep{www:tablesoccercoach}:
Man legt das Gerät über das Tor und startet das Programm. 
Nach einer zufälligen Zeit erscheint dann ein Signal und man muss schießen.
Neben dem akkustischen Signal gibt es noch ein visuelles Signal auf dem Display, das anzeigt, welche Lücke man schießen soll.
Ähnlich funktioniert der Visual-Kickertrainer, der durch eine LED-Leiste realisiert ist \citep{www:visualkickertrainer}.
\\
Zielgruppe: ab Amateurspieler 
\\
Trainingseffekt: Abrufbarkeit der Schüsse, Entscheidungsgeschwindigkeit
%%%%%%%%%%%
\item {\bf Kickertrainer:}
Der Kickertrainer ist im Prinzip ein verkürzter Tisch: Das Spielfeld ist 3 Stangen lang \citep{www:kickertrainer}. 
Damit kann man ihn schnell auf- und abbauen und zu Hause unterbringen, falls man nicht den Platz für einen Kickertisch hat.
Zudem kann man sich mittels Holzplättchen verschiedene Lücken stellen, durch die man Passen oder Schiessen muss. 
\\
Zielgruppe: ab Anfänger 
\\
Trainingseffekt: Präzession beim Passen und Schiessen
%%%%%%%%%%%
\item {\bf Kicker Maschine:} 
Die Kicker Maschine ist eine junge Entwicklung, die es noch nicht zu kaufen gibt, aber eine interessante Trainingsmöglichkeit bietet \citep{www:kickermaschine}. 
Durch ein Motor-betriebenes Hebelsystem werden die gegnerischen Stangen automatisch hin- und herbewegt. 
Dadurch kann man das Passen und Schiessen üben, während sich die gegnerischen Stangen rythmisch bewegen.
\\
Zielgruppe: ab Amateuerspieler 
\\
Trainingseffekt: Deckungsanalyse und Timing

\end{itemize}


%%%%%%%%%%%%%%%%%%%%%%%%%%%%%%%%%%%%%%%%%%%%%%%%%%%%%%%%%%%%%
% regeln
\chapter{Regeln}
\label{regeln}


\section{Grundlagen}

Anlehnung an ITSF Basisregeln / P4P Rookie Regeln / DTFB Grundregeln

\section{Regelfahrplan}

Schrittweise Hinführung zum ITSF Regelwerk (Einmannpass, Pass mit laufendem Ball, ...)

\section{ITSF Regelwerk}

Verweis 



%%%%%%%%%%%%%%%%%%%%%%%%%%%%%%%%%%%%%%%%%%%%%%%%%%%%%%%%%%%%%
% technik und taktik 
\chapter{Technik und Taktik}
\label{technik}

\section{Motivation: Mit Tricks zum Profi}
\label{technik:motivation}

\begin{itemize}
\item Tricks sind kontrolliertes Spiel
\item Ballkontrolle, Passen
\item Schießen, Systeme
\item Verteidigen
\end{itemize}

\section{Körper- und Griffhaltung}
\label{technik:haltung}

\begin{itemize}
\item Körperhaltung (Schulterbreiter Stand, nicht auf den Griffen abstützen, sich nicht im weg stehen, vor den Stangen)
\item Hand/Griffhaltung
\end{itemize}

\section{Offensive -- mit Ball}
\label{technik:offensive}


Grundsätzliches zum Erlernen der \gls{offensive}: ,,Aktion'', altersgerechtes Training,  für Kinder und Jugendliche: weniger statisches Techniktraining, mehr spielerisches Erlernen, siehe Spielformen,

GRAFIK: mit Bereichen der Spielfiguren.

Verweise: Ungeblogt / Youtube / Sammelwebseite (DTFJ)

\begin{itemize}
\item Ballführung (auf einer Stange): 
\begin{itemize}
\item Ruhender Ball und Puppenwechsel.
\item Tictac oder Vorne-hinten Klemmen.
\item Auf der 2er-Stange zwischen der 1. und der 2. Puppe.
\item Auf der 3er-Stange zwischen der 1. und der 2. Puppe oder der 2. und der 3.Puppe oder mit der 1. und 3. Puppe.
\item Auf der 5er-Stange zwischen der zwei benachbarten Puppen oder mit zwei Puppen und dabei eine Puppe auslassen.
\end{itemize}
\item Passen (zwischen zwei Stangen):
\begin{itemize}
\item Ins Feld, gerade oder an die Bande.
\item Kanten- oder Brushpass.
\item Ballanahme.
\item Von der 5er- auf die 3er-Stange.
\item Von der 2er-Stange auf die 3er-Stange.
\item Von der 2er-Stange auf die 5er-Stange.
\end{itemize}
\item Torschüsse: 
\begin{itemize}
\item Schieber, Zieher, Abroller oder Jet.
\item Von der 3er-Stange oder der 2er-Reihe.
\item Rechts- oder Linksschuss.
\item Kurze oder lange Schüsse.
\item Trickschüsse
\end{itemize}
\end{itemize}

\section{Defensive -- ohne Ball}
\label{technik:defensive}

Grundsätzliches zur \gls{defensive}: ,,Reaktion'', Stellungsspiel, Bälle blocken und fangen, Lieblingsschuss des Gegners herausfinden, Antizipieren: Was kann passieren?
\begin{itemize}
\item Ballbesitz des Gegners im Abwehrbereich 
\begin{itemize}
\item Deckung als Stürmer
\item Deckung als Torwart: (statisch) im kurzen/langen Eck
\item Deckung im Doppel
\item Deckung im Einzel
\end{itemize}
\item Ballbesitz des Gegners im Mittelfeld (5er-Reihe)
\begin{itemize}
\item Deckung als Stürmer (5er-Reihe): Pass verhindern, kleinster Fahrbereich, an die Bande fahren
\item Deckung als Torwart: Torschüße, direkter Ballweg
\end{itemize}
\item Ballbesitz des Gegners im Sturm (3er-Reihe)
\begin{itemize}
\item Deckung als Torwart: Reaktion, fahren, shaken, Wechseln
\end{itemize}
\end{itemize}





%%%%%%%%%%%%%%%%%%%%%%%%%%%%%%%%%%%%%%%%%%%%%%%%%%%%%%%%%%%%%
% systemspiel 
\chapter{Systemspiel (Taktik)}
\label{taktik}

\section{Motivation: Strategien zu Gewinnen.}


\section{Offensive -- mit Ball}

\subsection{Abwehr (Torwart und 2er Reihe)}
\label{taktik:offensive:abwehr}

\begin{itemize}
\item Torschüsse und Pässe auf den Sturm 
\item mehrere OPtionen von einem Punkt aus
\end{itemize}

\subsubsection{Zieher}
Schräges System

\subsubsection{Pin}
\href{http://ungeblogtkickern.blogspot.de/2014/12/schusssystem-linkslang-pin-aus-der.html}{Gerades System}
(Besser wäre Rechtslang Pin)

\subsubsection{Banden}
Von Mittelpunkt



\subsection{Sturm (3er Reihe)}
\label{taktik:offensive:sturm}

Torschüsse, Duell gegen den Torwart.

\begin{itemize}
\item \href{http://ungeblogtkickern.blogspot.de/2015/11/3-punkte-theorie-auf-der-3er-reihe.html}{3 Punkte Theorie (Lücken)}
\item Mittel-Systeme
\item Lang-Systeme
\item (Drei)Viertel-Systeme
\item Systemunterschiede (Welches System das richtige für mich?)
\end{itemize}

\subsubsection{Jet Mitte}
\href{http://ungeblogtkickern.blogspot.de/2015/06/system-jet-mitte.htmli}{
Statisches Mittesystem}
\subsubsection{Pin Mitte}
\href{http://ungeblogtkickern.blogspot.de/2015/08/system-pin-mitte.html}{
Dynamisches Mittesystem}
\subsubsection{Pin Rechtslang}
\href{http://ungeblogtkickern.blogspot.de/2015/07/system-pin-rechtslang.html}{
Langes Langsystem}

\begin{figure}
%\begin{wrapfigure}{r}{0.4\textwidth} 
\centering 
\includegraphics[width=0.25\textwidth]{img/schuss3er_lang.png} 
\caption{Rechtslang} 
\label{fig:rechtslang} 
\end{figure}
%\end{wrapfigure}

\subsubsection{Jet Linkslang}
\href{http://ungeblogtkickern.blogspot.de/2015/07/system-jet-linkslang.html}{
Kurzes Langsystem}
\subsubsection{Zieher}
\href{http://ungeblogtkickern.blogspot.de/2015/09/system-zieher.html}{
Statisches, schräges Langsystem}

 
\subsection{Passen (5er Reihe)}
\label{taktik:offensive:passen}

\begin{itemize}
\item 2 Punkte Theorie
\item Passeigenschaften
\item http://ungeblogtkickern.blogspot.de/2015/11/passspiel-1-passeigenschaften.html
\item http://ungeblogtkickern.blogspot.de/2015/11/passspiel-2-setup.html
\end{itemize}


\section{Defensive -- ohne Ball}
\label{taktik:defensive}

Grundsätzliches
\begin{itemize}
\item Den Ball erobern für Ballbesitz und eigene Offensivaktionen. Mit den Figuren die direkten Ballwege zum Tor blockieren und den Ball annehmen/fangen. 
\item Antizipieren: Was kann passieren? Gegner lesen: Was hat der Gegner vor? (Spielart, Spielniveau, Lieblingsschuss, Entscheidung, ...)
\end{itemize}

Strategien:
,,Re-aktion'': Auf den Ball reagieren, also die Laufbahn des Balls mitverfolgen oder ,,Aktion'':
\begin{itemize}
\item Locken (Lücke)
\item Erschweren (Shuffle)
\item Kreuzen (Wegziehen)
\item Außenmann 
\item \href{http://ungeblogtkickern.blogspot.de/2015/06/defensivbewegungen-im-verteidigerbereich.html}{Artikel auf Ungeblogt}
\end{itemize}


\subsection{Abwehr: Torwart- und 2er-Reihe}
\label{taktik:defensive:halten}

3 Punkte Deckung


\subsection{Mittelfeld (5er-Block)}
\label{taktik:defensive:5erblock}

\begin{itemize}
\item Locken (Lücke)
\item Erschweren (Shuffle)
\item Zurückfahren (Wegziehen)
\item http://ungeblogtkickern.blogspot.de/2015/05/5er-defensive.html
\item http://ungeblogtkickern.blogspot.de/2015/06/defensivbewegungen-auf-der-5er-reihe.html
\item Passeigenschaften ausnutzen, Rebound
\item http://ungeblogtkickern.blogspot.de/2015/11/passspiel-1-passeigenschaften.html
\item http://ungeblogtkickern.blogspot.de/2015/11/passspiel-2-setup.html
\end{itemize}  



\subsection{Aktives Stellungsspiel}
\label{taktik:defensive:aktivesstellungsspiel}

\begin{itemize}
\item Aufbau an der 5er Reihe
\item Schräge Systeme
\item Gerade Systeme
\item Wechsel zwischen verschiedenen Systemen
\item Runterklappen Sturm und Mittelfeld
\end{itemize}


%%%%%%%%%%%%%%%%%%%%%%%%%%%%%%%%%%%%%%%%%%%%%%%%%%%%%%%%%%%%%
% mentales spiel und spielpsychologie 
\chapter{Spielpsychologie}
\label{regeln}

\section{Grundlagen}
Spaß haben, Tricks zeigen, ausprobieren und natürlich Tore schießen 

Gewinnen, Verlieren und auf das nächste Mal freuen 



\section{Fortgeschrittene}
nicht für die Grundlagen

\subsection{Kontrolliert}
Zeitlassen bei Aktionen und Timeouts, Quote, Überlegenheit, nicht beinflussen
\subsection{Arbeiten}
Aktive Stangenbewegung, mehr Risiko, schnelle Überraschungsaktionen, den Gegner unter Druck setzen

\section{Profi}
nicht für die Grundlagen


%%%%%%%%%%%%%%%%%%%%%%%%%%%%%%%%%%%%%%%%%%%%%%%%%%%%%%%%%%%%%
% spielformen
%%%%%%%%%%%%%%%%%%%%%%%%%%%%%%%%%%%%%%%
\chapter{Spielformen}
\label{spielformen}

Tabellarische Übersicht


%%%%%%%%%%%%%%%%%%%%%%%%%%%%%%%%%%%%%%%
%%%%%%%%%%%%%%%%%%%%%%%%%%%%%%%%%%%%%%%
\section{Zählweisen}
\label{spielformen:zaehlweisen}

Beim Tischfußball muss man Tore schießen und am besten mehr als sein Gegenspieler, um am Ende zu gewinnen.
Es gibt jedoch verschiedene Zählweisen, auf die man sich vor Beginn des Spiels einigt oder die bei einem Turnier gespielt werden. 
%%%%%%%%%%%%%%%%%%%%%%%%%%%%%%%%%%%%%%%
\subsection{Ein Satz}
\label{spielformen:zaehlweisen:einsatz}

Wenn man einfach so kickert ohne ein Turnier zu spielen, spielt man oft einen Satz bis 6 Tore: Wer zuerst 6 Tore geschossen hat, gewinnt die Partie.
Als wichtigste Variante gilt die Unentschiedenregel: Bei 5 zu 5 Toren geht der Satz unentschieden aus.

Obwohl bei vielen Tischmodellen die Torzähler 10 Tore zählen können, spielen die meisten Spieler bis 6 Tore -- aber auch ein Satz bis 5 oder 7 Toren sind verbreitet.

%%%%%%%%%%%%%%%%%%%%%%%%%%%%%%%%%%%%%%%
\subsection{Mehrere Gewinnsätze}
\label{spielformen:zaehlweisen:gewinnsaetze}

Wie beim Tennis, spielt man bei Tischfußballpartien über mehrere Gewinnsätze -- das heißt, wer zuerst eine festgelegte Anzahl von Sätzen gewinnt, gewinnt die Partie. Am gängigsten sind:
\begin{itemize}
\item Zwei Gewinnsätze (engl. "Best of three"): Derjenige, der zuerst zwei Sätze gewinnt, gewinnt die Partie. Mit 2:0 Sätzen hat man also gewonnen. 
Es können maximal drei Sätze gespielt werden: Falls es nach den ersten beiden Sätzen 1:1 steht, gibt es einen dritten entscheidenden Satz, und die Partie geht 1:2 oder 2:1 aus. 
\item Drei Gewinnsätze (engl. "Best of five"): Derjenige, der zuerst drei Sätze gewinnt, gewinnt die Partie. Steht es 2:2 nach den ersten vier Sätzen, gibt es einen fünften und entscheidenden Satz. 
\end{itemize}
Die Sätze werden hier bis 5 Tore gespielt. Beim Entscheidungsatz, also Satz 3 bei zwei Gewinnsätzen oder Satz 5 bei drei Gewinnsätzen, gibt es eine Verlängerung, wenn 4:4 Tore steht: Jetzt braucht man zwei Tore Vorsprung, um den letzten Satz und damit die Partie zu gewinnen. Bei drei Gewinnsätzen kann einen knappe Partie zum Beispiel 4:5, 5:4, 4:5, 5:4 und 8:6 in Toren und damit 3:2 in Sätzen ausgeht. 

Mehrere Gewinnsätze werden oft bei Turnieren in der KO-Runde gespielt. Hierbei setzt sich am Ende meist das bessere Team durch.




%%%%%%%%%%%%%%%%%%%%%%%%%%%%%%%%%%%%%%%
%%%%%%%%%%%%%%%%%%%%%%%%%%%%%%%%%%%%%%%
\section{Spiele mit unterschiedlich vielen Personen}
\label{spielformen:npersonen}

Neben den zwei Grundkategorien \nameref{spielformen:npersonen:einzel} und \nameref{spielformen:npersonen:doppel} und dem \nameref{spielformen:npersonen:team}, gibt es auch die Möglichkeit mit drei, fünf, sechs, sieben, acht oder sogar mehr Personen an einem Tisch zu kickern.
Damit kann man ein Training oder ein Turnier so gestalten, dass möglichst alle Teilnehmer viel spielen können. 


%%%%%%%%%%%%%%%%%%%%%%%%%%%%%%%%%%%%%%%
\subsection{Einzel}
\label{spielformen:npersonen:einzel}

\begin{itemize}
\item Benötigte Spieler: 2
\item Aufstellung: Eins gegen eins; jeder bedient mit seinen zwei Händen seine vier Stangen; ein Ball.
\item Regeln: Normale Regeln 
\end{itemize}
 
%%%%%%%%%%%%%%%%%%%%%%%%%%%%%%%%%%%%%%%
\subsection{Doppel}
\label{spielformen:npersonen:doppel}

\begin{itemize}
\item Benötigte Spieler: 4
\item Aufstellung: Zwei gegen zwei; jedes Team hat einen Torwart, der die 1er- und 2er-Stange bedient, und einem Stürmer, der die 5er- und 3er-Stange bedient; ein Ball.
\item Regeln: Normale Regeln 
\end{itemize}

%%%%%%%%%%%%%%%%%%%%%%%%%%%%%%%%%%%%%%%
\subsection{Teammodus}
\label{spielformen:npersonen:team}

\begin{itemize}
\item Benötigte Spieler: 2mal mindestens 3 Spieler
\item Spielplan: Es gibt einen Spielplan mit Einzel- und Doppelpartien, die typerweise auf einen Satz mit Unentschieden
\item Regeln: Normale Regeln 
\end{itemize}
 
Team (2mal 3+)

%%%%%%%%%%%%%%%%%%%%%%%%%%%%%%%%%%%%%%%
\subsection{Drei gegen drei (6 Personen)}
\label{spielformen:npersonen:dreigegendrei}
3:3 (6)

%%%%%%%%%%%%%%%%%%%%%%%%%%%%%%%%%%%%%%%
\subsection{Vier gegen vier (8 Personen)}
\label{spielformen:npersonen:viergegenvier}

%%%%%%%%%%%%%%%%%%%%%%%%%%%%%%%%%%%%%%%
\subsection{Zwei gegen einen (3 Personen)}
\label{spielformen:npersonen:zweigegeneinen}

%%%%%%%%%%%%%%%%%%%%%%%%%%%%%%%%%%%%%%
\subsection{Jeder mit und gegen jeden (3 Personen)}
\label{spielformen:npersonen:jedergegenjeden}

%%%%%%%%%%%%%%%%%%%%%%%%%%%%%%%%%%%%%%
\subsection{Runde (4 und mehr Personen)}
\label{spielformen:npersonen:runde}





%%%%%%%%%%%%%%%%%%%%%%%%%%%%%%%%%%%%%%%
\subsection{Alleine am Tisch}
\label{spielformen:npersonen:alleine}

Wenn man mal alleine am Tisch steht, kann man eine ganze Menge ganz in Ruhe üben oder ausprobieren:
\begin{itemize}
\item Ballführung: Man spielt sich den Ball zwischen den Puppen hin und her, entweder durch Tic-Tac oder Klemmen des Balls oder durch Spielen an die Bande 
\item Passen:
\item Auf das Tor schießen: 
\end{itemize}
Diese Übungen kann



%%%%%%%%%%%%%%%%%%%%%%%%%%%%%%%%%%%%%%%%
%%%%%%%%%%%%%%%%%%%%%%%%%%%%%%%%%%%%%%%%
%%%%%%%%%%%%%%%%%%%%%%%%%%%%%%%%%%%%%%%%
\section{Spiele mit Sonderregeln}
\label{spielformen:sonderregeln}

Hier findet ihr eine Sammlung an Spielarten, bei denen die normalen Regeln durch Sonderregeln ergänzt werden. 
Dadurch lernt man mit Spaß und spielerisch bestimmte Trainingsinhalte, da die Sonderregeln einen bestimmte Fokus setzen. 

%%%%%%%%%%%%%%%%%%%%%%%%%%%%%%%%%%%%%%%%
\subsection{Goalie-Einzel}
\label{spielformen:sonderregeln:goalie}

\begin{itemize}
\item Benötigte Spieler: 2
\item Niveau: Anfänger
\item Aufstellung: 3er und 5er Stangen werden hochgeklappt und fixiert; ein Ball.
\item Regeln: Normale Regeln plus Sonderregel des Torwartbereichs: 
  \begin{itemize}
  \item Anstoß im Torwartbereich (siehe \nameref{spielformen:sonderregeln:abstoss}).
  \item Sobald der Ball den Torwartbereich verläßt und z.B. im mittleren Spielfeldbereich unerreichbar liegen bleibt, bekommt der Gegner den Ball und bringt den Ball neu ins Spiel.
  \item Springt der Ball aus dem Tisch, bekommt wie gewohnt derjenigen den Ball, der nicht rausgeschossen hat. 
  \end{itemize}
\item Trainingseffekt: defensive und offensive Grundlagen; Torwartdasein, wie Stellungsspiel, Ball stoppen und behalten; Ballführung und Spielfeldverständnis
\item Variationen: \nameref{spielformen:sonderregeln:mehrerebaelle}, \nameref{spielformen:sonderregeln:unterschiedlichebaelle}, \nameref{spielformen:sonderregeln:onetouch}, \nameref{spielformen:sonderregeln:speedball}
\item Info: Goalie-Einzel wird bei vielen großen Turnieren als Nebendisziplin angeboten. Also selbst die Profis haben viel Spaß bei dieser Spielform.
\end{itemize}


%%%%%%%%%%%%%%%%%%%%%%%%%%%%%%%%%%%%%%%%
\subsection{Abstoß}
\label{spielformen:sonderregeln:abstoss}

\begin{itemize}
\item Benötigte Spieler: 2 oder 4 
\item Niveau: Anfänger
\item Aufstellung: Einzel-/Doppelaufstellung, ein Ball.
\item Regeln: Normale Regeln, aber der Anstoß wird im Torwartbereich ausgeführt. 
\item Trainingseffekt: Vereinfachen des Spiels durch Weglassen des Anstosses auf der 5er-Reihe; Aufwertung des Torwarts im Doppel
%\item Variationen: \nameref{spielformen:sonderregeln:dreigegendrei}, \nameref{spielformen:sonderregeln:unterschiedlichebaelle}, \nameref{spielformen:sonderregeln:onetouch}, \nameref{spielformen:sonderregeln:speedball}
\end{itemize}

%%%%%%%%%%%%%%%%%%%%%%%%%%%%%%%%%%%%%%%%
\subsection{Mehrere Bälle}
\label{spielformen:sonderregeln:unterschiedlichebaelle}
Rollerball

%%%%%%%%%%%%%%%%%%%%%%%%%%%%%%%%%%%%%%%%
\subsection{Unterschiedliche Bälle}
\label{spielformen:sonderregeln:mehrerebaelle}
Rollerball


%%%%%%%%%%%%%%%%%%%%%%%%%%%%%%%%%%%%%%%%
\subsection{Elfmeterschießen}
\label{spielformen:sonderregeln:elfemeter}
Torwartduell (Goalie War) anstelle von Einzel (2)

%%%%%%%%%%%%%%%%%%%%%%%%%%%%%%%%%%%%%%%%
\subsection{One touch}
\label{spielformen:sonderregeln:onetouch}
Sofort schießen

%%%%%%%%%%%%%%%%%%%%%%%%%%%%%%%%%%%%%%%%
\subsection{Speedball}
\label{spielformen:sonderregeln:speedball}

\begin{itemize}
\item Benötigte Spieler: 2 oder 4 
\item Aufstellung: Einzel-/Doppelaufstellung, ein Ball.
\item Regeln: Man darf den Ball maximal drei Sekunden auf einer Stange halten, aber
nicht stoppen! Also direkt spielen oder innerhalb einer Stange passen und sofort
weiterspielen.
\item Trainingseffekt: Reaktion, Überraschungsschüsse, Spaß, Kreativität
\item Varianten: Speedball geht auch in den Varianten 3 gegen 3 oder 4 gegen 4.  
\item Info: In Italien gibt es viele Turniere mit Speedball-Regeln. Bei der Weltmeisterschaft 2015 in Turin wurde Speedball als Nebendisziplin ausgetragen. 
\end{itemize}

%%%%%%%%%%%%%%%%%%%%%%%%%%%%%%%%%%%%%%%%
\subsection{Nur eine Technik}
\label{spielformen:sonderregeln:nureinetechnik}



%%%%%%%%%%%%%%%%%%%%%%%%%%%%%%%%%%%%%%%%
\subsection{Pass-Bonus}
\label{spielformen:sonderregeln:passbonus}
Einzel und Doppel; Punktvergabe für Pässe

%%%%%%%%%%%%%%%%%%%%%%%%%%%%%%%%%%%%%%%%
\subsection{Zeit lassen}
\label{spielformen:sonderregeln:zeitlassen}
mind. 10 Sekunden 

%%%%%%%%%%%%%%%%%%%%%%%%%%%%%%%%%%%%%%%%
\subsection{Weitere Spiele}
\label{spielformen:sonderregeln:weiteres}

Rundschlagen der Stangen

%%%%%%%%%%%%%%%%%%%%%%%%%%%%%%%%%%%%%%%%
%%%%%%%%%%%%%%%%%%%%%%%%%%%%%%%%%%%%%%%%
%%%%%%%%%%%%%%%%%%%%%%%%%%%%%%%%%%%%%%%%
\section{Spielorganisation}

\subsection{Forderspiel (1 Tisch)}

Variante:

\subsection{Auf Zeit (1 Tisch)}

\subsection{DYP-Varianten}



%%%%%%%%%%%%%%%%%%%%%%%%%%%%%%%%%%%%%%%%%%%%%%%%%%%%%%%%%%%%%
% turniere 
\chapter{Turniere und Meisterschaften}
\label{turniere}

Turniere und Meisterschaften sind die sportlichen Wettkämpfe beim Tischfussball. 
Neben den Teilnehmern, die sich sportlich messen, ist die Organisation vor, während und nach einem Turnier oder einer Meisterschaft für einen gelungenen Wettkampf entscheidend.
In diesem Kapitel wird erklärt wie man einen Wettkampf vorbereitet (Kapitel \ref{turniere:vorbereitung}), ihn durchführt (Kapitel \ref{turniere:durchfuehrung}) und danach die Ergebnisse darstellt (Kapitel \ref{turniere:ergebnisse}).

%Der Fokus liegt hierbei auf einen Standort, wie ein Jugendhaus, eine Schule oder einen Betrieb, wo typischerweise 1 Tisch steht.
%An entsprechender Stelle wird auch auf größere Turniere mit mehreren Tischen und vielen Spielern, die über mehrer Tage gehen können, eingegangen. 
%Tabelle \ref{tab:turniere} zeigt eine Übersicht, was es für typische Turnierformate im Tischfßball gibt.

%\begin{table}
%\centering
%\begin{tabular}{p{1.3cm}|p{1.3cm}|p{1.3cm}|p{2cm}|p{2cm}|p{2cm}} 
%Tische 	& Teams & Dauer & Disziplinen & Durchführung & Reichweite \\ 
%\hline 
%\hline 
%1-2 	& 3-12 & 1-2 h & 1 & wöchentlich-monatlich & lokal \\ 
%\hline 
%bis 10 	& bis 50 & ca. 4 h & 1 & wöchentlich-monatlich & lokal \\ 
%\hline 
%bis 20 	& bis 100 & 6-10 h & 1 & monatlich & regional \\ 
%\hline 
%bis 20 	& bis 200 & 2 Tage & 6-8 & monatlich & regional \\ 
%\hline 
%bis 20 	& bis 100 & ca. 6-10 h & 1 & monatlich & Rangliste \\ 
%\hline 
%\end{tabular} 
%\caption{Caption}
%\label{tab:turniere}
%\end{table} 
 
%%%%%%%%%%%%%%%%%%%%%%%%%%%%%%%%%%%%%%%%
\section{Vorbereitung}
\label{turniere:vorbereitung}

Vor einem Turnier sollte sich der Organisator überlegen
\begin{itemize}
\item wie viele {\bf Teams} erwartet werden, 
\item wie viele {\bf Tische} zur Verfügung stehen und
\item wie viel Zeit man zur Durchführung haben wird ({\bf Dauer}).
\end{itemize}
Nach diesen Faktoren kann man einen \nameref{turniere:vorbereitung:modus} festlegen und die \nameref{turniere:vorbereitung:spielrunden} festlegen, damit die Teilnehmer ausreichend viel spielen können. 
Spätestens ab einer gewissen Turniergröße lohnt es sich, die Turnierplanung in einer sogenannten \nameref{turniere:vorbereitung:ausschreibung} festzuhalten, die die wichtigsten Informationen über den Wettkampf übersichtlich erhalten sollte.

%%%%%%%%%%%%%%%%%%%%%%%%%%%%%%%%%%%%%%%%
%%%%%%%%%%%%%%%%%%%%%%%%%%%%%%%%%%%%%%%%
\subsection{Spielmodus}
\label{turniere:vorbereitung:modus}

Je nach Teilnehmeranzahl, Tischkapazität und
\nameref{turniere:ergebnisse:rahmen}
gibt es verschiedene Wettkampfsformate.

\nameref{spielformen:zaehlweisen:einsatz}
\nameref{spielformen:zaehlweisen:gewinnsaetze}

\nameref{spielformen:npersonen:einzel}
\nameref{spielformen:npersonen:doppel} 
\nameref{spielformen:npersonen:team} 

%%%%%%%%%%%%%%%%%%%%%%%%%%%%%%%%%%%%%%%%
\subsubsection{Fordern}
\label{turniere:vorbereitung:modus:fordern}

\begin{itemize}
\item Spieleranzahl: 3-5 Teams pro Tisch
\item Dauer: ab einer halben Stunde
\item \nameref{spielformen:npersonen}: typischerweise \nameref{spielformen:npersonen:doppel} 
\item \nameref{spielformen:zaehlweisen}: typischerweise \nameref{spielformen:zaehlweisen:einsatz}
\item Modus: 
\item \nameref{turniere:ergebnisse:rahmen}:
typischerweise \nameref{turniere:ergebnisse:rahmen:training} 
\item Hintergrund:
\end{itemize}


%%%%%%%%%%%%%%%%%%%%%%%%%%%%%%%%%%%%%%%%
\subsubsection{DYP}
\label{turniere:vorbereitung:modus:dyp}

wechselnder Partner

%%%%%%%%%%%%%%%%%%%%%%%%%%%%%%%%%%%%%%%%
\subsubsection{Challenger}
\label{turniere:vorbereitung:modus:challenger}

Vorrunde und KO-Runden

%%%%%%%%%%%%%%%%%%%%%%%%%%%%%%%%%%%%%%%%
%\subsubsection{Meisterschaft}
%\label{turniere:vorbereitung:modus:challenger}

%%%%%%%%%%%%%%%%%%%%%%%%%%%%%%%%%%%%%%%%
\subsubsection{Liga}
\label{turniere:vorbereitung:modus:liga}

Mehrere Spieltage

%%%%%%%%%%%%%%%%%%%%%%%%%%%%%%%%%%%%%%%%
%%%%%%%%%%%%%%%%%%%%%%%%%%%%%%%%%%%%%%%%
\subsection{Spielrunden}
\label{turniere:vorbereitung:spielrunden}

\subsubsection{Vorrunde}
\label{turniere:vorbereitung:spielrunden:vorrunde}

Jeder gegen jeden
Feste Rundenzahl mit Zufall
Schweizer System

\subsubsection{KO-Runde}
\label{turniere:vorbereitung:spielrunden:ko}

Felder
Doppel-KO

\subsubsection{Hin- und Rückrunde}
\label{turniere:vorbereitung:spielrunden:hinrueck}

Jeder gegen jeden
Heim und Auswärts


%%%%%%%%%%%%%%%%%%%%%%%%%%%%%%%%%%%%%%%%
\subsection{Ausschreibung}
\label{turniere:vorbereitung:ausschreibung}

%%%%%%%%%%%%%%%%%%%%%%%%%%%%%%%%%%%%%%%%
%%%%%%%%%%%%%%%%%%%%%%%%%%%%%%%%%%%%%%%%
\section{Durchführung}
\label{turniere:durchfuehrung}
Software/App? (Kickertool.de) / Papier

\subsection{Anmeldung}
\label{turniere:durchfuehrung:anmeldung}

\subsection{Turnierphase}
\label{turniere:durchfuehrung:turnierphase}

\subsection{Siegerehrung}
\label{turniere:durchfuehrung:siegerehrung}





%%%%%%%%%%%%%%%%%%%%%%%%%%%%%%%%%%%%%%%%
%%%%%%%%%%%%%%%%%%%%%%%%%%%%%%%%%%%%%%%%
\section{Ergebnisse}
\label{turniere:ergebnisse}

%%%%%%%%%%%%%%%%%%%%%%%%%%%%%%%%%%%%%%%%
%%%%%%%%%%%%%%%%%%%%%%%%%%%%%%%%%%%%%%%%
\subsection{Rahmen}
\label{turniere:ergebnisse:rahmen}

\subsection{Training}
\label{turniere:ergebnisse:rahmen:training}

\subsection{Ranglisten}
\label{turniere:ergebnisse:rahmen:rangliste}
Forderpyramide

\subsection{Meisterschaft}
\label{turniere:ergebnisse:rahmen:meisterschaft}

Turnier einmal im Jahr: Jugendhausmeister, Schulmeister, Vereinmeister, Landesmeister, Deutsche Meister, Weltmeister

Über eine Saison, z.B. Liga


%%%%%%%%%%%%%%%%%%%%%%%%%%%%%%%%%%%%%%%%
\subsection{Darstellung}
\label{turniere:ergebnisse:formate}

Ehrentafel/Gewinnerfotos
Einzelergebnisse
Aktualisierte Tabelle oder Rangliste  

%%%%%%%%%%%%%%%%%%%%%%%%%%%%%%%%%%%%%%%%
\subsection{Preise}
\label{turniere:ergebnisse:preise}

Wanderpokale
Medaillen und Pokale
Urkunden
T-Shirts





%%%%%%%%%%%%%%%%%%%%%%%%%%%%%%%%%%%%%%%%
%\section{Modus für ein Jugendstandort}


%%%%%%%%%%%%%%%%%%%%%%%%%%%%%%%%%%%%%%%%%%%%%%%%%%%%%%%%%%%%%
% weiteres
\chapter{Tischfußball-Standorte und Strukturen}
\label{weiteres}


\section{Standorte}

Fokus:
\begin{itemize}
\item Kicker-Gruppe im Jugendhaus
\item Kicker-AG in der Schule
\item Betriessportangebot
\item Vereinstraining einer Anfängergruppe  
\end{itemize}

\subsection{Rahmenbedingungen}

\begin{itemize}
\item Teilnehmer: 2-20 und 1-2 Betreuer/Trainer
\item Tischanzahl: 1-2 Tische
\item Termine pro Woche: 1-2mal je 1,5- 2 Stunden
\end{itemize}

\subsection{Regelmäßiges Programm}

\begin{itemize}
\item Training (standardmäßig): Spielformen, Trainingsturniere, Inhalte abseits vom Tisch   
\item Ranglistenturnier (monatlich): 
Rangliste  
\item Jahresmeisterschaft (jährlich):
\end{itemize}  

\paragraph{Inhalte abseits des Tischs}
\begin{itemize}
\item Tischpflege:  
\item Bewegungsangebot: 
\item Grillen:
\item Freundschaftsspiele:
\item Teamname und Logo:
\item Profispiel anschauen: Video 
\end{itemize} 

\subsection{Empfehlungen}
\begin{itemize}
\item Kicker-Gruppe im Jugendhaus: 
Kooperation mit Verein,
Teilnahme an der Jugenhausmeisterschaft,  
optionale Teilnahme Landesmeisterschaft und einer Landesliga
\item Kicker-AG in der Schule: 
Kooperation mit Verein, 
Teilnahme an der Schulmeisterschaft, 
optionale Teilnahme Landesmeisterschaft und einer Landesliga
\item Vereinstraining einer Anfängergruppe: 
Kooperation mit einem Jugendhaus oder einer Schule durch personelle Unterstüzung einer Kicker-Gruppe bzw. Kicker-AG, 
Teilnahme Landesmeisterschaft und einer Landesliga/Bundesliga
\item Betriessportangebot: 
Kooperation mit Verein, 
Teilnahme an der Betriebssportmeisterschaft und Liga,
\end{itemize}

%%%%%%%%%%%%%%%%%%%%%%%%%%%%%%%%%%%%%%%%%

\section{Kontakte und Verbände}

\subsection{Landesverband}

\begin{itemize}
\item Vereine
\item Spieler und Spielernummer (Spielerpass)
\item Turnierergebnisse von Ranglistenturnier 
\item Landesliga und Landesmeisterschaft
\end{itemize}

\subsection{DTFJ}

\begin{itemize}
\item Idee und Ziele
\item Status, \cite{ehring2010}
\end{itemize}

\subsection{DTFB}

\gls{dtfb}:
\begin{itemize}
\item Landesverbände, Vereine, Spieler
\item Turnierergebnisse von Ranglistenturnier 
\item Bundesligen (DTFL): Herren, Damen, Senioren und Jugend; und Deutsche Meisterschaft am Ende der Saison meist im November.
\item Nationalmannschaften
\end{itemize}

Jugend:
\begin{itemize}
\item U18 Nationalmannschaft mit Nationalcoach
\item Kader, erweiterter Kader und Sichtungskader
\item Jährlicher Kaderlehrgang
\end{itemize}

\subsection{ITSF Europe}

European Championsleague

\subsection{ITSF}

\begin{itemize}
\item Nationale Verbände
\item Weltranglistenturniere
\item Weltmeisterschaften: Einzel, Doppel, Nationa-Team
\item Regelkommission
\end{itemize}

%\section{Abzeichen}
% (Level, Altersklassen)
%Verbindung mit Regeln z.B. Gold darf keine Einmannpässe spielen...
%Handicaps im allgemeinen



%%%%%%%%%%%%%%%%%%%%%%%%%%%%%%%%%%%%%%%%%%%%%%%%%%%%%%%%%%%%%%%%%%%%%
% Abkürzungen und Begriffe
\printglossary[type=\acronymtype, title=Abkürzungen, toctitle=Abkürzungen] % prints just the list of acronyms

\printglossary[title=Begriffe, toctitle=Begriffe] % if no option is supplied the default glossary is printed.


%%%%%%%%%%%%%%%%%%%%%%%%%%%%%%%%%%%%%%%%%%%%%%%%%%%%%%%%%%%%%%%%%%%%%%
% Bibliography %%%%
%\cleardoublepage
\renewcommand{\bibname}{Literatur und Links}
\addcontentsline{toc}{chapter}{Literatur und Links}
%\bibliographystyle{authordate3}
%\bibliographystyle{plainnat}
%\bibliographystyle{abbrvnat} 
%\bibliographystyle{kluwer} 
\bibliographystyle{bib_jde} 
\bibliography{quellen}


%%%%%%%%%%%%%%%%%%%%%%%%%%%%%%%%%%%%%%%%%%%%%%%%%%%%%%%%%%%%%%%%%%%%%%
% Impressum
%%%%%%%%%%%%%%%%%%%%%%%%%%%%%%%%%%%%%%%%%%%%%%%%%%%%%%%%%%%%%%%%%%%%%%
% Impressum
\thispagestyle{empty}


\addcontentsline{toc}{chapter}{Impressum}

\null
\vfill
\vfill
{\raggedright \normalfont \bfseries Impressum}
\vspace{0.5cm}

{
  \raggedright

\begin{tabular}{ll}
  Autoren: & Lukas, Jan 
    \\%[3ex]
  Co-Autoren: & Erik, Fabian, Jana
    \\%[3ex]
  Ort: &  Hamburg
    \\%[3ex]
  Datum der Veröffentlichung: & XX XX XX
    \\%[3ex]
  Kontakt: & jugendarbeit@dtfb.de
    \\%[3ex]
\end{tabular}
}
\vfill



\end{document}
