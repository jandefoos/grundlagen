\chapter{Einleitung}

\section{Spielprinzip}

Das Runde muss ins Eckige! Wie beim Fußball ist beim Tischfußball, also dem Fußball auf dem Tisch, das Ziel eines Teams mit seinen elf Spielern den Ball ins gegnerische Tor zu bringen.  
Im Gegensatz zum Fußball sind es meist aber nur zwei Menschen, die in einem Tischfußball-Team, dem sogenannten Doppel, gegen ein anderes Doppel spielen.
Jede Spielerin und jeder Spieler kann durch Bewegen und Drehen der Stangen mit den daran befestigten Spielfiguren den Ball beeinflussen und Tischfußball spielen.
Dabei gibt es natürlich wie bei jeden anderen Sport bestimmte Spielregeln:
\begin{itemize}
\item Der Tisch und die Bälle unterliegen bestimmten Anforderungen.
\item Es gibt Regeln, die festlegen, was erlaubt beim Spielen ist und was ein Fouls sein kann.
\item Es gibt Aufstellungsregeln bei verschiedenen Spiel- und Turnierformen. 
\end{itemize}


\section{Mehr als nur ein Spiel}

\begin{itemize}
\item Spiel für jederfrau und jedermann, für Jung und Alt. 
\item Soziale Aspekte: Zusammenkommen und Kennenlernen, modernes Vereinsleben.
\item Individuelle Entwicklung: Motorik, Konzentration, Verlieren lernen
\end{itemize}

\section{Tischfußball als Sport}

\begin{itemize}
\item Vereine und Verbände
\item Einzel, Doppel, Team
\item Turniere und Ranglisten
\item Weltmeisterschaften, Deutsche Meisterschaften, Landesmeisterschaften, ...
\end{itemize}

