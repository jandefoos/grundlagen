\chapter{Einleitung}
\label{einleitung}

\section{Spielprinzip}
\label{einleitung:prinzip}

Das Runde muss ins Eckige! Wie beim Fußball ist beim Tischfußball, also dem Fußball auf dem Tisch, das Ziel eines Teams mit seinen elf Spielern den Ball ins gegnerische Tor zu bringen.  
Im Gegensatz zum Fußball sind es meist aber nur zwei Menschen, die in einem Tischfußball-Team, dem sogenannten Doppel, gegen ein anderes Doppel spielen.
Jede Spielerin und jeder Spieler kann durch Bewegen und Drehen der Stangen mit den daran befestigten Spielfiguren den Ball beeinflussen und Tischfußball spielen.
Dabei gibt es natürlich wie bei jeden anderen Sport bestimmte Spielregeln:
\begin{itemize}
\item Der Tisch und die Bälle unterliegen bestimmten Anforderungen.
\item Es gibt Regeln, die festlegen, was erlaubt beim Spielen ist und was ein Fouls sein kann.
\item Es gibt Aufstellungsregeln bei verschiedenen Spiel- und Turnierformen. 
\end{itemize}


\section{Mehr als nur ein Spiel}
\label{einleitung:spiel}

Tischfußball ist ein Spiel für jedermann: Egal ob Junge oder Mädchen, Frau oder Mann und Jung oder Alt, das Spiel kann jeder spielen und erlernen. Sogar für Menschen mit Behinderung, die im Rollstuhl sitzen, gibt es extra dafür angepasste Tische. 

Beim Tischfußball lernt man sich und den Gegenüber ganz schnell auf spielerische Art und Weise kennen. 
Einer der beiden Hauptgründe, warum Tischfußball fasziniert, ist das gemeinsame Zusammenkommen und der geminsame Spaß. 
Bei dem sozialen Zusammenkommen sind Inhalte wie Respekt vor dem Anderen, Verlieren lernen, Fair Play und Konflikte lösen neben dem gemeinsamen Spaß ein Bestandteil. 

Der zweite Hauptgrund ist die Herausforderung: Nicht nur der Wettkampf mit dem Gegenüber kann motivieren, sondern auch die Herausforderung an sich selbst.
Neben der Forderung der motorischen und koordinativen Fähigkeiten werden auch die Konzentration und der individuelle Lernspaß gefördert.       

%\begin{itemize}
%\item Spiel für jederfrau und jedermann, für Jung und Alt. 
%\item Soziale Aspekte: Zusammenkommen und Kennenlernen, modernes Vereinsleben.
%\item Individuelle Entwicklung: Motorik, Konzentration, Verlieren lernen
%\end{itemize}

\section{Tischfußball als Sport}
\label{einleitung:sport}

Diese individuellen und sozialen Entwicklungsmöglichkeiten sind die beste Grundlage, um Tischfußball als Sport auszuüben. 
In den letzten Jahren haben sich auf allen Ebenen Breiten- und Leistungssport-Strukturen etalbliert:
\begin{itemize}
\item Es gibt viele Tischfußball-Gruppen in Jugendhäusern, Schulen, Universitäten und Betrieben.
\item Es gibt Sport-Vereine und Verbände, die den Sport organisieren.
\item Es gibt drei Haupt-Disziplinen: Einzel, Doppel und Team.
\item Es gibt offizielle Turniere und Ranglisten.
\item Und es gibt Landesmeisterschaften, Deutsche Meisterschaften und Weltmeisterschaften.
\end{itemize}

