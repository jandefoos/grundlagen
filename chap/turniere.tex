\chapter{Turniere und Meisterschaften}
\label{turniere}

Turniere und Meisterschaften sind die sportlichen Wettkämpfe beim Tischfussball. 
Neben den Teilnehmern, die sich sportlich messen, ist die Organisation vor, während und nach einem Turnier oder einer Meisterschaft für einen gelungenen Wettkampf entscheidend.
In diesem Kapitel wird erklärt wie man einen Wettkampf vorbereitet (Kapitel \ref{turniere:vorbereitung}), ihn durchführt (Kapitel \ref{turniere:durchfuehrung}) und danach die Ergebnisse darstellt (Kapitel \ref{turniere:ergebnisse}).



%Der Fokus liegt hierbei auf einen Standort, wie ein Jugendhaus, eine Schule oder einen Betrieb, wo typischerweise 1 Tisch steht.
%An entsprechender Stelle wird auch auf größere Turniere mit mehreren Tischen und vielen Spielern, die über mehrer Tage gehen können, eingegangen. 
%Tabelle \ref{tab:turniere} zeigt eine Übersicht, was es für typische Turnierformate im Tischfußball gibt.

%\begin{table}
%\centering
%\begin{tabular}{p{1.3cm}|p{1.3cm}|p{1.3cm}|p{2cm}|p{2cm}|p{2cm}} 
%Tische 	& Teams & Dauer & Disziplinen & Durchführung & Reichweite \\ 
%\hline 
%\hline 
%1-2 	& 3-12 & 1-2 h & 1 & wöchentlich-monatlich & lokal \\ 
%\hline 
%bis 10 	& bis 50 & ca. 4 h & 1 & wöchentlich-monatlich & lokal \\ 
%\hline 
%bis 20 	& bis 100 & 6-10 h & 1 & monatlich & regional \\ 
%\hline 
%bis 20 	& bis 200 & 2 Tage & 6-8 & monatlich & regional \\ 
%\hline 
%\end{tabular} 
%\caption{Caption}
%\label{tab:turniere}
%\end{table} 
 
%%%%%%%%%%%%%%%%%%%%%%%%%%%%%%%%%%%%%%%%
\section{Vorbereitung}
\label{turniere:vorbereitung}

Vor einem Turnier sollte sich der Organisator überlegen
\begin{itemize}
\item was für eine Wertigkeit hat das Turnier: Training, Rangliste, Meisterschaft?
\item wie viele {\normalfont \bfseries Teams} erwartet werden, 
\item wie viele {\normalfont \bfseries Tische} zur Verfügung stehen und
\item wie viel Zeit man zur Durchführung haben wird ({\normalfont \bfseries Dauer}).
\end{itemize}
Nach diesen Faktoren legt man einen \nameref{turniere:vorbereitung:modus} fest. 
Ab einer gewissen Turniergröße lohnt es sich, die Turnierplanung in einer sogenannten \nameref{turniere:vorbereitung:ausschreibung} festzuhalten, die die wichtigsten Informationen über den Wettkampf übersichtlich beinhalten sollte.

%%%%%%%%%%%%%%%%%%%%%%%%%%%%%%%%%%%%%%%%
%%%%%%%%%%%%%%%%%%%%%%%%%%%%%%%%%%%%%%%%
\subsection{Spielmodus}
\label{turniere:vorbereitung:modus}

Je nach Teilnehmeranzahl, Tischkapazität, zeitlichen Rahmen und
\nameref{turniere:ergebnisse:rahmen}
gibt es verschiedene Wettkampfsformate.


%%%%%%%%%%%%%%%%%%%%%%%%%%%%%%%%%%%%%%%%
\subsubsection{Gewinner bleibt}
\label{turniere:vorbereitung:modus:fordern}

\begin{itemize}
\item Spieleranzahl: 3-5 Teams pro Tisch
\item Dauer: ab einer halben Stunde
\item \nameref{spielformen:npersonen}: typischerweise \nameref{spielformen:npersonen:doppel}, insbesondere als \nameref{turniere:vorbereitung:modus:dyp}-Variante
\item \nameref{spielformen:zaehlweisen}: typischerweise \nameref{spielformen:zaehlweisen:einsatz}
\item Modus: Es gibt eine Startreihenfolge und die ersten beiden Teams spielen gegeneinander. Der Gewinner einer Partie bleibt am Tisch stehen und das nächste Team in der Reihe ist der nächste Gegner. Das ausscheidende Team stellt sich hinten an. 
\item Gewinner: Das Team, welches hintereinander eine vorher bestimmte Anzahl an Spielen gewonnen hat, gewinnt das Turnier.  
\item \nameref{turniere:ergebnisse:rahmen}:
Typischerweise im \nameref{turniere:ergebnisse:rahmen:training}. Zusätzlich kann eine \nameref{turniere:ergebnisse:rahmen:rangliste} angelegt werden, für die die Gewinner 1 Punkt erhalten. 
\item Hintergrund: Als ,,Fordern'' ist dieses Spiel auch im freien Spiel bekannt: Spielen zwei Teams ein Spiel, gilt das ungeschriebene Gesetz ,,es darf gefordert werden''. Durch das Klopfen mit der Hand auf den Tischrand, fordert ein wartendes Team den Gewinner der Partie. 
\end{itemize}


%%%%%%%%%%%%%%%%%%%%%%%%%%%%%%%%%%%%%%%%
\subsubsection{DYP}
\label{turniere:vorbereitung:modus:dyp}

\begin{itemize}
\item Spieleranzahl: 3-5 Teams pro Tisch
\item Dauer: ab einer halben Stunde
\item \nameref{spielformen:npersonen}: \nameref{spielformen:npersonen:doppel} mit wechselnden Partner
\item \nameref{spielformen:zaehlweisen}: typischerweise \nameref{spielformen:zaehlweisen:einsatz}
\item Modus: 
Wenn man \nameref{turniere:vorbereitung:modus:fordern} oder eine Vorrunde spielt wird in jeder Runde ein neuer Spielpartner zugelost, das heißt in jeder Runde gibt es neue Doppelkombination. \\
Bei der \nameref{turniere:vorbereitung:modus:fordern} wird eine Liste geführt und die Spieler des Verlieredoppels losen, wer zuerst eingetragen wird und mit dem nächsten Losgewinner spielt. \\
Bei einer \nameref{turniere:vorbereitung:spielrunden:vorrunde} sammelt jeder Spieler für sich Punkte. Nach jeder Runde werden alle Spieler neu zusammengelost, z.B. mit Zetteln zum ziehen oder einer geeigneten Turnier-Software.
\item Gewinner: Das Team, welches hintereinander eine vorher bestimmte Anzahl an Spielen gewonnen hat, gewinnt das Turnier.  
\item \nameref{turniere:ergebnisse:rahmen}:
typischerweise als \nameref{turniere:ergebnisse:rahmen:training}s-Turnier
\item Trainingseffekt: Durch die wechselnden Partner lernt man mit verschiedenen Spieler gut zusammen zu doppeln. D.h. sich abzusprechen, in welcher Konstellation man spielt oder wie man verteidigt.
\item Hintergrund: DYP ist die Abkürzung für ,,Draw Your Partner'', was übersetzt so viel heißt wie ,,zieh deinen Partner''. 
\end{itemize}

%%%%%%%%%%%%%%%%%%%%%%%%%%%%%%%%%%%%%%%%
\subsubsection{Einzel/Doppel-Turnier}
\label{turniere:vorbereitung:modus:turnier}

\begin{itemize}
\item Spieleranzahl: 3-5 Teams pro Tisch
\item Dauer: ab einer halben Stunde bis 6 Stunden
\item \nameref{spielformen:npersonen}: \nameref{spielformen:npersonen:doppel} oder \nameref{spielformen:npersonen:einzel} 
\item Modus: Ein klassisches Turnier besteht aus einer Vorrunde und einer anschliessenden KO-Runde. Man kann aber auch nur eine Vorrunde oder nur eine KO-Runde spielen. \\
In einer \nameref{turniere:vorbereitung:spielrunden:vorrunde} hat jedes Team gleich viele Spiele und die Ergebnisse fließen in eine Tabelle ein. 
Der Tabellenführer nach einer vorher abgemachten Rundenanzahl gewinnt. \\
Für eine \nameref{turniere:vorbereitung:spielrunden:ko} braucht man einen angepassten KO-Baum. Die Gewinner kommen eine Runde weiter, die Verlierer scheiden aus. 
\item \nameref{turniere:ergebnisse:rahmen}:
\nameref{turniere:ergebnisse:rahmen:training}s-, \nameref{turniere:ergebnisse:rahmen:rangliste}n oder \nameref{turniere:ergebnisse:rahmen:meisterschaft}s-Turnier
\end{itemize}

%%%%%%%%%%%%%%%%%%%%%%%%%%%%%%%%%%
\paragraph{Vorrunde:}
\label{turniere:vorbereitung:spielrunden:vorrunde}

Während einer Vorrunde spielt man typischerweise einen Satz (siehe \nameref{spielformen:zaehlweisen:einsatz}) oder \nameref{spielformen:zaehlweisen:gewinnsaetze}. 
Es gibt verschiedene Varianten, wie viele Runden und nach welchem ster man diese Runden spielt:
\begin{itemize}
\item Jeder gegen jeden: 
Jedes Team spiel gegen jedes Team einmal. 
Zum Beispiel bei 4 Teams sind das insgesamt schon 3+2+1 = 6 Spiele. 
Dieser Modus bietet sich daher für kleinere Gruppen an.
\item Gruppen: 
Wie bei der Fußball-Weltmeisterschaft bildet man zum Beispiel 4er-Gruppen, d.h. 4 Teams pro Gruppe, die jeweils gegeneinander Spielen (Jeder gegen jeden). 
So können die Gruppen parallel ihre Vorrundenspiele durchführen.
\item Zufallsgeloste Runden:
Pro Vorrunde werden die Begegnungen gelost. Falls es einen ungerade Anzahl von Teams ist, gibt es ein Freilos, d.h. das Team, welches das Freilos zugelost bekommt, gewinnt das Spiel autoamtisch in dieser Runde. 
\item Schweizer System:
Bei dieser Varianten werden die Vorrunden ebenfalls gelost. Für die nächste Runde wird jedoch jeweils der aktuelle Tabellenstand mit einbezogen, so dass Teams, die in derselben Tabellenregion bzw. die gleiche Punktzahl haben, eher zusammengelost werden.
So wird sichergestellt, dass pro Runde jeweils etwa gleichstarke Teams gegeneinander spielen, und dass es vor allem nach mehreren Runden stets spannende Spiele gibt.  
\end{itemize}

Anhand der Vorrundenspiele berechnet man eine Tabelle. Dabei gelten wie bei anderen Mannschaftssportarten zur Berechnung der Platzierung
\begin{itemize}
\item die Punkte (3 oder 2 Punkte pro Sieg und 1 Punkt pro Unentschieden), 
\item bei Punktlgeichheit Tore (Differenz, dann die mehr geschossenen Tore),
\item und bei Torgleichheit der direkte Vergleich.
\end{itemize}
Ausnahme: Beim Schweizer System wird die Tabelle nach Punktgleichheit durch zwei sogenannter Buchholz-Faktoren berechnet, in die rückwirkend die erreichten Punkte der Gegner gehen. 


%%%%%%%%%%%%%%%%%%%%%%%%%%%%%%%%%%%%%%%%%%%%5
\paragraph{KO-Runde:}
\label{turniere:vorbereitung:spielrunden:ko}

,,KO`` steht für englisch ,,knock out`` und bedeutet, wer verliert ist raus, wer gewinnt kommt eine Runde weiter bis zum Sieg im Finale. 


\begin{itemize}
\item Einfach-KO: Anhand der Teilnehmeranzahl wird der Spielbaum bestimmt: Halbfinale bei bis zu 4 Teams, Viertelfinale bei 8, Achtelfinale bei 16, Sechszehntelfinale bei 32 usw. 
Wurde eine Vorrunde gespielt, wird die Teilnehmerzahl dementsprechend mit Freilosen aufgefüllt. Anhand der Vorrundentabelle werden die Begegnungen für die erste KO-Runde gesetzt.
Z.B. 20 Teams haben eine Vorrunde gespielt, daher wird ein 32er-Baum gespielt und im Sechszehntelfinale spielt der 1. gegen den 32., der 2. gegen den 31. usw.
Bei 20 Teams gibt es jedoch 12 Freilose, so dass die ersten 12 platzierten das Sechzehntelfinale automatisch gewinnen. So lauten die ersten Spiele: 13.-20., 14.-19., 15.-18., 16.-17.  
\item Felder: Hierbei werden mehrer Eindach-KOs gespielt, in dem man die Vorrundentabelle in verschiedene Felder aufteilt. Z.B. die ersten 8 Platzierten spielen eine KO-Runde und die Plätze 17 bis 32 auch. 
So spielen Teams mit ähnlichem Niveau jeweils eine Platzierungsrunde. 
\item Doppel-KO: Beim Doppel-KO gibt es neben dem eben beschriebenen Einfach-KO (Gewinnerrunde) einen zweiten KO-Baum (Verliererrunde). Jedes Team startet und spielt in der Gewinnerrunde. Falls ein Team verliert, kommt es in die parallel laufende Verlierrunde. Die Gewinner der Gewinnerrunde und der Verliererrunde spielen am Ende eine Finale.
Dieser Modus hat den Vorteil, das jedes Team sich einen Ausrutscher erlauben darf und trotzdem noch das Turnier gewinnen kann. Daher wird ein Doppel-KO oft auch ohne Vorrunde gespielt.
Der Nachteil kann die weniger gut planbare Länge des Turniers sein, da die Verliererunde doppelt so viele Spielrunden wie die Gewinnerrunde haben.   
\end{itemize}


%%%%%%%%%%%%%%%%%%%%%%%%%%%%%%%%%%%%%%%%
\subsubsection{Team-Liga}
\label{turniere:vorbereitung:modus:liga}

Der Liga-Modus wird hauptsächlich für \nameref{spielformen:npersonen:team}s angeboten und erstreckt sich meist über eine Saison, also ein Jahr. Der Modus entspricht etwa dem Modus der Fußball-Bundesliga: 
\begin{itemize}
\item Spieleranzahl: Ab 8 Teams mit Heimspielstätte
\item Dauer: über eine Saison
\item \nameref{spielformen:npersonen}: \nameref{spielformen:npersonen:team} 
\item Modus: 
\begin{itemize}
\item In einer Grupper oder Liga spielt jeder gegen jeden.
\item Es gibt eine Hin- und Rückrunde, also ein Heim- und ein Auswärtsspiel pro gegnerisches Team.
\item Die Spielergebnisse fließen in eine Tabelle (Punkte, Spiele, Tore). 
\end{itemize}
Wer nach allen Spielen an der Tabellenspitze steht, gewinnt. 
\item \nameref{turniere:ergebnisse:rahmen}:
\nameref{turniere:ergebnisse:rahmen:meisterschaft}
\end{itemize}



%%%%%%%%%%%%%%%%%%%%%%%%%%%%%%%%%%%%%%%%
\subsection{Ausschreibung}
\label{turniere:vorbereitung:ausschreibung}

Nach dem der Modus und der Termin für ein Wettkampf festgelegt wurde, sollte man die Infos zusammenfassen (Ausschreibung). Wichtige Informationen sind:
\begin{itemize}
\item Ausrichter: Name des Vereins, Verbands, Jugendhaus, ..
\item Termin: Datum und Uhrzeit des Turnierbeginns
\item Ort: Adresse
\item Ansprechpartner für weitere Infos und Fragen: Organisator
\item Disziplin: Einzel oder Doppel. Zusätzlich kann man Spezial-Turniere nur für Damen, Herren, Jugendliche oder Senioren anbieten. Wie beim Tennis, gibt es auch manchmal eine Mixed-Disziplin, in der ein Doppel-Team aus einer Dame und einem Herr sich zusammen setzt.   
\item Regeln: ITSF Regeln, eventuell angepasst.
\item Spielmodus: Rundenanzahl in der Vorrunde und KO-Rundenmodus.
\item Tische: Art und vor allem Anzahl der Tische. 
\item Voranmeldung: Sollten voraussichtlich viele Teams zu einem Turnier kommen, kann sich eine maximale Anzahl der Teams und eine Voranmeldung lohnen, um eine ausgewogene und ausreichende Spielzeit für alle Teams zu gewährleisten. 
Die maximale Anzahl der Teams hängt letztendlich mit der Tischkapazität, dem gewählten Spielmodus und dem verfügbaren Zeitrahmen ab.  
\item Preise: Etwa Urkunden oder Pokale für die ersten drei Plätze.
\end{itemize}
Diese Informationen kann man frühzeitig bekannt geben, damit für alle der Wettkampfrahmen klar ist.
Die Informationen kann man auch für ein Plakat verwenden, um das Turnier besser zu bewerben. 


%%%%%%%%%%%%%%%%%%%%%%%%%%%%%%%%%%%%%%%%
%%%%%%%%%%%%%%%%%%%%%%%%%%%%%%%%%%%%%%%%
\section{Durchführung}
\label{turniere:durchfuehrung}

Eine erfolgreiche Durchführung eines Turniers bedarf einer guten Planung und einer wachsamen \gls{turnierleitung}, die dafür sorgt, dass die Anmeldung, die Spielphase und die Siegerehrung reibungslos verläuft. 
Dafür braucht man vor allem einen guten Überblick, wer die Teams sind, welche Begegnungen anstehen und wie die Ergebnisse der Begegnungen sind. 

Bei wenigen Teams und einem einfachen Modus kann man als \gls{turnierleitung} die Begegnungen mit ,,Stift und Papier`` festhalten. Dafür gibt es im Internet auch einige Spielplan-Vorlagen, zum Beispiel bei \href{http://www.tischfussball-online.com/dies-das/turnierplaene.html}{tischfussball-online.de}. In der Regel wird eine Computer-Software dafür verwendet, um bei vielen Teilnehmern die Übersicht zu behalten und die Ergebnisse gleich online stellen zu können. Folgende Programme sind empfehlenswert:
\begin{itemize}
\item \href{http://kickertool.de/}{Kickertool}: Umfangfreiche, intuitive und kostenlose Online-Software mit allen gängigen Modi. Eine Offline-Version ist laut der Entwickler in Planung.
\item \href{http://www.heise.de/download/kickermaschine-1191093.html}{Kickermaschine:} Java-basierte, herunterladbare Software für Kicker-Turniere mit allen gängigen Modi.
\item TiFu: Die gängigste Software bei \gls{dtfb}-Turnieren (vor allem  Challengern) mit gängigen Modi, vorzeitigem Losen in der Vorrunde nach Schweizer System, die zudem eine Schnittstelle zum Sportsmanager anbietet. Kontaktperson: Christoph Hardt (\gls{dtfb}).
\item Sportsmanager: Joomla-Plugin zur Abwicklung von Teamwettbewerbe, Spielerverwaltung für  Vereine, Teams und Ranglisten und Ergebnisdarstellung, was vom \gls{dtfb} und vielen Landesverbände und Vereinen genutzt wird.  Kontaktperson: Sven Nickel (\gls{dtfb})
\item \href{http://www.table-soccer.org/page/fast}{FAST:} Vom ITSF entwickelt, für dessen Turniere. FAST steht kostenlos zur Verfügung, ist jedoch eine umfangreiche Software, da sie auch eine Spielerverwaltung für Turnierserien und Ranglisten mit sich bringt.
\end{itemize}

\subsection{Anmeldung}
\label{turniere:durchfuehrung:anmeldung}

Bevor das Turnier beginnt läuft die Anmeldung vor Ort. Bis kurz vor Turnierbeginn, sollten sich alle Teilnehmer bei der Turnieleitung anmelden. Die Turnierleitung sollte beachten, dass man nach Anmeldeschluss noch Zeit braucht, um anhand der Anmeldeliste die erste Spielrunde zu starten. Bei manchen Turnier-Software kann man sogar nach Turnierbeginn noch Teilnehmer hinzufügen, falls man das als Turnierleitung zu lassen will.

\subsection{Spielphase}
\label{turniere:durchfuehrung:turnierphase}

Um einen möglichst reibungslosen Verlauf zu gewähren sind zwei Dinge im Wesentlichen zu beobachten:
\begin{itemize}
\item Ergebnismeldung: Das Gewinnerteam sollte dafür verantwortlich sein, das Ergebnis bei der Turnierleitung zu melden. 
\item Spielaufruf: Nach der Ergebnismeldung kann die Turnierleitung den freien Tisch mit der nächsten Partie belegen. 
Verwendet man eine Turnier-Software werden die anstehenden Spiele autoamtisch so sortiert, dass das Turnier am schnellsten fortschreitet.
Für den Spieleraufruf lohnt sich bei größeren Gruppen eine Beschallungsanlage mit Mikrofon. 
In jedem Fall lohnt sich bei Verwendung einer Turnier-Software ein großer Monitor oder ein Beamer, um den Turnierverlauf für alle anzuzeigen.
\end{itemize}


\subsection{Siegerehrung}
\label{turniere:durchfuehrung:siegerehrung}


%%%%%%%%%%%%%%%%%%%%%%%%%%%%%%%%%%%%%%%%
%%%%%%%%%%%%%%%%%%%%%%%%%%%%%%%%%%%%%%%%
\section{Ergebnisse}
\label{turniere:ergebnisse}

%%%%%%%%%%%%%%%%%%%%%%%%%%%%%%%%%%%%%%%%
%%%%%%%%%%%%%%%%%%%%%%%%%%%%%%%%%%%%%%%%
\subsection{Wettkampfrahmen}
\label{turniere:ergebnisse:rahmen}

\subsubsection{Training}
\label{turniere:ergebnisse:rahmen:training}

\subsubsection{Rangliste}
\label{turniere:ergebnisse:rahmen:rangliste}

\begin{itemize}
\item Punkteschlüssel, maximale Turnierzahl, Turniersaison
\item Forderpyramide
\end{itemize}

\subsubsection{Meisterschaft}
\label{turniere:ergebnisse:rahmen:meisterschaft}

\begin{itemize}
\item Einzel/Doppel-Turnier einmal im Jahr: Jugendhausmeister, Schulmeister, Vereinmeister, Landesmeister, Deutsche Meister, Weltmeister
\item Team: Über eine Saison, z.B. Liga oder Pokal
\end{itemize}


%%%%%%%%%%%%%%%%%%%%%%%%%%%%%%%%%%%%%%%%
\subsection{Darstellung}
\label{turniere:ergebnisse:formate}

\begin{itemize}
\item Gewinnerfotos, Ehrentafel
\item Einzelergebnisse
\item Aktualisierte Tabelle oder Rangliste  
\end{itemize}

%%%%%%%%%%%%%%%%%%%%%%%%%%%%%%%%%%%%%%%%
\subsection{Preise}
\label{turniere:ergebnisse:preise}

\begin{itemize}
\item Wanderpokale
\item Medaillen und Pokale
\item Urkunden
\item T-Shirts
\end{itemize}

%%%%%%%%%%%%%%%%%%%%%%%%%%%%%%%%%%%%%%%%
\section{Beispiele}
\label{turniere:beispiele}


%%%%%%%%%%%%%%%%%%%%%%%%%%%%%%%%%%%%%%%%%%%%5
\subsection{Jugendhausturnier}
\label{turniere:beispiele:jugend}




%%%%%%%%%%%%%%%%%%%%%%%%%%%%%%%%%%%%%%%%%%%%5
\paragraph{Challenger}
\label{turniere:beispiele:challenger}
