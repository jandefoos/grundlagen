\chapter{Turniere und Meisterschaften}
\label{turniere}

Turniere und Meisterschaften sind die sportlichen Wettkämpfe beim Tischfussball. 
Neben den Teilnehmern, die sich sportlich messen, ist die Organisation vor, während und nach einem Turnier oder einer Meisterschaft für einen gelungenen Wettkampf entscheidend.
In diesem Kapitel wird erklärt wie man einen Wettkampf vorbereitet (Kapitel \ref{turniere:vorbereitung}), ihn durchführt (Kapitel \ref{turniere:durchfuehrung}) und danach die Ergebnisse darstellt (Kapitel \ref{turniere:ergebnisse}).

%Der Fokus liegt hierbei auf einen Standort, wie ein Jugendhaus, eine Schule oder einen Betrieb, wo typischerweise 1 Tisch steht.
%An entsprechender Stelle wird auch auf größere Turniere mit mehreren Tischen und vielen Spielern, die über mehrer Tage gehen können, eingegangen. 
%Tabelle \ref{tab:turniere} zeigt eine Übersicht, was es für typische Turnierformate im Tischfßball gibt.

%\begin{table}
%\centering
%\begin{tabular}{p{1.3cm}|p{1.3cm}|p{1.3cm}|p{2cm}|p{2cm}|p{2cm}} 
%Tische 	& Teams & Dauer & Disziplinen & Durchführung & Reichweite \\ 
%\hline 
%\hline 
%1-2 	& 3-12 & 1-2 h & 1 & wöchentlich-monatlich & lokal \\ 
%\hline 
%bis 10 	& bis 50 & ca. 4 h & 1 & wöchentlich-monatlich & lokal \\ 
%\hline 
%bis 20 	& bis 100 & 6-10 h & 1 & monatlich & regional \\ 
%\hline 
%bis 20 	& bis 200 & 2 Tage & 6-8 & monatlich & regional \\ 
%\hline 
%bis 20 	& bis 100 & ca. 6-10 h & 1 & monatlich & Rangliste \\ 
%\hline 
%\end{tabular} 
%\caption{Caption}
%\label{tab:turniere}
%\end{table} 
 
%%%%%%%%%%%%%%%%%%%%%%%%%%%%%%%%%%%%%%%%
\section{Vorbereitung}
\label{turniere:vorbereitung}

Vor einem Turnier sollte sich der Organisator überlegen
\begin{itemize}
\item wie viele {\bf Teams} erwartet werden, 
\item wie viele {\bf Tische} zu Verfügung stehen und
\item wie viel man Zeit zur Durchführung haben wird ({\bf Dauer}).
\end{itemize}
Nach diesen Faktoren kann man einen \nameref{turniere:vorbereitung:modus} festlegen und die \nameref{turniere:vorbereitung:spielrunden} festlegen, damit die Teilnehmer ausreichend viel spielen können. 
Spätestens ab einer gewissen Turniergröße lohnt es sich, die Turnierplanung in einer sogenannten \nameref{turniere:vorbereitung:ausschreibung} festzuhalten, die die wichtigsten Informationen über den Wettkampf übersichtlich erhalten sollte.

%%%%%%%%%%%%%%%%%%%%%%%%%%%%%%%%%%%%%%%%
%%%%%%%%%%%%%%%%%%%%%%%%%%%%%%%%%%%%%%%%
\subsection{Spielmodus}
\label{turniere:vorbereitung:modus}

Je nach Teilnehmeranzahl, Tischkapazität und
\nameref{turniere:ergebnisse:rahmen}
gibt es verschiedene Wettkampfsformate.

\nameref{spielformen:zaehlweisen:einsatz}
\nameref{spielformen:zaehlweisen:gewinnsaetze}

\nameref{spielformen:npersonen:einzel}
\nameref{spielformen:npersonen:doppel} 
\nameref{spielformen:npersonen:team} 

%%%%%%%%%%%%%%%%%%%%%%%%%%%%%%%%%%%%%%%%
\subsubsection{Fordern}
\label{turniere:vorbereitung:modus:fordern}

\begin{itemize}
\item Spieleranzahl: 3-5 Teams pro Tisch
\item Dauer: ab einer halben Stunde
\item \nameref{spielformen:npersonen}: typischerweise \nameref{spielformen:npersonen:doppel} 
\item \nameref{spielformen:zaehlweisen}: typischerweise \nameref{spielformen:zaehlweisen:einsatz}
\item Modus: 
\item \nameref{turniere:ergebnisse:rahmen}:
typischerweise \nameref{turniere:ergebnisse:rahmen:training} 
\item Hintergrund:
\end{itemize}


%%%%%%%%%%%%%%%%%%%%%%%%%%%%%%%%%%%%%%%%
\subsubsection{DYP}
\label{turniere:vorbereitung:modus:dyp}

wechselnder Partner

%%%%%%%%%%%%%%%%%%%%%%%%%%%%%%%%%%%%%%%%
\subsubsection{Challenger}
\label{turniere:vorbereitung:modus:challenger}

Vorrunde und KO-Runden

%%%%%%%%%%%%%%%%%%%%%%%%%%%%%%%%%%%%%%%%
%\subsubsection{Meisterschaft}
%\label{turniere:vorbereitung:modus:challenger}

%%%%%%%%%%%%%%%%%%%%%%%%%%%%%%%%%%%%%%%%
\subsubsection{Liga}
\label{turniere:vorbereitung:modus:liga}

Mehrere Spieltage

%%%%%%%%%%%%%%%%%%%%%%%%%%%%%%%%%%%%%%%%
%%%%%%%%%%%%%%%%%%%%%%%%%%%%%%%%%%%%%%%%
\subsection{Spielrunden}
\label{turniere:vorbereitung:spielrunden}

\subsubsection{Vorrunde}
\label{turniere:vorbereitung:spielrunden:vorrunde}

Jeder gegen jeden
Feste Rundenzahl mit Zufall
Schweizer System

\subsubsection{KO-Runde}
\label{turniere:vorbereitung:spielrunden:ko}

Felder
Doppel-KO

\subsubsection{Hin- und Rückrunde}
\label{turniere:vorbereitung:spielrunden:hinrueck}

Jeder gegen jeden
Heim und Auswärts


%%%%%%%%%%%%%%%%%%%%%%%%%%%%%%%%%%%%%%%%
\subsection{Ausschreibung}
\label{turniere:vorbereitung:ausschreibung}

%%%%%%%%%%%%%%%%%%%%%%%%%%%%%%%%%%%%%%%%
%%%%%%%%%%%%%%%%%%%%%%%%%%%%%%%%%%%%%%%%
\section{Durchführung}
Software/App? (Kickertool.de) / Papier
\label{turniere:durchfuehrung}


%%%%%%%%%%%%%%%%%%%%%%%%%%%%%%%%%%%%%%%%
%%%%%%%%%%%%%%%%%%%%%%%%%%%%%%%%%%%%%%%%
\section{Ergebnisse}
\label{turniere:ergebnisse}

%%%%%%%%%%%%%%%%%%%%%%%%%%%%%%%%%%%%%%%%
%%%%%%%%%%%%%%%%%%%%%%%%%%%%%%%%%%%%%%%%
\subsection{Rahmen}
\label{turniere:ergebnisse:rahmen}

\subsection{Training}
\label{turniere:ergebnisse:rahmen:training}

\subsection{Ranglisten}
\label{turniere:ergebnisse:rahmen:rangliste}
Forderpyramide

\subsection{Meisterschaft}
\label{turniere:ergebnisse:rahmen:meisterschaft}

Turnier einmal im Jahr: Jugendhausmeister, Schulmeister, Vereinmeister, Landesmeister, Deutsche Meister, Weltmeister

Über eine Saison, z.B. Liga


%%%%%%%%%%%%%%%%%%%%%%%%%%%%%%%%%%%%%%%%
\subsection{Darstellung}
\label{turniere:ergebnisse:formate}

Ehrentafel/Gewinnerfotos
Einzelergebnisse
Aktualisierte Tabelle oder Rangliste  

%%%%%%%%%%%%%%%%%%%%%%%%%%%%%%%%%%%%%%%%
\subsection{Preise}
\label{turniere:ergebnisse:preise}

Wanderpokale
Medaillen und Pokale
Urkunden
T-Shirts





%%%%%%%%%%%%%%%%%%%%%%%%%%%%%%%%%%%%%%%%
%\section{Modus für ein Jugendstandort}
