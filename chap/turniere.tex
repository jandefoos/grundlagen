\chapter{Turniere und Meisterschaften}
\label{turniere}

Turniere und Meisterschaften sind die sportlichen Wettkämpfe beim Tischfussball. 
Neben den Teilnehmern, die sich sportlich messen, ist die Organisation vor, während und nach einem Turnier oder einer Meisterschaft für einen gelungenen Wettkampf entscheidend.
In diesem Kapitel wird erklärt wie man einen Wettkampf vorbereitet (Kapitel \ref{turniere:vorbereitung}), ihn durchführt (Kapitel \ref{turniere:durchfuehrung}) und danach die Ergebnisse darstellt (Kapitel \ref{turniere:ergebnisse}).



%Der Fokus liegt hierbei auf einen Standort, wie ein Jugendhaus, eine Schule oder einen Betrieb, wo typischerweise 1 Tisch steht.
%An entsprechender Stelle wird auch auf größere Turniere mit mehreren Tischen und vielen Spielern, die über mehrer Tage gehen können, eingegangen. 
%Tabelle \ref{tab:turniere} zeigt eine Übersicht, was es für typische Turnierformate im Tischfußball gibt.

%\begin{table}
%\centering
%\begin{tabular}{p{1.3cm}|p{1.3cm}|p{1.3cm}|p{2cm}|p{2cm}|p{2cm}} 
%Tische 	& Teams & Dauer & Disziplinen & Durchführung & Reichweite \\ 
%\hline 
%\hline 
%1-2 	& 3-12 & 1-2 h & 1 & wöchentlich-monatlich & lokal \\ 
%\hline 
%bis 10 	& bis 50 & ca. 4 h & 1 & wöchentlich-monatlich & lokal \\ 
%\hline 
%bis 20 	& bis 100 & 6-10 h & 1 & monatlich & regional \\ 
%\hline 
%bis 20 	& bis 200 & 2 Tage & 6-8 & monatlich & regional \\ 
%\hline 
%\end{tabular} 
%\caption{Caption}
%\label{tab:turniere}
%\end{table} 
 
%%%%%%%%%%%%%%%%%%%%%%%%%%%%%%%%%%%%%%%%
\section{Vorbereitung}
\label{turniere:vorbereitung}

Vor einem Turnier sollte sich der Organisator überlegen
\begin{itemize}
\item was für eine Wertigkeit hat das Turnier: Training, Rangliste, Meisterschaft?
\item wie viele {\normalfont \bfseries Teams} erwartet werden, 
\item wie viele {\normalfont \bfseries Tische} zur Verfügung stehen und
\item wie viel Zeit man zur Durchführung haben wird ({\normalfont \bfseries Dauer}).
\end{itemize}
Nach diesen Faktoren legt man einen \nameref{turniere:vorbereitung:modus} fest. 
Ab einer gewissen Turniergröße lohnt es sich, die Turnierplanung in einer sogenannten \nameref{turniere:vorbereitung:ausschreibung} festzuhalten, die die wichtigsten Informationen über den Wettkampf übersichtlich beinhalten sollte.

%%%%%%%%%%%%%%%%%%%%%%%%%%%%%%%%%%%%%%%%
%%%%%%%%%%%%%%%%%%%%%%%%%%%%%%%%%%%%%%%%
\subsection{Spielmodus}
\label{turniere:vorbereitung:modus}

Je nach Teilnehmeranzahl, Tischkapazität, zeitlichen Rahmen und
\nameref{turniere:ergebnisse:rahmen}
gibt es verschiedene Wettkampfsformate.


%%%%%%%%%%%%%%%%%%%%%%%%%%%%%%%%%%%%%%%%
\subsubsection{Gewinner bleibt}
\label{turniere:vorbereitung:modus:fordern}

\begin{itemize}
\item Spieleranzahl: 3-5 Teams pro Tisch
\item Dauer: ab einer halben Stunde
\item \nameref{spielformen:npersonen}: typischerweise \nameref{spielformen:npersonen:doppel}, insbesondere als \nameref{turniere:vorbereitung:modus:dyp}-Variante
\item \nameref{spielformen:zaehlweisen}: typischerweise \nameref{spielformen:zaehlweisen:einsatz}
\item Modus: Es gibt eine Startreihenfolge und die ersten beiden Teams spielen gegeneinander. Der Gewinner einer Partie bleibt am Tisch stehen und das nächste Team in der Reihe ist der nächste Gegner. Das ausscheidende Team stellt sich hinten an. 
\item Gewinner: Das Team, welches hintereinander eine vorher bestimmte Anzahl an Spielen gewonnen hat, gewinnt das Turnier.  
\item \nameref{turniere:ergebnisse:rahmen}:
Typischerweise im \nameref{turniere:ergebnisse:rahmen:training}. Zusätzlich kann eine \nameref{turniere:ergebnisse:rahmen:rangliste} angelegt werden, für die die Gewinner 1 Punkt erhalten. 
\item Hintergrund: Als ,,Fordern'' ist dieses Spiel auch im freien Spiel bekannt: Spielen zwei Teams ein Spiel, gilt das ungeschriebene Gesetz ,,es darf gefordert werden''. Durch das Klopfen mit der Hand auf den Tischrand, fordert ein wartendes Team den Gewinner der Partie. 
\end{itemize}


%%%%%%%%%%%%%%%%%%%%%%%%%%%%%%%%%%%%%%%%
\subsubsection{DYP}
\label{turniere:vorbereitung:modus:dyp}

\begin{itemize}
\item Spieleranzahl: 3-5 Teams pro Tisch
\item Dauer: ab einer halben Stunde
\item \nameref{spielformen:npersonen}: \nameref{spielformen:npersonen:doppel} mit wechselnden Partner
\item \nameref{spielformen:zaehlweisen}: typischerweise \nameref{spielformen:zaehlweisen:einsatz}
\item Modus: 
Wenn man \nameref{turniere:vorbereitung:modus:fordern} oder eine Vorrunde spielt wird in jeder Runde ein neuer Spielpartner zugelost, das heißt in jeder Runde gibt es neue Doppelkombination. \\
Bei der \nameref{turniere:vorbereitung:modus:fordern} wird eine Liste geführt und die Spieler des Verlieredoppels losen, wer zuerst eingetragen wird und mit dem nächsten Losgewinner spielt. \\
Bei einer \nameref{turniere:vorbereitung:spielrunden:vorrunde} sammelt jeder Spieler für sich Punkte. Nach jeder Runde werden alle Spieler neu zusammengelost, z.B. mit Zetteln zum ziehen oder einer geeigneten Turnier-Software.
\item Gewinner: Das Team, welches hintereinander eine vorher bestimmte Anzahl an Spielen gewonnen hat, gewinnt das Turnier.  
\item \nameref{turniere:ergebnisse:rahmen}:
typischerweise als \nameref{turniere:ergebnisse:rahmen:training}s-Turnier
\item Trainingseffekt: Durch die wechselnden Partner lernt man mit verschiedenen Spieler gut zusammen zu doppeln. D.h. sich abzusprechen, in welcher Konstellation man spielt oder wie man verteidigt.
\item Hintergrund: DYP ist die Abkürzung für ,,Draw Your Partner'', was übersetzt so viel heißt wie ,,zieh deinen Partner''. 
\end{itemize}

%%%%%%%%%%%%%%%%%%%%%%%%%%%%%%%%%%%%%%%%
\subsubsection{Einzel/Doppel-Turnier}
\label{turniere:vorbereitung:modus:turnier}

\begin{itemize}
\item Spieleranzahl: 3-5 Teams pro Tisch
\item Dauer: ab einer halben Stunde bis 6 Stunden
\item \nameref{spielformen:npersonen}: \nameref{spielformen:npersonen:doppel} oder \nameref{spielformen:npersonen:einzel} 
\item Modus: Ein klassisches Turnier besteht aus einer Vorrunde und einer anschliessenden KO-Runde. Man kann aber auch nur eine Vorrunde oder nur eine KO-Runde spielen. \\
In einer \nameref{turniere:vorbereitung:spielrunden:vorrunde} hat jedes Team gleich viele Spiele und die Ergebnisse fließen in eine Tabelle ein. 
Der Tabellenführer nach einer vorher abgemachten Rundenanzahl gewinnt. \\
Für eine \nameref{turniere:vorbereitung:spielrunden:ko} braucht man einen angepassten KO-Baum. Die Gewinner kommen eine Runde weiter, die Verlierer scheiden aus. 
\item \nameref{turniere:ergebnisse:rahmen}:
\nameref{turniere:ergebnisse:rahmen:training}s-, \nameref{turniere:ergebnisse:rahmen:rangliste}n oder \nameref{turniere:ergebnisse:rahmen:meisterschaft}s-Turnier
\end{itemize}

%%%%%%%%%%%%%%%%%%%%%%%%%%%%%%%%%%
\paragraph{Vorrunde:}
\label{turniere:vorbereitung:spielrunden:vorrunde}

Während einer Vorrunde spielt man typischerweise einen Satz (siehe \nameref{spielformen:zaehlweisen:einsatz}) oder \nameref{spielformen:zaehlweisen:gewinnsaetze}. 
Es gibt verschiedene Varianten, wie viele Runden und nach welchem ster man diese Runden spielt:
\begin{itemize}
\item Jeder gegen jeden: 
Jedes Team spiel gegen jedes Team einmal. 
Zum Beispiel bei 4 Teams sind das insgesamt schon 3+2+1 = 6 Spiele. 
Dieser Modus bietet sich daher für kleinere Gruppen an.
\item Gruppen: 
Wie bei der Fußball-Weltmeisterschaft bildet man zum Beispiel 4er-Gruppen, d.h. 4 Teams pro Gruppe, die jeweils gegeneinander Spielen (Jeder gegen jeden). 
So können die Gruppen parallel ihre Vorrundenspiele durchführen.
\item Zufallsgeloste Runden:
Pro Vorrunde werden die Begegnungen gelost. Falls es einen ungerade Anzahl von Teams ist, gibt es ein Freilos, d.h. das Team, welches das Freilos zugelost bekommt, gewinnt das Spiel autoamtisch in dieser Runde. 
\item Schweizer System:
Bei dieser Varianten werden die Vorrunden ebenfalls gelost. Für die nächste Runde wird jedoch jeweils der aktuelle Tabellenstand mit einbezogen, so dass Teams, die in derselben Tabellenregion bzw. die gleiche Punktzahl haben, eher zusammengelost werden.
So wird sichergestellt, dass pro Runde jeweils etwa gleichstarke Teams gegeneinander spielen, und dass es vor allem nach mehreren Runden stets spannende Spiele gibt.  
\end{itemize}

Anhand der Vorrundenspiele berechnet man eine Tabelle. Dabei gelten wie bei anderen Mannschaftssportarten zur Berechnung der Platzierung
\begin{itemize}
\item die Punkte (3 oder 2 Punkte pro Sieg und 1 Punkt pro Unentschieden), 
\item bei Punktlgeichheit Tore (Differenz, dann die mehr geschossenen Tore),
\item und bei Torgleichheit der direkte Vergleich.
\end{itemize}
Ausnahme: Beim Schweizer System wird die Tabelle nach Punktgleichheit durch zwei sogenannter Buchholz-Faktoren berechnet, in die rückwirkend die erreichten Punkte der Gegner gehen. 


%%%%%%%%%%%%%%%%%%%%%%%%%%%%%%%%%%%%%%%%%%%%5
\paragraph{KO-Runde:}
\label{turniere:vorbereitung:spielrunden:ko}

,,KO`` steht für englisch ,,knock out`` und bedeutet, wer verliert ist raus, wer gewinnt kommt eine Runde weiter bis zum Sieg im Finale. 


\begin{itemize}
\item Einfach-KO: Anhand der Teilnehmeranzahl wird der Spielbaum bestimmt: Halbfinale bei bis zu 4 Teams, Viertelfinale bei 8, Achtelfinale bei 16, Sechszehntelfinale bei 32 usw. 
Wurde eine Vorrunde gespielt, wird die Teilnehmerzahl dementsprechend mit Freilosen aufgefüllt. Anhand der Vorrundentabelle werden die Begegnungen für die erste KO-Runde gesetzt.
Z.B. 20 Teams haben eine Vorrunde gespielt, daher wird ein 32er-Baum gespielt und im Sechszehntelfinale spielt der 1. gegen den 32., der 2. gegen den 31. usw.
Bei 20 Teams gibt es jedoch 12 Freilose, so dass die ersten 12 platzierten das Sechzehntelfinale automatisch gewinnen. So lauten die ersten Spiele: 13.-20., 14.-19., 15.-18., 16.-17.  
\item Felder: Hierbei werden mehrer Eindach-KOs gespielt, in dem man die Vorrundentabelle in verschiedene Felder aufteilt. Z.B. die ersten 8 Platzierten spielen eine KO-Runde und die Plätze 17 bis 32 auch. 
So spielen Teams mit ähnlichem Niveau jeweils eine Platzierungsrunde. 
\item Doppel-KO: Beim Doppel-KO gibt es neben dem eben beschriebenen Einfach-KO (Gewinnerrunde) einen zweiten KO-Baum (Verliererrunde). Jedes Team startet und spielt in der Gewinnerrunde. Falls ein Team verliert, kommt es in die parallel laufende Verlierrunde. Die Gewinner der Gewinnerrunde und der Verliererrunde spielen am Ende eine Finale.
Dieser Modus hat den Vorteil, das jedes Team sich einen Ausrutscher erlauben darf und trotzdem noch das Turnier gewinnen kann. Daher wird ein Doppel-KO oft auch ohne Vorrunde gespielt.
Der Nachteil kann die weniger gut planbare Länge des Turniers sein, da die Verliererunde doppelt so viele Spielrunden wie die Gewinnerrunde haben.   
\end{itemize}


%%%%%%%%%%%%%%%%%%%%%%%%%%%%%%%%%%%%%%%%
\subsubsection{Team-Liga}
\label{turniere:vorbereitung:modus:liga}

Der Liga-Modus wird hauptsächlich für \nameref{spielformen:npersonen:team}s angeboten und erstreckt sich meist über eine Saison, also ein Jahr. Der Modus entspricht etwa dem Modus der Fußball-Bundesliga: 
\begin{itemize}
\item Spieleranzahl: Ab 8 Teams mit Heimspielstätte
\item Dauer: über eine Saison
\item \nameref{spielformen:npersonen}: \nameref{spielformen:npersonen:team} 
\item Modus: 
\begin{itemize}
\item In einer Grupper oder Liga spielt jeder gegen jeden.
\item Es gibt eine Hin- und Rückrunde, also ein Heim- und ein Auswärtsspiel pro gegnerisches Team.
\item Die Spielergebnisse fließen in eine Tabelle (Punkte, Spiele, Tore). 
\end{itemize}
Wer nach allen Spielen an der Tabellenspitze steht, gewinnt. 
\item \nameref{turniere:ergebnisse:rahmen}:
\nameref{turniere:ergebnisse:rahmen:rangliste}
\end{itemize}



%%%%%%%%%%%%%%%%%%%%%%%%%%%%%%%%%%%%%%%%
\subsection{Ausschreibung}
\label{turniere:vorbereitung:ausschreibung}

Nach dem der Modus und der Termin für ein Wettkampf festgelegt wurde, sollte man die Infos zusammenfassen (Ausschreibung). Wichtige Informationen sind:
\begin{itemize}
\item Ausrichter: Name des Vereins, Verbands, Jugendhaus, ..
\item Termin: Datum und Uhrzeit des Turnierbeginns
\item Ort: Adresse
\item Ansprechpartner für weitere Infos und Fragen: Organisator
\item Disziplin: Einzel oder Doppel. Zusätzlich kann man Spezial-Turniere nur für Damen, Herren, Jugendliche oder Senioren anbieten. Wie beim Tennis, gibt es auch manchmal eine Mixed-Disziplin, in der ein Doppel-Team aus einer Dame und einem Herr sich zusammen setzt.   
\item Regeln: ITSF Regeln, eventuell angepasst.
\item Spielmodus: Rundenanzahl in der Vorrunde und KO-Rundenmodus.
\item Tische: Art und vor allem Anzahl der Tische. 
\item Voranmeldung: Sollten voraussichtlich viele Teams zu einem Turnier kommen, kann sich eine maximale Anzahl der Teams und eine Voranmeldung lohnen, um eine ausgewogene und ausreichende Spielzeit für alle Teams zu gewährleisten. 
Die maximale Anzahl der Teams hängt letztendlich mit der Tischkapazität, dem gewählten Spielmodus und dem verfügbaren Zeitrahmen ab.  
\item Preise: Etwa Urkunden oder Pokale für die ersten drei Plätze.
\end{itemize}
Diese Informationen kann man frühzeitig bekannt geben, damit für alle der Wettkampfrahmen klar ist.
Die Informationen kann man auch für ein Plakat verwenden, um das Turnier besser zu bewerben. 


%%%%%%%%%%%%%%%%%%%%%%%%%%%%%%%%%%%%%%%%
%%%%%%%%%%%%%%%%%%%%%%%%%%%%%%%%%%%%%%%%
\section{Durchführung}
\label{turniere:durchfuehrung}

Eine erfolgreiche Durchführung eines Turniers bedarf einer guten Planung und einer wachsamen \gls{turnierleitung}, die dafür sorgt, dass die Anmeldung, die Spielphase und die Siegerehrung reibungslos verläuft. 
Dafür braucht man vor allem einen guten Überblick, wer die Teams sind, welche Begegnungen anstehen und wie die Ergebnisse der Begegnungen sind. 

Bei wenigen Teams und einem einfachen Modus kann man als \gls{turnierleitung} die Begegnungen mit ,,Stift und Papier`` festhalten. Dafür gibt es im Internet auch einige Spielplan-Vorlagen, zum Beispiel bei \href{http://www.tischfussball-online.com/dies-das/turnierplaene.html}{tischfussball-online.de}. In der Regel wird eine Computer-Software dafür verwendet, um bei vielen Teilnehmern die Übersicht zu behalten und die Ergebnisse gleich online stellen zu können. Folgende Programme sind empfehlenswert:
\begin{itemize}
\item \href{http://kickertool.de/}{Kickertool}: Umfangfreiche, intuitive und kostenlose Online-Software mit allen gängigen Modi. Eine Offline-Version ist laut der Entwickler in Planung.
\item \href{http://www.heise.de/download/kickermaschine-1191093.html}{Kickermaschine:} Java-basierte, herunterladbare Software für Kicker-Turniere mit allen gängigen Modi.
\item TiFu: Die gängigste Software bei \gls{dtfb}-Turnieren (vor allem  Challengern) mit gängigen Modi, vorzeitigem Losen in der Vorrunde nach Schweizer System, die zudem eine Schnittstelle zum Sportsmanager anbietet. Kontaktperson: Christoph Hardt (\gls{dtfb}).
\item Sportsmanager: Joomla-Plugin zur Abwicklung von Teamwettbewerbe, Spielerverwaltung für  Vereine, Teams und Ranglisten und Ergebnisdarstellung, was vom \gls{dtfb} und vielen Landesverbände und Vereinen genutzt wird.  Kontaktperson: Sven Nickel (\gls{dtfb})
\item \href{http://www.table-soccer.org/page/fast}{FAST:} Vom ITSF entwickelt, für dessen Turniere. FAST steht kostenlos zur Verfügung, ist jedoch eine umfangreiche Software, da sie auch eine Spielerverwaltung für Turnierserien und Ranglisten mit sich bringt.
\end{itemize}

\subsection{Anmeldung}
\label{turniere:durchfuehrung:anmeldung}

Bevor das Turnier beginnt läuft die Anmeldung vor Ort. Bis kurz vor Turnierbeginn, sollten sich alle Teilnehmer bei der Turnieleitung anmelden. Die Turnierleitung sollte beachten, dass man nach Anmeldeschluss noch Zeit braucht, um anhand der Anmeldeliste die erste Spielrunde zu starten. Bei manchen Turnier-Software kann man sogar nach Turnierbeginn noch Teilnehmer hinzufügen, falls man das als Turnierleitung zu lassen will.

\subsection{Spielphase}
\label{turniere:durchfuehrung:turnierphase}

Um einen möglichst reibungslosen Verlauf zu gewähren sind zwei Dinge im Wesentlichen zu beobachten:
\begin{itemize}
\item Ergebnismeldung: Das Gewinnerteam sollte dafür verantwortlich sein, das Ergebnis bei der Turnierleitung zu melden. 
\item Spielaufruf: Nach der Ergebnismeldung kann die Turnierleitung den freien Tisch mit der nächsten Partie belegen. 
Verwendet man eine Turnier-Software werden die anstehenden Spiele autoamtisch so sortiert, dass das Turnier am schnellsten fortschreitet.
Für den Spieleraufruf lohnt sich bei größeren Gruppen eine Beschallungsanlage mit Mikrofon. 
In jedem Fall lohnt sich bei Verwendung einer Turnier-Software ein großer Monitor oder ein Beamer, um den Turnierverlauf für alle anzuzeigen.
\end{itemize}


\subsection{Siegerehrung}
\label{turniere:durchfuehrung:siegerehrung}

Die Turnierleitung sollte das Finale für alle ankündigen, damit sich ein paar Zuschauer unter den Spielern für das Topspiel finden. 
Nach dem Finale sollte direkt die Siegerehrung erfolgen, um einen würdigen Turnierabschluss zu haben.
Typische und mögliche Preise sind:
\begin{itemize}
\item Medaillen und Pokale: Bei vielen Turnieren gibt es Preise für Platz 1-3. Bei einem Doppelturnier sollte jeder Spieler eine Medaille oder Pokal bekommen, so dass man doppelt soviele Preise braucht.
\item Wanderpokal: Ein Wanderpokal bietet sich für eine Gruppe an, die ein regelmäßiges Turnier durchführen. Ein Wanderpokal hat zudem einen Lerneffekt, dass der Pokal zurückgegeben muss und so eine gewisse Verantwortung dem Gewinner aufgetragen wird.
\item Urkunden: Vor allem bei Turnieren mit Kindern und Jugendlichen sind Urkunden für alle Turnierteilnehmer eine gut Möglichkeit jeden zu ehren.
\item T-Shirts: Ein Shirt kann sich für eine Meisterschaft anbieten. Ein Shirt mit einem Aufdruck "Tischfußball-Meister 2016", was vom Gewinner dann auch in der Folge getragen wird, kann für alle Teilnehmer für das nächste Turnier motivierenden wirken, da es etwas Exklusives ist.
\end{itemize}



%%%%%%%%%%%%%%%%%%%%%%%%%%%%%%%%%%%%%%%%
%%%%%%%%%%%%%%%%%%%%%%%%%%%%%%%%%%%%%%%%
\section{Ergebnisse}
\label{turniere:ergebnisse}

Wie bei jedem anderen Sport ist die Veröffentlichung und die Darstellung der Ergebnisse sehr wichtig, damit die Leistung der Spieler honoriert wird und sich jeder in einer Tabelle, Rangliste, oder ähnlichem mit anderen vergleichen kann (Kapitel \ref{turniere:ergebnisse:formate}). 
Desweiteren können einzelne Spiel-Ergebnisse für einen bestimmten größeren Wettkampfrahmen dienen (Kapitel \ref{turniere:ergebnisse:rahmen}).

%%%%%%%%%%%%%%%%%%%%%%%%%%%%%%%%%%%%%%%%
\subsection{Darstellung}
\label{turniere:ergebnisse:formate}

Die Darstellung der Ergebnisse findet heutzutage ausserhalb der Spielstätte insbesondere im Internet statt. Folgende Veröffentlichungen werden häufig angewendet und halten den organisatorischen Aufwand gering:
\begin{itemize}
\item In der Spielspätte kann man eine Gewinnerwand installieren, an der Gewinnerfotos der regelmäßigen Turniere angebracht werden. So können sich die Gewinner lokal verewigen, und gleichzeitig informiert eine solche Ansicht auch über die Aktivitäten in der Spielstätte. 
\item Die einzelnen Ergebnisse einer jeden Partie eines Turniers (Kapitel \ref{turniere:vorbereitung:modus:turnier} oder einer Ligapartie (Kapitel \ref{turniere:vorbereitung:modus:liga} kann leicht mit der richtigen Turniersoftware (Kapitel \ref{turniere:durchfuehrung:turnierphase}) online auf einer Webseite darstellen.
\item So können ebenfalls Turnier-Ranglisten und Liga-Tabellen (Kapitel \ref{turniere:ergebnisse:rahmen:rangliste}) aktualisiert werden. Desweiteren lohnt es sich für die Historie einer "Hall of Fame" zu pflegen, die alle jährlichen Meister (Gewinner einer Meisterschaft, Kapitel \ref{turniere:ergebnisse:rahmen:meisterschaft}) fest hält.  
\end{itemize}

%%%%%%%%%%%%%%%%%%%%%%%%%%%%%%%%%%%%%%%%
\subsection{Wettkampfrahmen}
\label{turniere:ergebnisse:rahmen}

Mit den verschiedenen Spielmodi (Kapitel \ref{turniere:vorbereitung:modus}) können verschiedene Wettkampfrahmen angeboten werden. 

\subsubsection{Training}
\label{turniere:ergebnisse:rahmen:training}

Insbesondere die Spielmodi \nameref{turniere:vorbereitung:modus:fordern} und \nameref{turniere:vorbereitung:modus:dyp} bieten sich für ein abwechslungsreiches Training an. Zusammen mit den verschiedenen \nameref{spielformen} können so viele Trainingsinahlte erstellt werden, um unterschiedliche Aspekte zu schulen. 

Eine öffentliche Darstellung (Kapitel \ref{turniere:ergebnisse:formate}) ist nicht notwendig, eine Forderpyramide kann in der Spielstätte für Trainingsspiele gleichen Spielniveaus garantieren.
Hierbei handelt es sich um eine fortlaufende Rangliste, die schlangenförmig in einer Pyramideform angeordnet ist: Platz 1 bildet die Spitze, die Reihe drunter Platz 2 und 3, dann Platz 4-6 (bzw. Platz 4-8), Platz 7-10 (bzw. Platz 9-16) usw. 
Spieler können andere Spieler zu einem Spiel herausfordern, aber nur wenn der Spieler in der gleichen Reihe oder eine Reihe drüber in der Forderpyramide steht. Forderungen müssen angenommen werden.


\subsubsection{Rangliste oder Tabellen}
\label{turniere:ergebnisse:rahmen:rangliste}

Insbesondere der Spielmodus \nameref{turniere:vorbereitung:modus:turnier} bietet sich als Turnierserie über eine Saison an.
Zwei Turnierserien sind im regionalen und lokalen Spielbetrieb typisch: wöchentliche 2-4 Stunden Turniere und monatliche 4-8 Stunden Turniere. 
Durch einen definierten Punkteschlüssel bekommt jeder Teilnehmer Punkte entsprechend seiner Turnierergebnisse, die in eine \textbf{Rangliste} einfliessen. 
Zusätzlich gibt es oft eine Deckelung, dass nur die besten 20-30 \% Turnierergebnisse aller Turniere dieser Turnierserie in die Rangliste einfliessen, damit die Teilnehmer nicht jedes Turnier mitspielen müssen. 
Ein typisches Beispiel für diesen Wettkampfrahmen ist die Challenger-Turnierserie des \gls{dtfb} (siehe Kapitel \ref{turniere:beispiele:challenger}).

Der Spielmodus \ref{turniere:vorbereitung:modus:liga} bietet sich an über eine Jahres-Saison zu spielen und die Ergebnisse der Partien in eine \textbf{Tabelle} einfliessen zu lassen. Dabei gibt es für einen Sieg 2:0 Punkte, für ein Unentschieden 1:1 und bei einer Niederlage 0:2. Der Tabellenplatz ergibt sich durch  
\begin{itemize}
\item mehr Punkte,
\item bei Punktgleichheit durch mehr gewonnener Spiele und
\item bei Spielgleichheit durch mehr geschossene Tore.
\end{itemize}
Der Erstplatzierte ist in diesem Wettkampf Meister einer Liga. Ein Beispiel ist die Berliner Landesliga (Kapitel \ref{turniere:beispiele:landesliga}). 

\subsubsection{Meisterschaft}
\label{turniere:ergebnisse:rahmen:meisterschaft}

Bei individual Wettbewerben, wie Einzel- und Doppel-Turnierserien, gibt es neben den Erstplatzierten einer Rangliste am Ende einer Saison oft noch ein einzelnes Turnier, was als Meisterschaft gespielt wird. 
Der Gewinner dieses Turnier ist der entsprechende Meister, beispielsweise "Deutscher Meister 2016 im Junioren Doppel".

Bei Team-Wettbewerben gibt es neben der Meisterschaft über eine Saison mit Hin- und Rückrunde eine Pokalrunde, bei dem alle Mannschaften aus allen (regionalen) Ligen teilnehmen und in einer KO-Runde den Pokal-Gewinner ausspielen.

Ranglistengewiner im Einzel/Doppel, Einzel-, Doppel-, und Team- Meister gibt es bzw. kann es für folgende Gruppen in Zukunft geben: 
\begin{itemize}
\item Jugendgruppen: Jugendhaus-Meister, Hamburger Jugendhaus-Meister, Deutscher Jugendhaus-Meister
\item Schulgruppen: Schul-Meister, Hamburger Schul-Meister, Deutscher Schul-Meister
\item Vereine: Vereins-Ranglistengewinner und Vereins-Meister
\item Landesverband: Hamburger Landesranglisten-Gewinner und Hamburger Landesmeister
\item Nationaler Verband: Deutscher Nationalranglisten-Gewinner und Deutscher Meister
\item Kontinentaler Verband: Europäischer Ranglistengewinner und Europameister
\item Internationaler Verband: Weltramglistern-Gewinner und Weltmeister
\end{itemize}


%%%%%%%%%%%%%%%%%%%%%%%%%%%%%%%%%%%%%%%%
%%%%%%%%%%%%%%%%%%%%%%%%%%%%%%%%%%%%%%%%
\section{Beispiele}
\label{turniere:beispiele}

Zu den oben genannten Bestandteile eines Wettkampfs (\nameref{turniere}) werden hier Beispiele beschrieben.

%%%%%%%%%%%%%%%%%%%%%%%%%%%%%%%%%%%%%%%%%%%%5
\subsection{Monatliches Jugendhaus-Turnier}
\label{turniere:beispiele:jugend}

Dieser Wettkampfrahmen ist eine Rangliste (Kapitel \ref{turniere:ergebnisse:rahmen:rangliste}), die für monatliche Turniere geführt wird.

\begin{itemize}
\item \textbf{Vorbereitung:} Zu Anfang einer Wintersaison (Oktober bis März) werden 6 Termine benannt (z.B. der 2. Mittwoch im Monat), an denen ein Turnier statt findet. Die Saison sollte an eine regionale Jugendhausmeisterschaft angepasst werden, die am Ende der Saison als Abschluss steht.   
Für typischerweise einen Tisch bietet sich ein Doppelturnier an. Ein Jugendbetreuer ist ausreichend, um ein solches Turnier auszurichten. Er sollte sich vor allem überlegen, ob ein Turniersoftware helfen könnte und sich damit im Vorfeld vertraut macht.  
\item \textbf{Spielmodus:} In einem Zeitrahmen von 2 Stunden bietet sich bei einem Tisch und mehr als 4 Teams, eine lange Vorrunde zu spielen und am Ende ein Finale. Für viele Spiele bietet sich ein Satz auf 6 Tore an mit 5:5 Unentschieden an. Als Spielplan kann ein Jeder-Gegen-Jeden-Modus oder das Schweizer System gewählt werden. Nach Ende der Vorrunde spielt der Zweit- und der Erstplatzierte eine Finalpartie aus.  
\item \textbf{Ausschreibung:} Der Organisator sollte ein Plakat oder ein Flyer mit den Terminen das Doppelturnier im Tischfußball bewerben. 
\item \textbf{Durchführung:} An jedem Turniertag können sich die Doppelteams beim Organisator bis zu einer bestimmten Uhrzeit anmelden. Der Organisatoren kann entscheiden, ob später ankommende Teams noch nachträglich einsteigen können, was bei einer rundenweiser Losung in der Vorrunde (z.B. Schweizer System) problemlos möglich ist. 
Der Turnierorganisator sollte das Voranschreiten des Turniers beobachten, um etwa nach einer Stunde bekannt zu geben, wie viele Vorrunden noch gespielt werden und wann damit die Abschlusstabelle feststeht. 
\item \textbf{Siegerehrung und Preise:} Preise sollten Medaillen oder kleine Pokale für die Gewinner und Urkunden für alle Teilnehmer sein. Ein Gruppenphoto aller Teilnehmer mit ihren Preisen sollte für eine Gewinnertafel im Jugendhaus gemacht werden. Falls gewünscht kann das Bild auch bei Facebook gepostet werden.
\item \textbf{Ergebnisse:} Neben der Gewinnertafel können die Ergebnisse mit der entsprechenden Turniersoftware auf eine Webseite hochgeladen werden oder alternativ ausgedruckt werden.
\end{itemize}

Dieses Wettkampf-Beispiel kann genau so gut für eine Gruppe an einer Schule, in einer Ausbildungstätte, in einem Betrieb oder auch in einem Verein durchgeführt werden.

%%%%%%%%%%%%%%%%%%%%%%%%%%%%%%%%%%%%%%%%%%%%5
\subsection{DTFB Challenger}
\label{turniere:beispiele:challenger}

Dieser Wettkampfrahmen sind eine landesweite und bundesweite Ranglisten (Kapitel \ref{turniere:ergebnisse:rahmen:rangliste}), die für regionale, typischerweise 6-16 Turnieren pro Saison geführt wird.
Bei mehr als zwei Tischen kann ein Challenger bei  einem Verein, einem Tischfußballzentrum oder einem Jugendstandort  ausgerichtet werden. Die Anmeldung muss über eine Ausschreibung vom Landesverband an den \gls{dtfb} Ranglistenwart erfolgen. Die folgende Beschreibung enthält Auszüge aus der \href{http://dtfb.de/images/Ranglistenturnierordnung_des_DTFB_Version_2016_1.pdf} {Ranglistenordnung vom März 2016}. 

\begin{itemize}
\item \textbf{Vorbereitung:} 
Als Turnierleitung bieten sich zwei Personen im Vorfeld an und mehrere Personen während des Turniers.
Die Auschreibung muss mindestens 6 Wochen vor Turnierbeginn angemeldet sein, damit das Turnier als DTFB Challenger in die nationale Rangliste einfliesst. Mit der Ausschreibung kann man das Turnier in den Spielernetzwerken bewerben.   
\item \textbf{Spielmodus:}
Es gelten folgende zeitliche Rahmenbedingungen: 
\begin{itemize}
\item Turnierbeginn: Einzel 9.30 - Doppel 10.30 Uhr
\item Einlass: mindestens 1 Stunde vor Turnierbeginn
\item Anmeldeschluss: spätestens 15 Minuten vor Turnierbeginn 
\item Start der Hauptrunde: spätestens 18.00 Uhr
\end{itemize}
Der Challenger-Modus gestaltet sich wie folgt:
\begin{itemize}
\item Alle Teilnehmer spielen Qualifikationsrunden nach dem Schweizer System und Buchholzzahl mit anschließenden Playoffs als Single-KO um die Endplatzierungen aus.
\item Challenger: In den Qualifikationsrunden muss ein Unentschieden als Wertung möglich sein, die Satz- und Toranzahl ist frei wählbar (Ideal: zwei Gewinnsätze bis 5 Siegtore mit Verlängerung im Entscheidungssatz bis max. 8 Tore oder 7:7 unentschieden). 
\item Abhängig von der Anzahl der Teilnehmer werden die Endplatzierungen in maximal 3 unterschiedlichen Divisionen (Profi, Amateur und Neuling) ausgespielt. 
\item Jede Division darf maximal 5 Runden (max. 32 Qualifikanten) dauern. Die maximale Teilnehmerzahl für alle Divisionen zusammen ergibt sich wie folgt: 3,6 x Tischanzahl = maximale Teilnehmerzahl in den KO-Runden.
\item In der Profi-Division werden 3 Gewinnsätze (Best-of 5) bis 5 Siegtore und mit Verlängerung im Entscheidungssatz (bis maximal 8 Siegtore). In allen anderen Divisionen werden 2 Gewinnsätze (Best-of-3) gespielt. 
\item Die Divisionen sind hierarchisch in der Reihenfolge (höchste zuerst) Profi, Amatuer, Neuling angeordnet. In der untergeordneten Division müssen mindestens gleich viele oder mehr Aktive starten wie in der übergeordneten.
\item In den Playoffs werden die Spieler gemäß deren Platzierung aus der Vorrunde gesetzt (z.B. 1-8; 2-7; ...).
\item Es findet neben dem Finale in jedem Wettbewerb ein Spiel um Platz 3 statt.
\end{itemize}
\item \textbf{Ausschreibung:}

\begin{itemize}
\item Bezeichnung des Turniers
\item Namen des Veranstalters und Ausrichters inkl. Kontaktmöglichkeit für Voranmeldungen
\item Beginn des Turniers
\item Ort der Austragung (Name und Adresse)
\item Anzahl und der Typ der Spieltische
\item verwendete Figuren und Bälle 
\item Disziplin (Doppel und/oder Einzel)
\item Modus (Vorrunde Schweizer System mit Unentschieden; anschließend KO-Runde; ...)
\item maximale Anzahl der Teilnehmer
\item Tag und Zeit des Meldeschlusses
\item Höhe der Gebühren (Organisationspauschale) inkl. Verfahren, wie die Zahlung zu erfolgen hat
\item Preise (Pokale, Medaillen oder Urkunden)
\item Infos zum Catering (Essen und Trinken vorhanden? Mitbringen von Speisen erlaubt?)
\item eventuelle Vorbehalte zur Änderung der Ausschreibung
\item wenn Bälle nicht vom Ausrichter zur Verfügung gestellt werden, muss vermerkt werden, ob und zu welchem Preis der Spielball vor Ort erworben werden kann
\end{itemize}

\item \textbf{Durchführung:} 
Auch bei einer Voranmeldung im Vorfeld des Turniers müssen die Spieler ihre Anwesenheit bei der Turnieleitung anmelden und die Organisationspauschale zahlen, um am Turnier teilzunehmen. Herbei bietet es sich an zu zweit zu sein, um gemeinsam Namen und Pauschalen anzunehmen.
Bei großen Turnieren bietet sich an mit Beamerweinländen den Turnierverlauf anzuzeigen, sowie stets ein Turnierleiter am Ergebnis-PC zu haben, um Ergebnisse einzutragen und anstehende Spiele an einem frei gewordenen Tisch auszurufen.
\item \textbf{Siegerehrung:} Nach dem Finale sollten die Turnierleitung zusammen mit den Anwesenden den Gewinnern gratulieren bei der Überreichung der Pokal. Dazu sollten Siegerphotos gemacht werden, die auf den Webseiten und in sozialen Netzwerken veröffentlicht werden können. 
\item \textbf{Ergebnisse:} Die Ergebnisse werden auf der Landesverbands- und DTFB-Webseite veröffentlicht und fliessen in die jeweiligen Ranglisten ein.
\end{itemize}

Dieser Modus wird auch als Mini-Challenger an einem Abend innerhalb von 4 Stunden durchgeführt, mit 4 Vorrunden in 2 Stunden und 2 Stunden für die KO-Felder. Diese Turniere fließen in die entsprechende Landesrangliste ein und benötigen keiner Anmeldung beim \gls{dtfb}. 


%%%%%%%%%%%%%%%%%%%%%%%%%%%%%%%%%%%%%%%%%%%%5
\subsection{Berliner Landesliga}
\label{turniere:beispiele:landesliga}

Dieser Wettkampfrahmen ist eine Team-Meisterschaft (Kapitel \ref{turniere:vorbereitung:modus:liga}), die im Ligamodus mit Hin- und Rückrunde gespielt wird.

\begin{itemize}
\item \textbf{Vorbereitung:} Bis zu einem bestimmten Termin können sich die Teams bei der Ligaleitung und ihren Spielerstamm anmelden. Danach werden die Spielplne erstellt und veröffentlicht. Die Ligaleitung sollte insbesondere bei mehreren Ligen aus mehreren Personen bestehen. 
\item \textbf{Spielmodus:} In einer Hin- und Rückrunde spielen die Teams einer Gruppe bzw. Liga einen Jeder-Gegen-Jeden-Modus. Dabei findet die Partien einmal Heim- und einmal Auswärts statt.
In einem Team werden 6 Spieler und bis zu 2 Auswechselspieler aufgestellt, 2 Spieler stellen das Doppel 1 (D1), was gegen D2 und D3 des Gegners spielt, und jeweils zwei Einzel-Spiele gegen die Spieler des D1 des Gegners. Entprechend das Doppel 2. Und Doppel 3 spielt auch zwei Spiele gegen D1 und D2, dann aber noch zwei Spiele gegen das gegnerische D3.
Jedes Spiel geht bis maximal 10 Tore, also bei 6 Toren Sieg und 5:5 Unentschieden.
\item \textbf{Ausschreibung:} Werbung wird für die Liga auf den Verbands- und Vereins-Webseiten sowie in den Spielstätten gemacht.
\item \textbf{Durchführung:} Die Partien werden von den Teams selbständig durchgeführt. Vor Beginn der Partie gleichen die Mannschaftsführer ihre Aufstellungen ab. Die Ergebnisse werden in einem Spielbogen eingetragen und das Endergebnis wird vom Heimteam gemeldet, damit des Ergebnis in die Tabelle fliesst. 
\item \textbf{Siegerehrung:} Für Ligen gibt es meist einen Wanderpokal für den Meister und oft auch Wimpel für Teams und Spieler als Ehrungen  am Saisonende.
\item \textbf{Ergebnisse:} Die Ergebnisse werden auf der Landesverbands-Webseite veröffentlicht und fliessen in die jeweiligen Liga-Tabelle ein.
\end{itemize}
Weitere detaillierte Details findet man auf den Webseiten des \href{http://www.tfvb.de/attachments/offiziell/20160123_Spielordnung.pdf}{Tischfußball-Verbands Berlin (TFVB)}.