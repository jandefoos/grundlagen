\chapter{Turniere und Meisterschaften}
\label{turniere}

Turniere und Meisterschaften sind die sportlichen Wettkämpfe beim Tischfussball. 
Neben den Teilnehmern, die sich sportlich messen, ist die Organisation vor, während und nach einem Turnier oder einer Meisterschaft für einen gelungenen Wettkampf entscheidend.
In diesem Kapitel wird erklärt wie man einen Wettkampf vorbereitet (Kapitel \ref{turniere:vorbereitung}), ihn durchführt (Kapitel \ref{turniere:durchfuehrung}) und danach die Ergebnisse darstellt (Kapitel \ref{turniere:ergebnisse}).

%Der Fokus liegt hierbei auf einen Standort, wie ein Jugendhaus, eine Schule oder einen Betrieb, wo typischerweise 1 Tisch steht.
%An entsprechender Stelle wird auch auf größere Turniere mit mehreren Tischen und vielen Spielern, die über mehrer Tage gehen können, eingegangen. 
%Tabelle \ref{tab:turniere} zeigt eine Übersicht, was es für typische Turnierformate im Tischfßball gibt.

%\begin{table}
%\centering
%\begin{tabular}{p{1.3cm}|p{1.3cm}|p{1.3cm}|p{2cm}|p{2cm}|p{2cm}} 
%Tische 	& Teams & Dauer & Disziplinen & Durchführung & Reichweite \\ 
%\hline 
%\hline 
%1-2 	& 3-12 & 1-2 h & 1 & wöchentlich-monatlich & lokal \\ 
%\hline 
%bis 10 	& bis 50 & ca. 4 h & 1 & wöchentlich-monatlich & lokal \\ 
%\hline 
%bis 20 	& bis 100 & 6-10 h & 1 & monatlich & regional \\ 
%\hline 
%bis 20 	& bis 200 & 2 Tage & 6-8 & monatlich & regional \\ 
%\hline 
%bis 20 	& bis 100 & ca. 6-10 h & 1 & monatlich & Rangliste \\ 
%\hline 
%\end{tabular} 
%\caption{Caption}
%\label{tab:turniere}
%\end{table} 
 
%%%%%%%%%%%%%%%%%%%%%%%%%%%%%%%%%%%%%%%%
\section{Vorbereitung}
\label{turniere:vorbereitung}

Vor einem Turnier sollte sich der Organisator überlegen
\begin{itemize}
\item was für eine Wertigkeit hat das Turnier: Training, Rangliste, Meisterschaft?
\item wie viele {\bf Teams} erwartet werden, 
\item wie viele {\bf Tische} zur Verfügung stehen und
\item wie viel Zeit man zur Durchführung haben wird ({\bf Dauer}).
\end{itemize}
Nach diesen Faktoren legt man einen \nameref{turniere:vorbereitung:modus} fest. 
Ab einer gewissen Turniergröße lohnt es sich, die Turnierplanung in einer sogenannten \nameref{turniere:vorbereitung:ausschreibung} festzuhalten, die die wichtigsten Informationen über den Wettkampf übersichtlich beinhalten sollte.

%%%%%%%%%%%%%%%%%%%%%%%%%%%%%%%%%%%%%%%%
%%%%%%%%%%%%%%%%%%%%%%%%%%%%%%%%%%%%%%%%
\subsection{Spielmodus}
\label{turniere:vorbereitung:modus}

Je nach Teilnehmeranzahl, Tischkapazität, zeitlichen Rahmen und
\nameref{turniere:ergebnisse:rahmen}
gibt es verschiedene Wettkampfsformate.


%%%%%%%%%%%%%%%%%%%%%%%%%%%%%%%%%%%%%%%%
\subsubsection{Gewinner bleibt}
\label{turniere:vorbereitung:modus:fordern}

\begin{itemize}
\item Spieleranzahl: 3-5 Teams pro Tisch
\item Dauer: ab einer halben Stunde
\item \nameref{spielformen:npersonen}: typischerweise \nameref{spielformen:npersonen:doppel}, insbesondere als \nameref{turniere:vorbereitung:modus:dyp}-Variante
\item \nameref{spielformen:zaehlweisen}: typischerweise \nameref{spielformen:zaehlweisen:einsatz}
\item Modus: Es gibt eine Startreihenfolge und die ersten beiden Teams spielen gegeneinander. Der Gewinner einer Partie bleibt am Tisch stehen und das nächste Team in der Reihe ist der nächste Gegner. Das ausscheidende Team stellt sich hinten an. 
\item Gewinner: Das Team, welches hintereinander eine vorher bestimmte Anzahl an Spielen gewonnen hat, gewinnt das Turnier.  
\item \nameref{turniere:ergebnisse:rahmen}:
Typischerweise im \nameref{turniere:ergebnisse:rahmen:training}. Zusätzlich kann eine \nameref{turniere:ergebnisse:rahmen:rangliste} angelegt werden, für die die Gewinner 1 Punkt erhalten. 
\item Hintergrund: Als ,,Fordern'' ist dieses Spiel auch im freien Spiel bekannt: Spielen zwei Teams ein Spiel, gilt das ungeschriebene Gesetz ,,es darf gefordert werden''. Durch das Klopfen mit der Hand auf den Tischrand, fordert ein wartendes Team den Gewinner der Partie. 
\end{itemize}


%%%%%%%%%%%%%%%%%%%%%%%%%%%%%%%%%%%%%%%%
\subsubsection{DYP}
\label{turniere:vorbereitung:modus:dyp}

\begin{itemize}
\item Spieleranzahl: 3-5 Teams pro Tisch
\item Dauer: ab einer halben Stunde
\item \nameref{spielformen:npersonen}: \nameref{spielformen:npersonen:doppel} mit wechselnden Partner
\item \nameref{spielformen:zaehlweisen}: typischerweise \nameref{spielformen:zaehlweisen:einsatz}
\item Modus: 
Wenn man \nameref{turniere:vorbereitung:modus:fordern} oder eine Vorrunde spielt wird in jeder Runde ein neuer Spielpartner zugelost, das heißt in jeder Runde gibt es neue Doppelkombination. \\
Bei der \nameref{turniere:vorbereitung:modus:fordern} wird eine Liste geführt und die Spieler des Verlieredoppels losen, wer zuerst eingetragen wird und mit dem nächsten Losgewinner spielt. \\
Bei einer \nameref{turniere:vorbereitung:spielrunden:vorrunde} sammelt jeder Spieler für sich Punkte. Nach jeder Runde werden alle Spieler neu zusammengelost, z.B. mit Zetteln zum ziehen oder einer geeigneten Turnier-Software.
\item Gewinner: Das Team, welches hintereinander eine vorher bestimmte Anzahl an Spielen gewonnen hat, gewinnt das Turnier.  
\item \nameref{turniere:ergebnisse:rahmen}:
typischerweise als \nameref{turniere:ergebnisse:rahmen:training}s-Turnier
\item Trainingseffekt: Durch die wechselnden Partner lernt man mit verschiedenen Spieler gut zusammen zu doppeln. D.h. sich abzusprechen, in welcher Konstellation man spielt oder wie man verteidigt.
\item Hintergrund: DYP ist die Abkürzung für ,,Draw Your Partner'', was übersetzt so viel heißt wie ,,zieh deinen Partner''. 
\end{itemize}

%%%%%%%%%%%%%%%%%%%%%%%%%%%%%%%%%%%%%%%%
\subsubsection{Einzel/Doppel-Turnier}
\label{turniere:vorbereitung:modus:turnier}

\begin{itemize}
\item Spieleranzahl: 3-5 Teams pro Tisch
\item Dauer: ab einer halben Stunde bis 6 Stunden
\item \nameref{spielformen:npersonen}: \nameref{spielformen:npersonen:doppel} oder \nameref{spielformen:npersonen:einzel} 
\item Modus: Ein klassisches Turnier besteht aus einer Vorrunde und einer anschliessenden KO-Runde. Man kann aber auch nur eine Vorrunde oder nur eine KO-Runde spielen. \\
In einer \nameref{turniere:vorbereitung:spielrunden:vorrunde} hat jedes Team gleich viele Spiele und die Ergebnisse fließen in eine Tabelle ein. 
Der Tabellenführer nach einer vorher abgemachten Rundenanzahl gewinnt. \\
Für eine \nameref{turniere:vorbereitung:spielrunden:ko} braucht man einen angepassten KO-Baum. Die Gewinner kommen eine Runde weiter, die Verliere scheiden aus. 
\item \nameref{turniere:ergebnisse:rahmen}:
\nameref{turniere:ergebnisse:rahmen:training}s-, \nameref{turniere:ergebnisse:rahmen:rangliste}n oder \nameref{turniere:ergebnisse:rahmen:meisterschaft}s-Turnier
\end{itemize}

%%%%%%%%%%%%%%%%%%%%%%%%%%%%%%%%%%
\paragraph{Vorrunde:}
\label{turniere:vorbereitung:spielrunden:vorrunde}

Während einer Vorrunde spielt man typischerweise \nameref{spielformen:zaehlweisen:einsatz}. 
Es gibt verschiedene Varianten, wie viele Runden und nach welchem Muster man diese Runden spielt:
\begin{itemize}
\item Jeder gegen jeden: 
Jedes Team spiel gegen jedes Team einmal. 
Zum Beispiel bei 4 Teams sind das 3+2+1 = 6 Spiele. 
Dieser Modus bietet sich für kleinere Gruppe an.
\item Gruppen: 
\item Zufallsgeloste Runden:
\item Schweizer System:
\end{itemize}
Tabelle 


%%%%%%%%%%%%%%%%%%%%%%%%%%%%%%%%%%%%%%%%%%%%5
\paragraph{KO-Runde}
\label{turniere:vorbereitung:spielrunden:ko}

\begin{itemize}
\item Einfach-KO
\item Felder
\item Doppel-KO
\end{itemize}

%%%%%%%%%%%%%%%%%%%%%%%%%%%%%%%%%%%%%%%%%%%%5
\paragraph{z.B. Challenger}
\label{turniere:vorbereitung:spielrunden:challenger}

%%%%%%%%%%%%%%%%%%%%%%%%%%%%%%%%%%%%%%%%
\subsubsection{Team-Liga}
\label{turniere:vorbereitung:modus:liga}

\begin{itemize}
\item Jeder gegen jeden
\item Heim und Auswärts
\item Tabelle
\end{itemize}

%%%%%%%%%%%%%%%%%%%%%%%%%%%%%%%%%%%%%%%%
\subsection{Ausschreibung}
\label{turniere:vorbereitung:ausschreibung}

%%%%%%%%%%%%%%%%%%%%%%%%%%%%%%%%%%%%%%%%
%%%%%%%%%%%%%%%%%%%%%%%%%%%%%%%%%%%%%%%%
\section{Durchführung}
\label{turniere:durchfuehrung}

\begin{itemize}
\item Papier, Vorlage
\item Software: Kickertool.de, Kickermaschine, TiFu, Sportsoftware
\end{itemize}

\subsection{Anmeldung}
\label{turniere:durchfuehrung:anmeldung}

\subsection{Turnierphase}
\label{turniere:durchfuehrung:turnierphase}

\subsection{Siegerehrung}
\label{turniere:durchfuehrung:siegerehrung}


%%%%%%%%%%%%%%%%%%%%%%%%%%%%%%%%%%%%%%%%
%%%%%%%%%%%%%%%%%%%%%%%%%%%%%%%%%%%%%%%%
\section{Ergebnisse}
\label{turniere:ergebnisse}

%%%%%%%%%%%%%%%%%%%%%%%%%%%%%%%%%%%%%%%%
%%%%%%%%%%%%%%%%%%%%%%%%%%%%%%%%%%%%%%%%
\subsection{Wettkampfrahmen}
\label{turniere:ergebnisse:rahmen}

\subsubsection{Training}
\label{turniere:ergebnisse:rahmen:training}

\subsubsection{Rangliste}
\label{turniere:ergebnisse:rahmen:rangliste}

\begin{itemize}
\item Punkteschlüssel, maximale Turnierzahl, Turniersaison
\item Forderpyramide
\end{itemize}

\subsubsection{Meisterschaft}
\label{turniere:ergebnisse:rahmen:meisterschaft}

\begin{itemize}
\item Einzel/Doppel-Turnier einmal im Jahr: Jugendhausmeister, Schulmeister, Vereinmeister, Landesmeister, Deutsche Meister, Weltmeister
\item Team: Über eine Saison, z.B. Liga oder Pokal
\end{itemize}


%%%%%%%%%%%%%%%%%%%%%%%%%%%%%%%%%%%%%%%%
\subsection{Darstellung}
\label{turniere:ergebnisse:formate}

\begin{itemize}
\item Gewinnerfotos, Ehrentafel
\item Einzelergebnisse
\item Aktualisierte Tabelle oder Rangliste  
\end{itemize}

%%%%%%%%%%%%%%%%%%%%%%%%%%%%%%%%%%%%%%%%
\subsection{Preise}
\label{turniere:ergebnisse:preise}

\begin{itemize}
\item Wanderpokale
\item Medaillen und Pokale
\item Urkunden
\item T-Shirts
\end{itemize}

%%%%%%%%%%%%%%%%%%%%%%%%%%%%%%%%%%%%%%%%
%\section{Modus für ein Jugendstandort}
