\chapter{Spielgerät und Zubehör}
\label{tisch}

Materialkunde (Begrifflichkeiten):



%%%%%%%%%%%%%%%%%%%%%%%%%%%%%%%%%%%%%%%%%%%%%%
\section{Der Tisch}
\label{tisch:tisch}

Der Tisch zusammen mit einem Ball ist das Spielgerät beim Tischfußball. 
Das etwa 1,1 m lange und 0,7 m breite Spielfeld ist im Tischkorpus, der auf vier Beinen steht, eingelassen und von Banden umrundet. An den Stirnseiten sind die Tore platziert, die mit einer Torauffangschale oder einem Ballrücklauf ausgestattet sind.  
Die jeweils 11 Spielfiguren sind auf 4 Stangen (für drei Spielbereiche) verteilt:
\begin{itemize}  
\item die Torwartstange, auch Torwart (\gls{abwehr}) 
\item die 2er-Stange, auch die Zwei (\gls{abwehr}) 
\item die 5er-Stange, auch die Fünf (\gls{mittelfeld})
\item die 3er-Stange (\gls{sturm})
\end{itemize}  
Die Stangen können mittels Griffen vor- und zurückgedreht und rein und rausgeschoben werden.
Wenn man am Tisch steht, ist die Spielrichtung von links nach rechts, also das linke Tor iist das eigene und das rechte Tor das vom Gegner.
Neben diesem Grundaufbau gibt es viele Unterschiede zwischen den vielen \nameref{tisch:tisch:modelle}n. 

\subsection{Tischmodelle}
\label{tisch:tisch:modelle}

Inzwischen gibt es Tischmodelle vieler Hersteller in allen Qualitäts- und Preis-Kategorien:
\begin{itemize}
\item günstige, aber auch billige Tische gibt es ab etwa 100 Euro 
\item qualitiative Tische für Hobbyspieler gibt es ab 300-600 Euro
\item Trainingstische für Turnierspieler gibt es ab 600-1200 Euro
\item offizieller Turniertische gibt es ab 1200 Euro 
\end{itemize}
Letztendlich sollte der Tisch dem Spielniveau angemessen sein, damit der Spielspaß bewahrt wird. Dennoch gibt es ein paar Grundsätze, die man beispielsweise bei einem Tischkauf beachten sollte:
\begin{itemize}
\item Der Korpus sollte ein gewisses Gewicht haben, damit er nicht leicht verrutscht. Turniertische wiegen über 100 kg.
\item Die Beine sollte nicht wackeln.
\item Die Stangen sollten sich nicht leicht verbiegen lassen.
\item {\bf Empfehlung:} Bei einer Anschaffung eines Tisches und von Bällen sollte man darauf achten, dass ein griffiges Ball-Handling von Vorteil für das Erlernen von einem kontrollierten Spiel ist.
\end{itemize}

In Tabelle \ref{tab:tische} werden die Merkmale der 5 offiziellen Tischmodelle des \gls{itsf} und der zwei weiteren Partnertische des \gls{dtfb} verglichen. In der Tabelle wird Bezug auf das jeweilige offizielle Turniermodell genommen, jedoch hat jeder dieser Tischhersteller Modelle für das Anfänger-, Jugend- oder Hobbyspieler-Niveau.    

{\small
\begin{center} 
\begin{table} 
\begin{tabular}{ p{1.5cm}||p{2cm}|p{2cm}|p{2cm}|p{2cm}|p{2cm}} 
 	& Figuren & Spielfläche & Tore & Griffe & Region \\ 
\hline 
\hline 
Leonhart (DTFB, ITSF) & Soccer (Plastik) & hart (Plastik) & ?20,5 cm breit & rund (Gummi) & Deutschland und Nachbarländer \\ 
\hline 
Ullrich  (DTFB) &  Soccer (Plastik) &  hart (Plastik) & 20,5 cm breit & 10-kantig (Gummi) & Deutschland \\ 
\hline 
Lettner (DTFB)  & Soccer (Plastik)  &  hart (Plastik) & ??? & ??? & Deutschland \\ 
\hline 
Bonzini (DTFB, ITSF)  & schwerer Fuss (Metall) & weich (Linoleum) & ??? & keilförmig (Plastik), wechselbar & Frankreich und Nord-Europa \\ 
\hline 
Garlando (ITSF)  & schmal, Soccer-ähnlich (Plastik) &  hart (Glass) & ??? & rund (Plastik und Holz) & Österreich und Südost-Europa \\ 
\hline 
Roberto (ITSF) & quaderförmig (Plastik) &  hart (Plastik) & ??? & rund (Gummi) & Italien und Südost-Europa \\ 
\hline 
Tornado (ITSF)  & keilförmig (Plastik) &  hart (Plastik) & ??? & 6-kantig (Holz oder Gummi) & Nordamerika und englischsprachige Länder \\ 
\end{tabular} 
\caption{Offizielle \gls{dtfb} und \gls{itsf} Tische mit ihren Eigenheiten.}
\label{tab:tische}
\end{table} 
\end{center}
}



%%%%%%%%%%%%%%%%%%%%%%%%%%%%%%%%%%%%%%%%%%%%%%
\subsection{Standort}
\label{tisch:tisch:standort}

Maße und 120 cm lang und 70 cm breit benötigter Platz

Tisch ausrichten: Beine sind Höhenverstellbar

%%%%%%%%%%%%%%%%%%%%%%%%%%%%%%%%%%%%%%%%%%%%%%
\subsection{Wartung und Pflege}
\label{tisch:tisch:wartung}

Stangen schmieren: Pronto oder Silikonöl
Reinigen: Spielfeld (Glasreininger), Lager
(Spiel-)zubehör 

%%%%%%%%%%%%%%%%%%%%%%%%%%%%%%%%%%%%%%%%%%%%%%
%%%%%%%%%%%%%%%%%%%%%%%%%%%%%%%%%%%%%%%%%%%%%%
\section{Bälle}
\label{tisch:baelle}



%%%%%%%%%%%%%%%%%%%%%%%%%%%%%%%%%%%%%%%%%%%%%%
\section{Zubehör}
\label{tisch:zubehoer}


\subsection{Griffbänder und Handschuhe}
\label{tisch:zubehoer:griffe}



Hintergrund: Griffwechselsystem

\subsection{Trainingsmaterialien}
\label{tisch:zubehoer:training}
Rodlocks/Gummis,  

\subsection{Kleidung und Schuhe}
\label{tisch:zubehoer:kleidung}
Kleidung: Trikots, Schuhe, ...
