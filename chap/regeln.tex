\chapter{Regeln}
\label{regeln}

Wie bei jedem Spiel und Sport sind die Regeln das entscheidende, wie ein Spiel verlaufen darf und kann und auch entwickeln kann. 
Die internationalen Tischfußballregeln (\nameref{regeln:itsf}) sind umfangreich, daher werden zunächst vereinfachte Regeln präsentiert (\nameref{regeln:grundlagen}), um die Regelkunde mit dem fortwährenden Tischfußball-Spielen zu erlernen (\nameref{regeln:regelfahrplan}).   

\section{Grundregelwerk}
\label{regeln:grundlagen}

Ein angemessenes Grundregelwerk für Kinder- und Jugendgruppen besteht aus folgenden Regeln \citep{www:hakitu}:
\begin{itemize}
\item Wir geben vor und nach jedem Spiel den Gegnern die Hand.
\item Wir kurbeln nicht.
\item Wir spielen bis ein Team 6 Tore hat.
\item Wir dürfen auch von der Mittellinie Tore schießen.
\item Wir überlassen nach jedem Tor dem Gegner den Ball an der Mittellinie.
\end{itemize}

Die Kurzversion des ITSF ist etwas genauer \cite{itsf_basic_rules}:
\begin{enumerate}
\item Losen:
Der Gewinner der Losung – z.B. durch 
Münzwurf – darf sich für den 
ANSTOSS oder die SEITENWAHL 
entscheiden.
\item Anstoß:
Der Ball wird zur mittleren Figur der 
5er-Reihe gelegt. Der Gegner wird 
gefragt, ob dieser „Fertig?“ ist. Wenn 
der Gegner mit „Fertig!“ antwortet, 
muss der Ball zu einer anderen Figur 
auf der 5er-Reihe gespielt werden. 
Dann beginnt das Spiel.
\item Ball im Aus:
Wenn der Ball durch einen Schuss 
das Spielfeld verlässt, bekommt der 
Gegner den Ball auf die 2er-Reihe um 
weiterzuspielen.
\item Time Out (Auszeit):
Jedes Team darf pro Satz zwei Time 
Outs nehmen. Ein Time Out dauert 30 
Sekunden. Ist der Ball im Spiel, darf 
nur das Team, das im Ballbesitz ist, 
ein Time Out nehmen. Ist der Ball 
nicht im Spiel, dürfen beide Teams 
ein Time Out nehmen.
\item Ball abgeben:
Nach jeder Spielunterbrechung muss 
der Ball mindestens zwei Figuren der 
gleichen Reihe berühren (“einmal 
abgeben“), bevor der Ball zu einer 
anderen Reihe gepasst wird.
\item Seitenwechsel:
Nach jedem Satz dürfen die 
Spieler/Teams die Seite wechseln.
\item Unerlaubtes Kurbeln:
Kurbeln, bei dem die Stange sich 
mehr als 360~$\circ$ ohne Ballberührung 
berührt, ist ein Foul. Bei einem Foul 
bringt der Gegner den Ball auf der 
5er-Reihe wieder ins Spiel.
\item Tor:
Ein Tor kann mit jeder Reihe und 
Figur erzielt werden. Ein Ball, der ins 
Tor geht und direkt wieder 
rauskommt, zählt als Tor.
\item Fair Play:
Jedes Bewegen, Schieben oder 
Anheben des Tisches ist verboten. Es 
ist auch verboten während des 
Spiels, in den Tisch hineinzugreifen 
oder die Stangen zu berühren.
\item Zeitlimtit:
Maximal 10 Sekunden darf der Ball 
auf der Fünferreihe geführt werden. 
Maximal 15 Sekunden sind es auf 
Abwehrreihe (Torwart- und 2er-
Stange) und Sturmreihe (3er-
Stange). Bei einer Zeitüberschreitung 
auf der 3er-Reihe erhält der Gegner 
den Ball auf die 2er-Reihe. Sonst gibt 
es Auflagerecht auf der gegnerischen 
5er-Reihe.
\end{enumerate}


\section{Regelfahrplan}
\label{regeln:regelfahrplan}

Mögliche Regelanpassung zur schrittweisen Hinführung zum ITSF Regelwerk sind 
\begin{itemize}
\item Anstoss im Abwehrbereich, später Anstoss auf der 5er-Reihe
\item Erst ist der Einmannpass später, später Pass nur mit laufendem Ball
\end{itemize}

\section{ITSF Regelwerk}
\label{regeln:itsf}

Das ITSF Regelwerk man auf der \href{http://www.table-soccer.org/page/rules}{ITSF Homepage} in verschiedenen Sprachen.

