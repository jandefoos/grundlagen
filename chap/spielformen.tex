%%%%%%%%%%%%%%%%%%%%%%%%%%%%%%%%%%%%%%%
\chapter{Spielformen}
\label{spielformen}

Tabellarische Übersicht


%%%%%%%%%%%%%%%%%%%%%%%%%%%%%%%%%%%%%%%
%%%%%%%%%%%%%%%%%%%%%%%%%%%%%%%%%%%%%%%
\section{Zählweisen}
\label{spielformen:zaehlweisen}

Beim Tischfußball muss man Tore schießen und am besten mehr als sein Gegenspieler, um am Ende zu gewinnen.
Es gibt jedoch verschiedene Zählweisen, auf die man sich vor Beginn des Spiels einigt oder die bei einem Turnier gespielt werden. 
%%%%%%%%%%%%%%%%%%%%%%%%%%%%%%%%%%%%%%%
\subsection{Ein Satz}
\label{spielformen:zaehlweisen:einsatz}

Wenn man einfach so kickert ohne ein Turnier zu spielen, spielt man oft einen Satz bis 6 Tore: Wer zuerst 6 Tore geschossen hat, gewinnt die Partie.
Als wichtigste Variante gilt die Unentschiedenregel: Bei 5 zu 5 Toren geht der Satz unentschieden aus.

Obwohl bei vielen Tischmodellen die Torzähler 10 Tore zählen können, spielen die meisten Spieler bis 6 Tore -- aber auch ein Satz bis 5 oder 7 Toren sind verbreitet.

%%%%%%%%%%%%%%%%%%%%%%%%%%%%%%%%%%%%%%%
\subsection{Mehrere Gewinnsätze}
\label{spielformen:zaehlweisen:gewinnsaetze}

Wie beim Tennis, spielt man bei Tischfußballpartien über mehrere Gewinnsätze -- das heißt, wer zuerst eine festgelegte Anzahl von Sätzen gewinnt, gewinnt die Partie. Am gängigsten sind:
\begin{itemize}
\item Zwei Gewinnsätze (engl. "Best of three"): Derjenige, der zuerst zwei Sätze gewinnt, gewinnt die Partie. Mit 2:0 Sätzen hat man also gewonnen. 
Es können maximal drei Sätze gespielt werden: Falls es nach den ersten beiden Sätzen 1:1 steht, gibt es einen dritten entscheidenden Satz, und die Partie geht 1:2 oder 2:1 aus. 
\item Drei Gewinnsätze (engl. "Best of five"): Derjenige, der zuerst drei Sätze gewinnt, gewinnt die Partie. Steht es 2:2 nach den ersten vier Sätzen, gibt es einen fünften und entscheidenden Satz. 
\end{itemize}
Die Sätze werden hier bis 5 Tore gespielt. Beim Entscheidungsatz, also Satz 3 bei zwei Gewinnsätzen oder Satz 5 bei drei Gewinnsätzen, gibt es eine Verlängerung, wenn 4:4 Tore steht: Jetzt braucht man zwei Tore Vorsprung, um den letzten Satz und damit die Partie zu gewinnen. Bei drei Gewinnsätzen kann einen knappe Partie zum Beispiel 4:5, 5:4, 4:5, 5:4 und 8:6 in Toren und damit 3:2 in Sätzen ausgeht. 

Mehrere Gewinnsätze werden oft bei Turnieren in der KO-Runde gespielt. Hierbei setzt sich am Ende meist das bessere Team durch.




%%%%%%%%%%%%%%%%%%%%%%%%%%%%%%%%%%%%%%%
%%%%%%%%%%%%%%%%%%%%%%%%%%%%%%%%%%%%%%%
\section{Spiele mit unterschiedlich vielen Personen}
\label{spielformen:npersonen}

Neben den zwei Grundkategorien \nameref{spielformen:npersonen:einzel} und \nameref{spielformen:npersonen:doppel} und dem \nameref{spielformen:npersonen:team}, gibt es auch die Möglichkeit mit drei, fünf, sechs, sieben, acht oder sogar mehr Personen an einem Tisch zu kickern.
Damit kann man ein Training oder ein Turnier so gestalten, dass möglichst alle Teilnehmer viel spielen können. 


%%%%%%%%%%%%%%%%%%%%%%%%%%%%%%%%%%%%%%%
\subsection{Einzel}
\label{spielformen:npersonen:einzel}

\begin{itemize}
\item Benötigte Spieler: 2
\item Aufstellung: Eins gegen eins; jeder bedient mit seinen zwei Händen seine vier Stangen; ein Ball.
\item Regeln: Normale Regeln 
\end{itemize}
 
%%%%%%%%%%%%%%%%%%%%%%%%%%%%%%%%%%%%%%%
\subsection{Doppel}
\label{spielformen:npersonen:doppel}

\begin{itemize}
\item Benötigte Spieler: 4
\item Aufstellung: Zwei gegen zwei; jedes Team hat einen Torwart, der die 1er- und 2er-Stange bedient, und einem Stürmer, der die 5er- und 3er-Stange bedient; ein Ball.
\item Regeln: Normale Regeln 
\end{itemize}

%%%%%%%%%%%%%%%%%%%%%%%%%%%%%%%%%%%%%%%
\subsection{Teammodus}
\label{spielformen:npersonen:team}

\begin{itemize}
\item Benötigte Spieler: 2mal mindestens 3 Spieler
\item Spielplan: Es gibt einen Spielplan mit Einzel- und Doppelpartien, die typerweise auf einen Satz mit Unentschieden
\item Regeln: Normale Regeln 
\end{itemize}
 
Team (2mal 3+)

%%%%%%%%%%%%%%%%%%%%%%%%%%%%%%%%%%%%%%%
\subsection{Drei gegen drei (6 Personen)}
\label{spielformen:npersonen:dreigegendrei}
3:3 (6)

%%%%%%%%%%%%%%%%%%%%%%%%%%%%%%%%%%%%%%%
\subsection{Vier gegen vier (8 Personen)}
\label{spielformen:npersonen:viergegenvier}

%%%%%%%%%%%%%%%%%%%%%%%%%%%%%%%%%%%%%%%
\subsection{Zwei gegen einen (3 Personen)}
\label{spielformen:npersonen:zweigegeneinen}

%%%%%%%%%%%%%%%%%%%%%%%%%%%%%%%%%%%%%%
\subsection{Jeder mit und gegen jeden (3 Personen)}
\label{spielformen:npersonen:jedergegenjeden}

%%%%%%%%%%%%%%%%%%%%%%%%%%%%%%%%%%%%%%
\subsection{Runde (4 und mehr Personen)}
\label{spielformen:npersonen:runde}





%%%%%%%%%%%%%%%%%%%%%%%%%%%%%%%%%%%%%%%
\subsection{Alleine am Tisch}
\label{spielformen:npersonen:alleine}

Wenn man mal alleine am Tisch steht, kann man eine ganze Menge ganz in Ruhe üben oder ausprobieren:
\begin{itemize}
\item Ballführung: Man spielt sich den Ball zwischen den Puppen hin und her, entweder durch Tic-Tac oder Klemmen des Balls oder durch Spielen an die Bande 
\item Passen:
\item Auf das Tor schießen: 
\end{itemize}
Diese Übungen kann



%%%%%%%%%%%%%%%%%%%%%%%%%%%%%%%%%%%%%%%%
%%%%%%%%%%%%%%%%%%%%%%%%%%%%%%%%%%%%%%%%
%%%%%%%%%%%%%%%%%%%%%%%%%%%%%%%%%%%%%%%%
\section{Spiele mit Sonderregeln}
\label{spielformen:sonderregeln}

Hier findet ihr eine Sammlung an Spielarten, bei denen die normalen Regeln durch Sonderregeln ergänzt werden. 
Dadurch lernt man mit Spaß und spielerisch bestimmte Trainingsinhalte, da die Sonderregeln einen bestimmte Fokus setzen. 

%%%%%%%%%%%%%%%%%%%%%%%%%%%%%%%%%%%%%%%%
\subsection{Goalie-Einzel}
\label{spielformen:sonderregeln:goalie}

\begin{itemize}
\item Benötigte Spieler: 2
\item Niveau: Anfänger
\item Aufstellung: 3er und 5er Stangen werden hochgeklappt und fixiert; ein Ball.
\item Regeln: Normale Regeln plus Sonderregel des Torwartbereichs: 
  \begin{itemize}
  \item Anstoß im Torwartbereich (siehe \nameref{spielformen:sonderregeln:abstoss}).
  \item Sobald der Ball den Torwartbereich verläßt und z.B. im mittleren Spielfeldbereich unerreichbar liegen bleibt, bekommt der Gegner den Ball und bringt den Ball neu ins Spiel.
  \item Springt der Ball aus dem Tisch, bekommt wie gewohnt derjenigen den Ball, der nicht rausgeschossen hat. 
  \end{itemize}
\item Trainingseffekt: defensive und offensive Grundlagen; Torwartdasein, wie Stellungsspiel, Ball stoppen und behalten; Ballführung und Spielfeldverständnis
\item Variationen: \nameref{spielformen:sonderregeln:mehrerebaelle}, \nameref{spielformen:sonderregeln:unterschiedlichebaelle}, \nameref{spielformen:sonderregeln:onetouch}, \nameref{spielformen:sonderregeln:speedball}
\item Info: Goalie-Einzel wird bei vielen großen Turnieren als Nebendisziplin angeboten. Also selbst die Profis haben viel Spaß bei dieser Spielform.
\end{itemize}


%%%%%%%%%%%%%%%%%%%%%%%%%%%%%%%%%%%%%%%%
\subsection{Abstoß}
\label{spielformen:sonderregeln:abstoss}

\begin{itemize}
\item Benötigte Spieler: 2 oder 4 
\item Niveau: Anfänger
\item Aufstellung: Einzel-/Doppelaufstellung, ein Ball.
\item Regeln: Normale Regeln, aber der Anstoß wird im Torwartbereich ausgeführt. 
\item Trainingseffekt: Vereinfachen des Spiels durch Weglassen des Anstosses auf der 5er-Reihe; Aufwertung des Torwarts im Doppel
%\item Variationen: \nameref{spielformen:sonderregeln:dreigegendrei}, \nameref{spielformen:sonderregeln:unterschiedlichebaelle}, \nameref{spielformen:sonderregeln:onetouch}, \nameref{spielformen:sonderregeln:speedball}
\end{itemize}

%%%%%%%%%%%%%%%%%%%%%%%%%%%%%%%%%%%%%%%%
\subsection{Mehrere Bälle}
\label{spielformen:sonderregeln:unterschiedlichebaelle}
Rollerball

%%%%%%%%%%%%%%%%%%%%%%%%%%%%%%%%%%%%%%%%
\subsection{Unterschiedliche Bälle}
\label{spielformen:sonderregeln:mehrerebaelle}
Rollerball


%%%%%%%%%%%%%%%%%%%%%%%%%%%%%%%%%%%%%%%%
\subsection{Elfmeterschießen}
\label{spielformen:sonderregeln:elfemeter}
Torwartduell (Goalie War) anstelle von Einzel (2)

%%%%%%%%%%%%%%%%%%%%%%%%%%%%%%%%%%%%%%%%
\subsection{One touch}
\label{spielformen:sonderregeln:onetouch}
Sofort schießen

%%%%%%%%%%%%%%%%%%%%%%%%%%%%%%%%%%%%%%%%
\subsection{Speedball}
\label{spielformen:sonderregeln:speedball}

\begin{itemize}
\item Benötigte Spieler: 2 oder 4 
\item Aufstellung: Einzel-/Doppelaufstellung, ein Ball.
\item Regeln: Man darf den Ball maximal drei Sekunden auf einer Stange halten, aber
nicht stoppen! Also direkt spielen oder innerhalb einer Stange passen und sofort
weiterspielen.
\item Trainingseffekt: Reaktion, Überraschungsschüsse, Spaß, Kreativität
\item Varianten: Speedball geht auch in den Varianten 3 gegen 3 oder 4 gegen 4.  
\item Info: In Italien gibt es viele Turniere mit Speedball-Regeln. Bei der Weltmeisterschaft 2015 in Turin wurde Speedball als Nebendisziplin ausgetragen. 
\end{itemize}

%%%%%%%%%%%%%%%%%%%%%%%%%%%%%%%%%%%%%%%%
\subsection{Nur eine Technik}
\label{spielformen:sonderregeln:nureinetechnik}



%%%%%%%%%%%%%%%%%%%%%%%%%%%%%%%%%%%%%%%%
\subsection{Pass-Bonus}
\label{spielformen:sonderregeln:passbonus}
Einzel und Doppel; Punktvergabe für Pässe

%%%%%%%%%%%%%%%%%%%%%%%%%%%%%%%%%%%%%%%%
\subsection{Zeit lassen}
\label{spielformen:sonderregeln:zeitlassen}
mind. 10 Sekunden 

%%%%%%%%%%%%%%%%%%%%%%%%%%%%%%%%%%%%%%%%
\subsection{Weitere Spiele}
\label{spielformen:sonderregeln:weiteres}

Rundschlagen der Stangen

%%%%%%%%%%%%%%%%%%%%%%%%%%%%%%%%%%%%%%%%
%%%%%%%%%%%%%%%%%%%%%%%%%%%%%%%%%%%%%%%%
%%%%%%%%%%%%%%%%%%%%%%%%%%%%%%%%%%%%%%%%
\section{Spielorganisation}

\subsection{Forderspiel (1 Tisch)}

Variante:

\subsection{Auf Zeit (1 Tisch)}

\subsection{DYP-Varianten}

