\chapter{Technik und Taktik}
\label{technik}

\section{Motivation: Mit Tricks zum Profi}
\label{technik:motivation}

\begin{itemize}
\item Tricks sind kontrolliertes Spiel
\item Ballkontrolle, Passen
\item Schießen, Systeme
\item Verteidigen
\end{itemize}

\section{Körper- und Griffhaltung}
\label{technik:haltung}

\begin{itemize}
\item Körperhaltung (Schulterbreiter Stand, nicht auf den Griffen abstützen, sich nicht im weg stehen, vor den Stangen)
\item Hand/Griffhaltung
\end{itemize}

\section{Offensive -- mit Ball}
\label{technik:offensive}


Grundsätzliches zum Erlernen der \gls{offensive}: ,,Aktion'', altersgerechtes Training,  für Kinder und Jugendliche: weniger statisches Techniktraining, mehr spielerisches Erlernen, siehe Spielformen,

GRAFIK: mit Bereichen der Spielfiguren.

Verweise: Ungeblogt / Youtube / Sammelwebseite (DTFJ)

\begin{itemize}
\item Ballführung (auf einer Stange): 
\begin{itemize}
\item Ruhender Ball und Puppenwechsel.
\item Tictac oder Vorne-hinten Klemmen.
\item Auf der 2er-Stange zwischen der 1. und der 2. Puppe.
\item Auf der 3er-Stange zwischen der 1. und der 2. Puppe oder der 2. und der 3.Puppe oder mit der 1. und 3. Puppe.
\item Auf der 5er-Stange zwischen der zwei benachbarten Puppen oder mit zwei Puppen und dabei eine Puppe auslassen.
\end{itemize}
\item Passen (zwischen zwei Stangen):
\begin{itemize}
\item Ins Feld, gerade oder an die Bande.
\item Kanten- oder Brushpass.
\item Ballanahme.
\item Von der 5er- auf die 3er-Stange.
\item Von der 2er-Stange auf die 3er-Stange.
\item Von der 2er-Stange auf die 5er-Stange.
\end{itemize}
\item Torschüsse: 
\begin{itemize}
\item Schieber, Zieher, Abroller oder Jet.
\item Von der 3er-Stange oder der 2er-Reihe.
\item Rechts- oder Linksschuss.
\item Kurze oder lange Schüsse.
\item Trickschüsse
\end{itemize}
\end{itemize}

\section{Defensive -- ohne Ball}
\label{technik:defensive}

\subsection{Grundsätzliches} 
\begin{itemize}
\item Stellungsspiel: mit den Figuren die direkten Ballwege zum Tor blockieren 
\item ,,Re-aktion'': Auf den Ball reagieren,  die Laufban des Balls verfolgen 
\item ,,Aktion'':  
\end{itemize}


Technik:
Bälle blocken und fangen

Taktik:

Grundsätzliches:

\begin{itemize}
\item Antizipieren: Was kann passieren?
\item Gegner lesen: Was hat der Gegner vor? (Spielart, Spielniveau, Lieblingsschuss, Entscheidung, ...)
\end{itemize}


\begin{itemize}
\item Ballbesitz des Gegners im Abwehrbereich 
\begin{itemize}
\item Deckung als Stürmer
\item Deckung als Torwart: (statisch) im kurzen/langen Eck
\item Deckung im Doppel
\item Deckung im Einzel
\end{itemize}
\item Ballbesitz des Gegners im Mittelfeld (5er-Reihe)
\begin{itemize}
\item Deckung als Stürmer (5er-Reihe): Pass verhindern, kleinster Fahrbereich, an die Bande fahren
\item Deckung als Torwart: Torschüße, direkter Ballweg
\end{itemize}
\item Ballbesitz des Gegners im Sturm (3er-Reihe)
\begin{itemize}
\item Deckung als Torwart: Reaktion, fahren, shaken, Wechseln
\end{itemize}
\end{itemize}



