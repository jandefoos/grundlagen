\chapter{Technik}
\label{technik}

\section{Motivation: Mit Tricks zum Profi}
\label{technik:motivation}

\begin{itemize}
\item Tricks sind kontrolliertes Spiel
\item Ballkontrolle, Passen
\item Schießen, Systeme
\item Verteidigen und Bälle fangen
\end{itemize}

\section{Körper- und Griffhaltung}
\label{technik:haltung}

\begin{itemize}
\item Körperhaltung (Schulterbreiter Stand, nicht auf den Griffen abstützen, sich nicht im weg stehen, vor den Stangen)
\item Hand/Griffhaltung
\end{itemize}




\begin{itemize}
\item Figur zum Ball: Spielen und annehmen
\item Klemmen (Puppenwechsel)
\item Tic-Tac
\end{itemize}


\section{Ballkontrolle}
\label{technik:ballkontrolle}

GRAFIK: mit Bereichen der Spielfiguren.

Ballgefühl und Dribbling

Grundsätzliches zum Erlernen der \gls{offensive}: ,,Aktion'', altersgerechtes Training,  für Kinder und Jugendliche: weniger statisches Techniktraining, mehr spielerisches Erlernen, siehe Spielformen,


\subsection{Ballführung (auf einer Stange)} 
\label{technik:ballkontrolle:eine}

\begin{itemize}
\item Ruhender Ball und Puppenwechsel.
\item Tictac oder Vorne-hinten Klemmen.
\item Auf der 2er-Stange zwischen der 1. und der 2. Puppe.
\item Auf der 3er-Stange zwischen der 1. und der 2. Puppe oder der 2. und der 3.Puppe oder mit der 1. und 3. Puppe.
\item Auf der 5er-Stange zwischen der zwei benachbarten Puppen oder mit zwei Puppen und dabei eine Puppe auslassen.
\end{itemize}


\subsection{Passen (zwischen zwei Stangen)}
\label{technik:ballkontrolle:zwei}

\begin{itemize}
\item Spielprinzipien: Ins Feld, gerade oder an die Bande.
\item Passtechniken: Kanten-, Stick und Brushpass. 
\href{http://ungeblogtkickern.blogspot.de/2015/09/schrag-schieen.html}{Artikel auf Ungeblogt}
\item Annahmetechniken: Ballanahme.
\end{itemize}

Passmöglichkeiten:
\begin{itemize}
\item Von der 5er- auf die 3er-Stange.
\item Von der 2er-Stange auf die 3er-Stange.
\item Von der 2er-Stange auf die 5er-Stange.
\end{itemize}


\section{Torschüsse}
\label{technik:torschuesse}

\begin{itemize}
\item Schussprinzipien: Schneller als der Gegner, Seitwärtsbewegung und Schussbewegung. Für Fortgeschrittene: Geschwindigkeit, Präzision, Konstanz, Abrufbarkeit
\item Schusstechniken: Schieber/Zieher, \href{http://ungeblogtkickern.blogspot.de/2014/07/schritt-fur-schritt-pin-schieen.html}{Abroller/Pin} oder Jet.
\item Von der 3er-Stange oder der 2er-Reihe.
\item Trickschüsse
\end{itemize}





\section{Defensive}
\label{technik:defensive}

\begin{itemize}
\item Defensiv-Prinzip: mit den Figuren die direkten Ballwege zum Tor blockieren (Seitwärtsbewegung)  
\item Klapprichtung der Figuren und Abstand der Figuren (Drehbewegung)
\item Bälle blocken, fangen, annehmen 
\end{itemize}
Daraus folgen verschiedene Stellungsspiele.


\subsection{Ballbesitz des Gegners im Abwehrbereich}
\label{technik:defensive:gegnerabwehr}

\begin{itemize}
\item Deckung als Stürmer
\item Deckung als Torwart: (statisch) im kurzen/langen Eck
\item Deckung im Doppel
\item Deckung im Einzel
\end{itemize}


\subsection{Ballbesitz des Gegners im Mittelfeld (5er-Reihe)}
\label{technik:defensive:gegnerabwehr}

\begin{itemize}
\item Deckung als Stürmer (5er-Reihe): Pass verhindern, kleinster Fahrbereich, an die Bande fahren
\item Deckung als Torwart: Torschüße, direkter Ballweg
\end{itemize}


\subsection{Ballbesitz des Gegners im Sturm (3er-Reihe)}
\label{technik:defensive:gegnerabwehr}

\begin{itemize}
\item Deckung als Torwart: Reaktion, fahren, shaken, Wechseln
\end{itemize}

