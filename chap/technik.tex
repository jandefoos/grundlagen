\chapter{Technik und Taktik}
\label{technik}

\section{Motivation: Mit Tricks zum Profi}
\label{technik:motivation}
Tricks sind kontrolliertes Spiel
Passen
Schießen
Systeme

\section{Körper- und Griffhaltung}
\label{technik:haltung}
Körperhaltung (Schulterbreiter Stand, nicht auf den Griffen abstützen, sich nicht im weg stehen, vor den Stangen)
Hand/Griffhaltung

\section{Offensive -- mit Ball}
\label{technik:offensive}


\gls{offensive}

allgemein: altersgerechtes Training, siehe Spielformen
Ballführung auf einer Stange
Ballführung zwischen unterschiedlichen Stangen (Passen, Annehmen)
Schießen
Handgelenk
Zieher/Schieber
Abroller
Jet
Verweise: Ungeblogt / Youtube / Sammelwebseite (DTFJ)

\begin{itemize}
\item Torschüsse: 
\begin{itemize}
\item Schieber, Zieher, Abroller oder Jet.
\item Von der 3er-Stange oder der 2er-Reihe.
\item Rechts- oder Linksschuss.
\item Kurze oder lange Schüsse.
\item Trickschüsse
\end{itemize}
\item Passen:
\begin{itemize}
\item Ins Feld, gerade oder an die Bande.
\item Kanten- oder Brushpass.
\item Von der 5er- auf die 3er-Stange.
\item Von der 2er-Stange auf die 3er-Stange.
\item Von der 2er-Stange auf die 5er-Stange.
\end{itemize}
\item Ballführung: 
\begin{itemize}
\item Ruhender Ball und Puppenwechsel.
\item Tictac oder Vorne-hinten Klemmen.
\item Auf der 2er-Stange zwischen der 1. und der 2. Puppe.
\item Auf der 3er-Stange zwischen der 1. und der 2. Puppe oder der 2. und der 3.Puppe oder mit der 1. und 3. Puppe.
\item Auf der 5er-Stange zwischen der zwei benachbarten Puppen oder mit zwei Puppen und dabei eine Puppe auslassen.
\end{itemize}
\end{itemize}

\section{Defensive -- ohne Ball}
\label{technik:defensive}


\gls{defensive}
Deckung als Torwart
passive Deckungen: statisch im kurzen Eck, ...
aktive Deckungen: bewegen und den Lieblingsschuss des Gegners herausfinden, ...
Deckung als Stürmer:
passiv: 
aktiv: bewegen


