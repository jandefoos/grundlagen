\chapter{Standorte und Strukturen für Kinder- und Jugend-Tischfußball} 
\label{jugend}

Die sportliche Landschaft des Tischfußballs ist im Jahre 2016 in Deutschland weitestgehend durch Spielerinnen und Spieler des höhreren Jugendalters, also ab Anfang 20, geprägt. 
Dennoch gibt es auch Kinder und Jugendliche die das Tischfußball-Spiel und auch das auch als Sport ausüben. 
Diese Kapitel soll einen typischen Jugendstandort vorstellen und Empfehlungen geben (Kapitel \ref{jugend:standorte}). Es werden die Verbände kurz vorgestellt und auf deren Jugendarbeit hingewiesen (Kapitel \ref{jugend:kontakte}). 

\section{Standorte}
\label{jugend:standorte}

Im Fokus stehen Standorte, die noch nicht einer sportlichen Tischfußball-Struktur angegliedert sind, die allerdings einen Tisch in ihrer Einrichtung stehen haben. Das können zum Beispiel folgende typische Standorte sein: 
\begin{itemize}
\item in einer offenen Kinder- oder Jugendeinrichtung ("Jugendhaus")
\item an einer Schule
\item in einem Betrieb oder an einer Hoschschule
\item aber auch in einem Verein für eine Anfängergruppe  
\end{itemize}

\subsection{Rahmenbedingungen}
\label{jugend:standorte:rahmen}

Folgende Rahmenbedingungen werden betrachtet:
\begin{itemize}
\item Tischanzahl: 1-2 Tische
\item Organisation: 1-2 Betreuer/Trainer
\item Teilnehmer: 2-20 Teilnehmer 
\item Termin: einmal pro Woche je 1,5 bis 2 Stunden
\end{itemize}

\subsection{Programm}
\label{jugend:standorte:programm}

Das Tischfußballangebot könnte folgendes Programm haben: 
\begin{itemize}
\item Training (jede Woche): Trainingsturniere (Kapitel \ref{turniere}) in verschiedenen Spielformen (Kapitel \ref{spielformen}), aber auch Inhalte abseits des Tischs können für Abwechslung sorgen:

\begin{itemize}
\item Tischpflege: Gemeinsam kann der Tisch geputzt und ggf. repariert werden.  
\item Bewegungsangebot: Spiele wie Fussball oder Basketball fördern die motorischen Fähigkeit
\item Grillen: Gemeinsames Erlebnis 
\item Freundschaftsspiele: Gemeinsames Ausflug in ein andere Jugendhaus mit einem Tischfußball-Freundschaftsspiel
\item Teamname und Logo: Finden eines Teamnamens und Entwerfen eines Logos für Auswärtsspiele   
\item Profispiel anschauen: Ein Ausflug zu einem Turnier oder einer Ligapartie eines nahen Vereins, alternativ im Internet ein Video anschauen 
\end{itemize} 
  
\item Ranglistenturnier (monatlich): Alle vier Wochen kann man ein Turnier spielen für eine Rangliste. So kann die Spielstärke abgebildet werden.  
\item Jahresmeisterschaft (jährlich): Ebenfalls eine Jahresmeisterschaft kann ein besonderer Wettkampf sein. Man sollte auch an einer überregionalen Jugendhausmeisterchaft teilnehmen, falls es dieses Angebot am Standort gibt. 
\end{itemize}  

\subsection{Empfehlungen}
\label{jugend:standorte:empfehlungen}

Weitere allgemeine Empfehlungen zum Aufbau des Tischfußball-Angebots:
\begin{itemize}
\item Kicker-Gruppe im Jugendhaus: 
Kooperation mit einem standort-nahen Verein,
Teilnahme an der Jugendhausmeisterschaft,  
optionale Teilnahme Landesmeisterschaft und einer Landesliga
\item Kicker-AG in der Schule: 
Kooperation mit einem standort-nahen Verein, 
Teilnahme an der Schulmeisterschaft, 
optionale Teilnahme Landesmeisterschaft und einer Landesliga
\item Vereinstraining einer Anfängergruppe: 
Kooperation mit einem Jugendhaus oder einer Schule durch personelle Unterstüzung einer Kicker-Gruppe bzw. Kicker-AG, 
Teilnahme Landesmeisterschaft und einer Landesliga,
Teilnahme an der Bundesliga
\item Betriessportangebot: 
Kooperation mit Verein, 
Teilnahme an der Betriebssportmeisterschaft und Liga,
\end{itemize}


%%%%%%%%%%%%%%%%%%%%%%%%%%%%%%%%%%%%%%%%%

\section{Angebote der Verbände}
\label{jugend:kontakte}

Die Sportverbände fördern die Jugendarbeit können dabei folgende Hilfestellung für die lokale Jugendarbeit anbieten.

\subsection{DTFJ}
\label{jugend:kontakte:dtfj}

Die Deutsche Tischfußball Jugend (DTFJ) gehört dem DTFB an, und ist das Organ was für die nationale Jugendarbeit im Tischfußball eintritt.  
Folgende Zwecke soll die DTFJ leisten: 
\begin{itemize}
\item Förderung des kinder- und jugendgerechten Tischfußballsports im Rahmen der Leibesübungen nach besten Kräften zu pflegen, zu fördern und seinen ideellen Charakter zu wahren
\item Förderung der lokalen und regionalen Jugendarbeit im Bereich des Tischfußballsports
\item Die DTFJ verfolgt keine politischen Ziele und vertritt den Grundsatz religiöser und weltanschaulicher Toleranz
\end{itemize}
Folgende Angebote hat die DTFJ oder werden in Zukunft aufgebaut:
\begin{itemize}
\item Anerkennung für gelungene Jugendarbeit
\item Empfehlungen für kinder- und jugendgerechten Tischfußball 
\item Hilfestellungen zum Aufbau von Tischfußballangeboten in offenen Jugendeinrichtungen, Schulen, Ausbildungsstätten und Vereinen
\item Promoten von Jugendtischfußball im sozialen Netzwerk Facebook
\item Aufbau eines Ausbildungs-Programms für Sozialarbeiter und Pädagogen 
\item Aufbau einer Trainerausbildung
\item Erstellung einer Informations- und Kommunikationsplattform für 
Jugend-Tischfußball und Ausbildung
\item Aufbau von Kontakten zu und möglicher Zusammenarbeit mit Sportjugendverbänden und Schulverbänden
\end{itemize}

Weitere Details auf der \href{http://www.dtfj.de/}{DTFJ-Homepage}.

\subsection{Landesverband}
\label{jugend:kontakte:landesverband}

Der zuständige Landesverband für einen Standort kann folgende Dinge anbieten:
\begin{itemize}
\item In Zusammenarbeit mit den Vereinen DTFB Junioren-Challenger anbieten
\item Jugend-Landesmeisterschaften 
\item Hilfestellung für Jugendarbeit in Vereinen und anderen Jugendstandorten
\end{itemize}


\subsection{DTFB}
\label{jugend:kontakte:dtfb}

Der Deutsche Tischfußball-Bund fördert die Jugendarbeit mit folgenden Maßnahmen:
\begin{itemize}
\item Organisation von nationalen Jugendvergleichswettkämpfen: Zusammen mit den Landesverbänden DTFB Challenger und eine nationale Rangliste, Deutsche Meisterschaft im Einzel und Doppel und im Team (Bundesliga) 
\item Veröffentlichung der Spielerprofile und der Turnierergebnisse, sowie der aktuelle Stand der nationale Junioren-Rangliste 
\item Berufung eines Bundescoach, der den Kader für die Junioren-Nationalmannschaft zusammenstellt
\item Kaderlehrgänge des erweiterten Kaders der Nationalmannschaft (Talentförderprogramm)
\item Unterstützung der Bundesligateilnehmer und der Nationalspieler bei internationalen Wettbewerben 
\end{itemize}

Weitere Informationen auf der \href{http://www.dtfb.de/}{DTFB-Homepage}.

%\subsection{ITSF Europe}
%\label{jugend:kontakte:europa}
%European Championsleague

\subsection{ITSF}
\label{jugend:kontakte:itsf}

Der ITSF trägt internationale Turnier sowie Weltmeisterschaften aus, wo auch die besten nationalen "Unter 18" die Junior-Disziplinen ausspielen.
Zudem setzt sich die "Education Comission" für folgende Dinge ein:
\begin{itemize}
\item The Education Commission promotes table soccer as a means of increasing social cohesion and ties of friendship through its values: conviviality, respect, fair play, team spirit and access for all.Nationale Verbände
\item The commission encourages international development of table soccer by providing tools and training in partnership with a diverse range of organisations worldwide.e
\item Folgende Dokumente stehen in verschiedenen Sprachen auf der Homepage zur Verfügung:
\begin{itemize}
\item Basic Rules: Hier werden die wichtigsten ITSF Regeln übersichtlich zusammengefasst.
\item Basic Guide: Ein Grundlagen-Dokument für den Anfänger 
\item Passport: Der Passport ist eine Art Abzeichen-Heft, in das der Spieler seine Fähigkeiten festhalten kann, wenn er diese einem Prüfer präsentiert
\end{itemize}
\end{itemize}
Weitere Informationen auf der \href{http://www.table-soccer.org/}{ITSF-Homepage}.


%\section{Abzeichen}
% (Level, Altersklassen)
%Verbindung mit Regeln z.B. Gold darf keine Einmannpässe spielen...
%Handicaps im allgemeinen
