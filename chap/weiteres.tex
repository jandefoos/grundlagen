\chapter{Tischfußball-Gruppen}

\section{Kicker-Gruppe}

\subsection{Inhalte abseits des Tischs}

\begin{itemize}
\item Tischpflege:  
\item Bewegungsangebot: 
\item Grillen:
\item Feundschaftsspiele:
\item Teamname und Logo:
\item 
\end{itemize} 


\subsection{Programm}

Kicker-Gruppe im Jugendhaus, Kicker-AG in der Schule, Betriessportangebot, Vereinstraining einer Anfängergruppe  

Rahmenbedingungen
\begin{itemize}
\item Teilnehmer: 1-2 Betreuer/Trainer  
\item Teilnehmer: 2-20
\item Tischanzahl: 1-2
\item Termine pro Woche: 1-2mal je 1,5- 2 Stunden
\item Programm intern: 
Trainingspiele und Trainingsturniere, 
1mal im Monat ein Turnier,
Rangliste von Monatsturnier,  
Jahresturnier
\end{itemize}

Empfehlungen und externes Programm
\begin{itemize}
\item Kicker-Gruppe im Jugendhaus: 
Kooperation mit Verein,
Teilnahme an der Jugenhausmeisterschaft,  
optionale Teilnahme Landesmeisterschaft und einer Landesliga
\item Kicker-AG in der Schule: 
Kooperation mit Verein, 
Teilnahme an der Schulmeisterschaft, 
optionale Teilnahme Landesmeisterschaft und einer Landesliga
\item Vereinstraining einer Anfängergruppe: 
Kooperation mit einem Jugendhaus oder einer Schule durch personelle Unterstüzung einer Kicker-Gruppe bzw. Kicker-AG, 
Teilnahme Landesmeisterschaft und einer Landesliga/Bundesliga
\item Betriessportangebot: 
Kooperation mit Verein, 
Teilnahme an der Betriebssportmeisterschaft und Liga,

\end{itemize}


\section{Kontakte und Verbände}

\subsection{Landesverband}

Landesverbände und Vereine
alle gemeldeten Spieler
Turnierergebnisse von Ranglistenturnier 

Landesliga und Landesmeisterschaft

\subsection{DTFJ}

Idee und Ziele

Status

\subsection{DTFB}

Landesverbände und Vereine
alle gemeldeten Spieler
Turnierergebnisse von Ranglistenturnier 

Jugend:
U18 Nationalmannschaft mit Nationalcoach
Kader, erweiterter Kader und Sichtungskader
Jährlicher Kaderlehrgang

Jugendwart: Erster Ansprechpartner in Sachen Jugend 
und Kontaktperson zu den Landesjugendwärte

Deutsche Meisterschaft: Am Ende des Jahres meist im November.

Bundesliga: Herren, Damen, Senioren und Jugend. Die Juniorenbundesliga findet einmal im Jahr an einem Wochenende meist im September statt. 

\subsection{ETSF}

European Championsleague

\subsection{ITSF}

Nationale Verbände

Weltranglistenturniere

Weltmeisterschaften


%\section{Abzeichen}
% (Level, Altersklassen)
%Verbindung mit Regeln z.B. Gold darf keine Einmannpässe spielen...
%Handicaps im allgemeinen
